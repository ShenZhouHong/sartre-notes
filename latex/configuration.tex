% Mathematical typesetting packages
% \usepackage{amsmath}                % Needed for most math things.
% \usepackage{amssymb}                % Additional mathematical symbols
% \usepackage{amsthm}                 % For theorem and proof environments
% \usepackage{tkz-euclide}            % Used for planar geometry (Euclidean)

% Scientific graphics and plotting
% \usepackage{tikz}                   % Used for graphical illustrations.
% \usepackage{pgfplots}               % Used for scientific graphs and charts

% Packages for typesetting code and pseudocode
% In order to use minted, you must edit your makefile to -use-shell-escape!
% \usepackage{minted}                 % Code highlighting: \begin{minted}{python}
% \usepackage{algorithm}              % Float environment for pseudocode
% \usepackage{algpseudocode}          % Typesetting library for pseudocode

% Optional LaTeX packages for additional functionality
% \usepackage[noframe]{showframe}     % Debug option to show margin frames.
% \usepackage{float}                  % For arranging floats
% \usepackage{graphicx}               % Required for embedding images
% \usepackage{booktabs}               % For prettier tables
% \usepackage{geometry}               % Sets more "reasonable" margin-sizes
% \usepackage{xeCJK}                  % For typesetting CJK characters
% \usepackage[l2tabu, orthodox]{nag}  % Verbose warnings for typesetting

\usepackage{extramarks}
\usepackage{enumitem}               % Makes the lists more compact, saving paper.
\setlist{nosep}

\usepackage{marginnote}
\renewcommand*{\marginfont}{\scriptsize\slshape}

% Force margin notes to use (sloppy) justification
\renewcommand\raggedrightmarginnote{\sloppy}
\renewcommand\raggedleftmarginnote{\sloppy}

% Settings used by \usepackage{fancyhdr}
% Header Content

\fancypagestyle{fancy}{
  % Add a line to the footer as well as the header
  % \renewcommand\headrulewidth{0.4pt}
  % \fancyhead[L]{Senior Seminar}
  % % \chead{}
  % \fancyhead[R]{St. John's College}

  % Footer Content
  \fancyfoot[C]{Page \thepage\ of \pageref*{LastPage}}
}

% Update the plain heading format so that the first page includes page n of m
\fancypagestyle{plain}{
  \renewcommand\headrulewidth{0pt}
  \fancyhead{}
  \fancyfoot[C]{Page \thepage\ of \pageref*{LastPage}}
}

% The biblatex package should go last!
% FYI: the biblatex package is used to generate a bibliography. However, unless
% intense citation management (e.g. 5+ sources) is used, biblatex should not
% be loaded, as it slows down the document compilation process by almost 2X.
% In order to enable it, see sections/endpage.tex and essay-name.tex's preamble
%
\usepackage[style=mla, backend=biber]{biblatex} % Nice MLA bibliography
\addbibresource{citations.bib} % Biblatex. See includes/formatting.tex

\SetCiteCommand{\autocite}

% This actually loads hyperref, lol
\usepackage[a-3u, mathxmp]{pdfx}

% Settings used by \usepackage{hyperref}
% Used to setup hyperlinks, as well as to properly annotate PDF metadata
\hypersetup{
  colorlinks=true,
  allcolors=blue,
}
\urlstyle{sf}

% Any additional user configuration goes below
