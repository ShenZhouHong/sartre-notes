% LaTeX Generic essay template
% Copyright (C) 2017  Shen Zhou Hong <shen@hong.io>
%
% This program is free software: you can redistribute it and/or modify
% it under the terms of the GNU General Public License as published by
% the Free Software Foundation, either version 3 of the License, or
% (at your option) any later version.
%
% This program is distributed in the hope that it will be useful,
% but WITHOUT ANY WARRANTY; without even the implied warranty of
% MERCHANTABILITY or FITNESS FOR A PARTICULAR PURPOSE.  See the
% GNU General Public License for more details.
%
% You should have received a copy of the GNU General Public License
% along with this program.  If not, see <https://www.gnu.org/licenses/>.

% For more information on documentclass configuration, see:
% https://texblog.org/2013/02/13/latex-documentclass-options-illustrated/#formula
\documentclass[
  10pt,       % 10pt, 11pt, 12pt
  letterpaper,    % a4paper, letterpaper, a5paper, b5paper, executivepaper, legalpaper
  final,      % draft
  onecolumn,  % twocolumn
  oneside,    % twoside
  notitlepage % titlepage
]{article}

% Template preamble files. These should not be modified.
% Accessibility packages for PDF/A-3u Compliance. These come first!
\usepackage{colorprofiles}
\usepackage{mmap}


% Accessibility packages for PDF/A-3u Compliance. These come first!
\usepackage{colorprofiles}
\usepackage[a-3u, mathxmp]{pdfx}
\usepackage{mmap}

% General packages in use
\usepackage{geometry}
\usepackage[factor=2000]{microtype}
\usepackage{setspace}               % Used to change line and paragraph spacing
\usepackage{csquotes}               \MakeOuterQuote{"}
\usepackage[T1]{fontenc}            % Supports international fonts
\usepackage[osf]{libertine}         % Beautiful libertine font
\usepackage[parfill]{parskip}       % For empty lines between paragraphs
\usepackage{hyperref}               % So citations can be links
\usepackage[compact]{titlesec}      % Sets section titles to be more compact
\usepackage{titling}                % In order to make the title higher on page
\usepackage{xcolor}                 % To make URL links blue
\usepackage{fancyhdr}               % For fancy headings
\usepackage{lastpage}               % Gives us \lastpage
\usepackage{lipsum}                 % For lorem lipsum placeholder text


% Settings used by \usepackage{fancyhdr}
% Header Content
\pagestyle{fancyplain}
\lhead{Left Header}
\chead{Center Header}
\rhead{Right Header}
% Footer Content
\lfoot{Left Footer}
\cfoot{Page \thepage\ of \pageref*{LastPage}}
\rfoot{Right Footer}

% Settings used by \usepackage{hyperref}
% Used to setup hyperlinks, as well as to properly annotate PDF metadata
\hypersetup{
  colorlinks=true,
  linkcolor=blue,
  urlcolor=blue,
}
\urlstyle{sf}

% Settings used by \usepackage{setspace}
% These commands are used to change the line-spacing of the text.
% See documentation at https://tex.stackexchange.com/questions/65849/
% \doublespacing    % for double-spacing
% \linespread{1.25} % for custom line spacing.
\onehalfspacing   % for 1.5 spacing

% Settings used by \usepackage[parfill]{parskip}
% Paragraph indentation. Set to 0em to disable.
\setlength{\parindent}{3em}

% Settings used by \usepackage{titling}
% The following command is used to "hang" the default LaTeX title from the top
% of the page (e.g. like a painting frame). The default LaTeX title sits a bit
% too low, so the following command sets it higher.
\setlength{\droptitle}{-2em}

% Settings used by \usepackage{microtype}
% These custom hyphenation rules are used to discourage excessive hyphenation
\pretolerance=5000
\tolerance=9000
\emergencystretch=0pt
\righthyphenmin=4
\lefthyphenmin=4
\microtypecontext{spacing=nonfrench}

% Settings used by \usepackage{pgfplots}
% \pgfplotsset{compat=1.17}
% \pgfplotsset{samples=300}

% Settings used by titlesec
% For section, subsection, and subsubsection numberings to be in margin.
% Sourced from: https://tex.stackexchange.com/a/523014
\newcommand{\marginsecnumber}[1]{%
  \makebox[0pt][r]{#1\hspace{6pt}}%
}
\titleformat{\section}
  {\normalfont\Large\bfseries}
  {\marginsecnumber\thesection}
  {0pt}
  {}
\titleformat{\subsection}
  {\normalfont\large\bfseries}
  {\marginsecnumber\thesubsection}
  {0pt}
  {}
\titleformat{\subsubsection}
  {\normalfont\normalsize\bfseries}
  {\marginsecnumber\thesubsubsection}
  {0pt}
  {}
\titleformat{\paragraph}[runin]
  {\normalfont\normalsize\bfseries}
  {\marginsecnumber\theparagraph}
  {0pt}
  {}
\titleformat{\subparagraph}[runin]
  {\normalfont\normalsize\bfseries}
  {\marginsecnumber\thesubparagraph}
  {0pt}
  {}
\titlespacing*{\subsection}{0pt}{*3.25}{*1.5}%


% This code, courtesy of Marcel Krüger, is designed to use LuaLaTeX's scripting
% capabilities to fix protrusion (for hanging punctuation) inside enumerate and
% itemize environments.
% https://tex.stackexchange.com/questions/629068/microtype-quotation-marks-in-itemize-not-aligned-properly-at-begin-of-an-item/629080#629080

\directlua{
  local func = luatexbase.new_luafunction'betterprotrusionboundary'
  local my_whatsit = luatexbase.new_whatsit'betterprotrusionboundary'
  local whatsit_id = node.id'whatsit'
  local glyph_id = node.id'glyph'
  local user_defined = node.subtype'user_defined'
  token.set_lua('betterprotrusionboundary', func, 'protected')
  local modes = tex.getmodevalues()
  lua.get_functions_table()[func] = function()
    local mode = tex.nest.top.mode
    if mode < 0 then mode = -mode end
    if modes[mode] == 'vertical' then
    token.put_next(token.new(func, token.command_id'lua_call'))
      return tex.forcehmode()
    end
    local n = node.new(whatsit_id, user_defined)
    n.user_id = my_whatsit
    n.type = 100
    n.value = token.scan_int()
    node.write(n)
  end

  luatexbase.add_to_callback('pre_linebreak_filter', function(head)
    for n, s in node.traverse_id(whatsit_id, head) do if s == user_defined and n.user_id == my_whatsit then
      assert(n.value == 1, 'boundarytypes beside 1 not yet supported')
      if n.value & 1 == 1 then
        for nn, id in node.traverse(n.next) do
          local char, fid = node.is_glyph(nn)
          if char then
            token.put_next(token.create'lpcode', token.new(fid, token.command_id'set_font'), token.new(char, token.command_id'char_given'))
            local width = (font.getparameters(fid).quad or 0) * token.scan_int() // 1000
            if not (width == 0) then
              local kern = node.new('kern', 1)
              % local kern = node.new('margin_kern', 0)
              kern.kern = -width
              % kern.glyph = char
              head = node.insert_after(head, n, kern)
            end
            break
          elseif not node.protrusion_skippable(nn) then
            break
          elseif fid == whatsit_id and nn.subtype == user_defined and nn.user_id == my_whatsit and nn.value & 1 == 1 then
            break
          end
        end
      end
      if n.value & 2 == 2 then
        local nn = n.prev
        while nn do
          local char, fid = node.is_glyph(nn)
          if char then
            token.put_next(token.create'rpcode', token.new(fid, token.command_id'set_font'), token.new(char, token.command_id'char_given'))
            local width = (font.getparameters(fid).quad or 0) * token.scan_int() // 1000
            if not (width == 0) then
              local kern = node.new('kern', 1)
              % local kern = node.new('margin_kern', 1)
              kern.kern = -width
              % kern.glyph = char
              head = node.insert_before(head, n, kern)
            end
            break
          elseif not node.protrusion_skippable(nn) then
            break
          end
          nn = nn.prev
        end
      end
      head = node.remove(head, n)
    end end
    return head
  end, 'betterprotrusionboundary')
}

\renewcommand\leftprotrusion{\betterprotrusionboundary1\relax}


% Project-specific preamble files. Include additional packages here.
% Mathematical typesetting packages
% \usepackage{amsmath}                % Needed for most math things.
% \usepackage{amssymb}                % Additional mathematical symbols
% \usepackage{amsthm}                 % For theorem and proof environments
% \usepackage{tkz-euclide}            % Used for planar geometry (Euclidean)

% Scientific graphics and plotting
% \usepackage{tikz}                   % Used for graphical illustrations.
% \usepackage{pgfplots}               % Used for scientific graphs and charts

% Packages for typesetting code and pseudocode
% In order to use minted, you must edit your makefile to -use-shell-escape!
% \usepackage{minted}                 % Code highlighting: \begin{minted}{python}
% \usepackage{algorithm}              % Float environment for pseudocode
% \usepackage{algpseudocode}          % Typesetting library for pseudocode

% Optional LaTeX packages for additional functionality
% \usepackage[noframe]{showframe}     % Debug option to show margin frames.
% \usepackage{float}                  % For arranging floats
% \usepackage{graphicx}               % Required for embedding images
% \usepackage{booktabs}               % For prettier tables
% \usepackage{geometry}               % Sets more "reasonable" margin-sizes
% \usepackage{xeCJK}                  % For typesetting CJK characters
% \usepackage[l2tabu, orthodox]{nag}  % Verbose warnings for typesetting

\usepackage{extramarks}
\usepackage{enumitem}               % Makes the lists more compact, saving paper.
\setlist{nosep}

\usepackage{marginnote}
\renewcommand*{\marginfont}{\scriptsize\slshape}

% Force margin notes to use (sloppy) justification
\renewcommand\raggedrightmarginnote{\sloppy}
\renewcommand\raggedleftmarginnote{\sloppy}

% Settings used by \usepackage{fancyhdr}
% Header Content

\fancypagestyle{fancy}{
  % Add a line to the footer as well as the header
  \renewcommand\headrulewidth{0.4pt}
  \fancyhead[L]{Senior Seminar}
  % \chead{}
  \fancyhead[R]{St. John's College}

  % Footer Content
  \fancyfoot[C]{Page \thepage\ of \pageref*{LastPage}}
}

% Update the plain heading format so that the first page includes page n of m
\fancypagestyle{plain}{
  \renewcommand\headrulewidth{0pt}
  \fancyhead{}
  \fancyfoot[C]{Page \thepage\ of \pageref*{LastPage}}
}

% The biblatex package should go last!
% FYI: the biblatex package is used to generate a bibliography. However, unless
% intense citation management (e.g. 5+ sources) is used, biblatex should not
% be loaded, as it slows down the document compilation process by almost 2X.
% In order to enable it, see sections/endpage.tex and essay-name.tex's preamble
%
\usepackage[style=mla, backend=biber]{biblatex} % Nice MLA bibliography
\addbibresource{citations.bib} % Biblatex. See includes/formatting.tex

\SetCiteCommand{\autocite}

% This actually loads hyperref, lol
\usepackage[a-3u, mathxmp]{pdfx}

% Settings used by \usepackage{hyperref}
% Used to setup hyperlinks, as well as to properly annotate PDF metadata
\hypersetup{
  colorlinks=true,
  allcolors=blue,
}
\urlstyle{sf}

% Any additional user configuration goes below


% Document Title, Author, and Date
\title{
  \textbf{Reading Notes:} \\
  On Jean-Paul Sartre's \emph{Being and Nothingness},\\
  a Study in Phenomenological Ontology
}
\author{
  Shen Zhou Hong \and St. John's College
}
\date{\today}

% Document Begins
\begin{document}
\maketitle
\begin{abstract}
\noindent
  A collection of notes, quotations, and miscellaneous remarks written during my reading of Jean-Paul Sartre's \emph{Being and Nothingness}, in fulfilment of my Senior Essay at St. John's College. Headings, sections and subsections are identical with Sartre's work.
\end{abstract}
\tableofcontents % Optional table-of-contents
% \clearpage

% Includes content body. If the essay is very long, it might make sense to split
% the content across multiple .tex files, and include them all here, e.g.
% \input{sections/preamble.tex}

\chapter{Introduction}

\section{The Idea of the Phenomenon}

\subsection*{Summary}
In the opening sections of \emph{Being and Nothingness}, \href{https://plato.stanford.edu/entries/sartre/}{Jean-Paul Sartre} (1905–1980) introduces us to the problem of modern philosophy,
\marginnote{This would be a good place to compare and contrast \href{https://plato.stanford.edu/entries/heidegger/}{Martin Heidegger}'s conception of the problem of philosophy, i.e. his \emph{Fundamental Problem of Metaphysics}.}
which is its usage of incompatible dualisms. He shows how \href{https://plato.stanford.edu/entries/phenomenology/}{Phenomenology} seemingly resolves these dualisms, but then introduces a dualism of its own: that of the finite and infinite, insofar appearances are concerned. He defines phenomenology as a philosophy in which the \emph{being}\footnote{\emph{Οὐσία}, translit. \emph{ousia} -- although Sartre's conception of it may be different from the Greek.} of existents are not somehow \enquote*{behind} the appearances of said existents (e.g. as \href{https://plato.stanford.edu/entries/kant/}{Immanuel Kant}'s ontology posits), but rather \emph{the being of an existent is in its appearances}. Sartre then concludes that in order to properly know the nature of such a being, we must investigate the being of appearances in further detail.

\begin{enumerate}
  \item As a preface, Sartre states that one of the accomplishments of modern philosophy was to reduce \enquote*{existents} to merely the \textcquote[1]{sartre}{series of appearances that manifest it}
  \item This is done in order to eliminate certain \href{https://plato.stanford.edu/entries/dualism/#ProForDua}{\enquote*{troublesome dualisms.}}
  \begin{enumerate}
    \item Such as the dualism of the internal, versus the external.
  \end{enumerate}
  \item \textcquote[2]{sartre}{An appearance refers to the total series of appearances, not to some hidden reality that siphons off all the existent's \emph{being} for itself.}
  \item \textbf{Noumenal world} versus \textbf{Phenomenal world}:
  \marginnote{How does Sartre's use of these words compare with the more traditional Kantian understanding?}
  by noumenal, we mean the material (and potentially inaccessible) world outside of us, versus the phenomenal world which consists of what appears to us. The phenomenal world can be understood as our mental world, in a sense.
  \item Sartre proceeds to give us \textbf{a working definition of Phenomenology:}
  \begin{enumerate}
    \item The being of an existent \emph{is in it's appearances}. \textcquote[2]{sartre}{For the being of an existent is precisely the way which it \emph{appears}}
    \item The phenomenon is the \emph{absolute-relative}. It remains relative, because the phenomenon has to \emph{appear} to someone.\footnote{Naturally, it is the exploration of this \emph{someone} which we are interested in. An understanding of the ontological foundations of \emph{conscious being} is the first step in the \emph{techné} of \href{https://plato.stanford.edu/entries/artificial-intelligence/}{artificial beings.}} However, it is absolute because the appearance is not merely our perception of a deeper, transcendent being.
    \item But rather, the \emph{appearance is being}.
  \end{enumerate}
  \item \marginnote{Is the very idea of dualisms being \enquote*{troubling} an unexamined preconception in the first place? If so, how can we substantiate it?}
  The benefit of phenomenology as opposed to other competing ontologies (e.g. Immanuel Kant's \href{https://plato.stanford.edu/entries/kant-metaphysics/}{transcendental metaphysics}) is that it resolves the `troublesome dualisms' of an internal versus external being. However, it brings out new problems:
  \begin{enumerate}
    \item The \textbf{dualism of the finite versus the infinite}: An object (i.e. existent) has an infinity of appearances, as there are an unlimited amount of phenomenal subjects to which the object can appear.
    \item \marginnote{It seems like phenomenology, at least when misunderstood-- can easily fall into the debasement of mere solipsism. How does Sartre defend his phenomenological ontology from such pitfalls?}
    This infinity is necessary for an \emph{objective phenomenology}. \textcquote[4]{sartre}{The reality of this cup is that it is there, and that it is \emph{not} me.}
    \begin{enumerate}
      \item \textcquote[4]{sartre}{We can express this by saying that the series of its appearances is connected by a \emph{principle} that does not depend on my whim.}
    \end{enumerate}
    \item However, keep in mind that as subjects, \emph{we only see a finite set of appearances for any given existent at any given time}.
    \begin{enumerate}
      \item And yet, the existent must have an infinite series of appearances.
      \item Hence, there is now \textbf{a dualism between finite and infinite appearances}.
      \item By resolving the dualism of the internal and the external, phenomenology has seemingly introduced a new dualism.
    \end{enumerate}
    \item \textcquote[4]{sartre}{Thus, a \emph{finite} appearance indicates itself in its finitude, but at the same time in order to be grasped as an appearance-of-that-which-appears, it demands to be surpassed towards the infinite.}
    \item Hence from this line of reasoning the nature of the phenomenon's being are endowed with certain \emph{transcendent} properties.
    \marginnote{What are these \emph{transcendent} properties of the phenomenon?}
  \end{enumerate}
  \item Recall that the essence, or being of an existent is now completely in its appearance. Hence, in order to properly ground phenomenology, we must investigate the \emph{being} of the appearance itself.
\end{enumerate}

\section{The Phenomenon of Being and the Being of the Phenomenon}

\subsection*{Summary}
Sartre presents an important distinction between the \emph{phenomenon-of-being}, and the \emph{being-of-phenomena}. We wish to study the latter, not the former -- even though strictly speaking we only have access to the former (right now, at any rate). Sartre asks us whether or not the phenomenon-of-being is reducible to the being-of-phenomena, to which he concludes this reduction is not possible. This is because any phenomena (appearance) is founded on being: specifically, Sartre says that \textcquote[7]{sartre}{being is the condition of a phenomena's disclosure}.
\marginnote{What would it mean for something to be the very precondition for disclosure?}
Hence, the being-of-phenomena is not in the phenomena itself, but rather has a \emph{transphenomenal foundation}. That is the ultimate conclusion of this section.

\begin{enumerate}
  \item \textbf{Eidetic reduction}: \emph{a technique in the study of essences in phenomenology whose goal is to identify the basic components of phenomena. Eidetic reduction requires that a phenomenologist examine the essence of a mental object, with the intention of drawing out the absolutely necessary and invariable components that make the mental object what it is. This is achieved by the method known as eidetic variation.
  \marginnote{Is there such a thing as a \enquote*{method} for phenomenological inquiry?}
  It involves imagining an object of the kind under investigation and varying its features. The changed feature is inessential to this kind if the object can survive its change, otherwise it belongs to the kind's essence.}
  \item We wish to study the nature of \emph{being}. We define earlier on that the being of any existent is in its phenomenon. Hence, we wish to study the \emph{being-of-phenomenon.} However, \emph{being} itself is also a phenomenon -- that's how we can talk about and reason about it.  \begin{enumerate}
    \item Hence, there are two concepts we need to understand clearly:
    \item \textbf{The phenomenon-of-being}: an appearance of being.
    \item \textbf{The being-of-phenomenon}: the being of appearances. (This is what we want to study!)
  \end{enumerate}
  \item \textcquote[6]{sartre}{Is the phenomenon-of-being [that we can reason about] identical to the \emph{being-of-the phenomena}?}
  \begin{enumerate}
    \item After all, remember -- the purpose of our investigation is in the \emph{being-of-the-phenomena}.
    \item Which admittedly, we can only reach through the intermediary of the phenomena (the appearance) of being.
    \item These are somewhat tricky and nit-picky differences, but it is important to keep them in mind!
  \end{enumerate}
  \item The being-of-phenomena cannot be simply resolved into the phenomenon-of-being \autocite[7]{sartre}.
  \begin{enumerate}
    \item This is because phenomena itself can only exist on the foundation of being. Sartre calls being \enquote{the condition of all disclosure} for appearances.
  \end{enumerate}
  \item This leads to a certain important ontological insight, which is that \emph{\textcquote[7]{sartre}{knowledge alone cannot account for being, i.e., that the being of the phenomenon cannot be reduced to the phenomenon of being.}}
  \item \textcquote[7]{sartre}{The phenomenon of being requires the transphenomenality of being.}
  \begin{enumerate}
    \item \textcquote[7]{sartre}{Although the being of the phenomenon is co-extensive with the phenomenon, it must escape the phenomenal condition in which existence is possible [only as a condition] that it is revealed [i.e. a phenomenon].}
  \end{enumerate}
  \item I'm not sure if I understand this completely, but essentially Sartre concludes that the being-of-phenomena has a transphenomenal nature, where it is grounded in something that is not strictly just phenomena.
\end{enumerate}

\section{The Prereflective Cogito and the Being of the Percipere}

\subsection*{Summary}
In this section, Sartre asks whether or not the being-of-phenomena can be found in \emph{knowledge}, which is the proportionality of an appearance's \emph{appearing} (e.g. we have more knowledge of a phenomenon, when it appears more strongly to us). He examines this knowledge-hypothesis of being (my term for it), before concluding that it is not an adequate answer, for to place the being of phenomena into knowledge invites an infinite regression.

He then (rather confusingly) goes on a tangent where he resolves this infinite regression, by showing that knowledge is related to consciousness -- for in order to know (percipere) something (a percipi), there has to be a knower (percipiens).
\marginnote{What is the neccessary relation between being and consciousness?}
He examines the nature of the percipiens (consciousness), and elaborates on it's characteristics. The most important is that consciousness is \emph{positional}, namely that it \emph{posits} towards something (i.e. an object of consciousness). But otherwise, consciousness is itself contentless and empty.

Finally, he explains how consciousness is self-conscious. He first posits that consciousness is by necessity self-conscious, and demonstrates the inadequacy of a simple consciousness-of-consciousness approach, because that too leads to an infinite regression. Instead, he shows that the self-consciousness of consciousness lies in a certain \emph{non-cognitive, non-positional consciousness}, which he calls the \emph{pre-reflective cogito}. It's called pre-reflective, because it is prior to the more ordinary positional (i.e. reflective) cogito.

In the most confusing part, he demonstrates that this pre-reflective cogito is an essential characteristic of consciousness itself, and that they are one and the same.
\marginnote{How does Sartre's non-symmetrical conception of consciousness relate against Hegel's conception of the lord and bondsmen?}
Every act of consciousness is both a consciousness of an object (positional), and conscious of itself. This is justified using Husserl's factual necessity. With this, he finishes his digression on the nature of consciousness.

\begin{enumerate}
  \item Some introductory Latin definitions:
  \begin{enumerate}
    \item \textbf{Percipere}: Infinitive. To perceive, to learn, to secure.
    \item \textbf{Percipi}: Passive, the perceived.
    \item \textbf{Percipiens} The perceiver.
  \end{enumerate}

  \subsection{The Knowledge-Hypothesis of Being}
  \item Sartre addresses the hypothesis that the being of an appearance (i.e. being-of-phenomenon) is rooted in the way that it \emph{appears}.
  \begin{enumerate}
    \item \textcquote[8]{sartre}{The being of an appearance is proportionate to its \emph{appearing}.}
    \item This hypothesis attempts to reduce the being-of-phenomenon to a simple matter of our knowledge (where knowledge is taken as the proportion of an existent's appearing to us).
  \end{enumerate}
  \item In Sartre's latin vocabulary, he christians this knowledge-hypothesis with the following proportionality: The phenomena-of-being appears to a perceiver, as a \emph{percipi} to an \emph{percipiens}. This proportionality is \emph{knowledge}.
  \begin{enumerate}
    \item Hence, under the knowledge-hypothesis, the being-of-phenomena lays within knowledge.
    \item However, this hypothesis is inadequate, since we essentially move the burden of \enquote*{where is being} on towards knowledge:
  \end{enumerate}
  \item \textcquote[8]{sartre}{If any metaphysics presupposes a theory of knowledge, it is equally true that any theory of knowledge presupposes a metaphysics}
  \marginnote{How does the pre-supposition of an underlying metaphysics behind a theory of knowledge relate against Gödel's meta-mathematical propositions? Is meta-mathematics akin to a metaphysics behind the theory of knowledge of maths?}
  \item A potential response to this inadequacy of the knowledge-hypothesis is to state that the being of the percipi is founded in the being percipiens, i.e. the perceiver.
  \begin{enumerate}
    \item \textcquote[9]{sartre}{One might allow that the percipi refers to a being that escapes the laws of appearance while stil maintaining that this transphenomenal being is the \emph{subject}'s being.}
    \item This points us towards the direction of consciousness. Could the being-of-phenomena lie within consciousness?
    \item In order to properly evaluate the knowledge-hypothesis of being, we must have a good understanding of consciousness. This is what Sartre does next.

    \subsection{Understanding Consciousness (As a Potential Resolution for the Being-of-Phenomena)}
    \item Sartre presents a very Husserl-style \textbf{phenomenological definition for consciousness}:
    \begin{enumerate}
      \item \textcquote[9]{sartre}{As Husserl showed, \emph{all} consciousness is consciousness \emph{of} something. In other words, there is no [act of] consciousness that does not \emph{posit} a transcendent object}
      \item Consciousness is a \emph{posit}-ional consciousness of objects in the world, where consciousness has no content, but only aims to reach (or in other words, \emph{posits}) towards objects.
      \item \textcquote[9]{sartre}{Every knowing consciousness can only be knowledge of it's [posited] object.}
      \item Essentially consciousness is an action towards things (it posits). It is not an object which has content.
    \end{enumerate}
    \item However, even though consciousness is a contentless action towards an object, \emph{consciousness must always be aware of its own consciousness towards said object}. In other words, the act of consciousness must be conscious of itself being said act.
    \begin{enumerate}
      \item This is absolutely necessary, otherwise if I was conscious of an object without being conscious that I am conscious of an object, I would become an \emph{unconscious conscious}, which is an absurdity.
      \item However, this strict requirement that consciousness must be aware of itself, also leads to it's downfall so far as a hypothesis for the being-of-phenomena as consciousness, which Sartre will present in the following.
    \end{enumerate}
  \end{enumerate}

  \subsection{The Inadequacy of Consciousness-of-Consciousness}
  \item \textcquote[10]{sartre}{The reduction of consciousness to knowledge effectively imports the subject-object duality within consciousness.}
  \item Furthermore, recall how consciousness must be aware of itself? This leads to a consciousness-of-consciousness. But wouldn't that consciousness need another consciousness, e.g. a consciousness-of-consciousness-of-consciousness? This by necessity leads to an infinite recursion.
  \begin{enumerate}
    \item \textcquote[11]{sartre}{Either we stop at some term within the series -- in which case the phenomenon in its totality collapses into the unknown (i.e. we always come up against a reflection that is not conscious of itself and is the final term) -- or we declare an infinite regress to be necessary, which is absurd.}
  \end{enumerate}

  \subsection{The (Reflective) Non-Positional Consciousness Towards Positional Consciousness Hypothesis}
  \item In order to prevent the absurdity of this infinite recursion, we cannot have an isomorphic relationship of consciousness-to-consciousness. Rather, we must have an \textcquote[11]{sartre}{immediate and non-cognitive relationship of self to self.}
  \begin{enumerate}
    \item \textcquote[11]{sartre}{Any positional consciousness of an object is at the same time a non-positional consciousness of itself}
  \end{enumerate}
  \item My understanding of this step is not very clear, but the essential meaning is as follows. Every act of consciousness must have a relationship to itself which is non-consciousness (or at least, non-positionally conscious). Sartre provides the following example, which I will reproduce:
  \begin{enumerate}
    \item Imagine yourself counting cigarettes (sartre's example). You count that there are twelve of them. That's an objective property which you are \emph{positionally} conscious of (i.e. your consciousness posits the twelve-ness of these cigarettes).
    \item However, this positional awareness (i.e. consciousness) of the number of cigarettes appears to you \emph{as a direct, immediate property of the world.} You don't have a positional consciousness of counting them.
    \begin{enumerate}
      \item \textcquote[12]{sartre}{I do not \enquote{know myself as counting}. Proof of this can be seen in the fact that children who are capable of spontaneous addition are unable to \emph{explain} afterwards how they did it.}
    \end{enumerate}
    \item Hence, any positional act of consciousness contains a reflective act that is non-cognitive and non-positional.
  \end{enumerate}
  \item \textcquote[12]{sartre}{Thus, reflection lacks any kind of primacy in relation to the reflected consciousness: it is not by the means of [reflection] that the [primary, positional] consciousness is revealed to itself.}
  \begin{enumerate}
    \item \textcquote[12]{sartre}{On the contrary, non-reflective consciousness is what makes reflection possible. There is a prereflective \emph{cogito}, which is the condition of the Cartesian [positional, primary] cogito}
    \marginnote{There is a good parallel to be made with Galileo's image of how only non-reflective objects are illuminated, in his \emph{Dialogues on Dual World Systems}.}
  \end{enumerate}
  \item In summary, consciousness is contentless and positional. However, consciousness itself requires a reflection against a non-cognitive, non-positional act of consciousness, which Sartre calls a pre-reflective \emph{cogito}
  \item What is the nature of this pre-reflective \emph{cogito?} Sartre explores it next.

  \subsection{The Nature of the Pre-Reflective Cogito}
  \item \textbf{The pre-reflective cogito}, i.e. the non-positional consciousness is essential to the positional consciousness.
  \item We cannot talk of positional consciousness at all, without acknowledging that it is integrally tied with the consciousness of itself. Sartre puts this into more definite words:
  \begin{enumerate}
    \item \textcquote[12]{sartre}{Any conscious existence exists as the consciousness of existing. The most basic consciousness of consciousness is not positional, because it and the consciousness of which it is conscious [i.e. the aforementioned positional consciousness] are one and the same.}
    \item \textcquote[12]{sartre}{In a single movement, consciousness determines itself as consciousness of perception, and as perception}
    \item \textcquote[13]{sartre}{[This integral consciousness of self] is the only possible mode of existence for any consciousness of something.}
  \end{enumerate}
  \item Let me try to explain the above in my own words. The syllogism is thus:
  \begin{enumerate}
    \item Consciousness is positional and contentless.
    \item However, to be conscious of something, we must \emph{also} be conscious of our consciousness of something. We must be conscious of our positional consciousness.
    \item It is not possible to have this second consciousness to be the same as the first consciousness, for then we would yield an infinite regression, which is absurd.
    \item Hence, the second consciousness must be by necessity non-cognitive.
    \item We conclude that the consciousness of consciousness must be integral to the first consciousness.
  \end{enumerate}
  \item It's not the best explanation, but it captures the gist of Sartre's idea.

  \subsection{Sartre's Elaboration on the Nature and Qualities of Consciousness}
  \item Given the above understanding of consciousness, Sartre proceeds to explain and explicate certain characteristics. I'll give a rough sketch of the points he makes:
  \item Consciousness in its being implies the existence of its essence. There's no essence behind consciousness. This is justified by \textbf{Husserl's doctrine of factual necessity}:
  \begin{enumerate}
    \item Consciousness does not necessarily have to exist. But once it does exist, its non-existence is inconceivable.
    \marginnote{Can the neccessary existence of consciousness, as justified by my own consciousness right now -- be an argument against the non-existence of other consciousnesses? Or in other words, does my own consciousness presuppose the consciousness of other people?}
  \end{enumerate}
  \item \textcquote[15]{sartre}{Consciousness is prior to nothingness, and \enquote{derives from being}}
  \item \textcquote[15]{sartre}{This does not imply at all that consciousness is the foundation of its being.}
  \item The sort of conclusion of this section is that consciousness has a certain bootstrapping nature -- it's derives from it's own being (?)
  \item \textcquote[16]{sartre}{[Consciousness] is a pure ``appearance,'' where this means it exists only to the extent to which it appears. But it is precisely because consciousness is pure appearance, because it is a total void (since the entire world is outside it), because of this identity within it between its appearance and its existence, that it can be considered as the absolute.}
\end{enumerate}

\section{The Being of the Percipi}
\subsection*{Summary}
This chapter of the introduction is ontologically sophisticated (as always, with Sartre) but thankfully simple in its general logical progression. After coming to a good understanding that consciousness has it's own being (which is a transphenomenal being), we ask ourselves the big question: \emph{\enquote{Is the transphenomenal being-of-phenomena} located in the transphenomenal being-of-consciousness?}

At first, this seems like a plausible argument. However, Sartre spends this chapter refuting it -- demonstrating that the being-of-phenomena absolutely cannot be derived or founded in the being-of-consciousness itself. He presents two arguments which propose the above, and then refutes them. The first argument is that phenomena are \emph{passive}, and hence it must derive it's being from the consciousness to whom it appears, which is active. Sartre refutes this argument by demonstrating that \emph{passivity is a relationship between being and (another) being}, and hence phenomena must have its own being, and not just be an emptiness to which the being-of-consciousness fills.

The second argument is that phenomena is relative to the consciousness to whom it appears, but Sartre likewise refutes this argument by showing that phenomena and consciousness do not have a two-way relationship. Phenomena is relative to consciousness only, there is no case that consciousness is relative to phenomena. Hence phenomena cannot derive its being from something that it has no access to.

Essentially, the goal of this chapter is to demonstrate that \textbf{phenomena must have its own being independent of the being-of-consciousness.} Thus, there is such a thing as a being-of-phenomena.

\begin{enumerate}
  \item After the lengthy elaboration in the preceding section, we come to the definite thesis that there is such a thing that is consciousness, which has its own being. Furthermore, consciousness is an essential condition for appearances.
  \item \textcquote[17]{sartre}{We have escaped idealism, according to which being is measured by knowledge \ldots\ for idealism, all being is \emph{known}, including thought itself \ldots\ and the philosopher in search of thought is obliged to consult the constituted sciences in order to derive thought from them as their condition of possibility.}
  \begin{enumerate}
    \item This is the poor neuroscientist cutting slices of the pre-frontal cortex trying to find the being of thought.
  \end{enumerate}
  \item Essentially, we now have a good working understanding of the \emph{being-of-percipiens} -- i.e. the being-of-consciousness. However, we must now ask ourselves: \emph{\enquote{Is the being-of-consciousness the foundation of the being-of-phenomena?}}
  \item At first, this seems plausible. After all, we demonstrate in the prior lengthy digression that consciousness is a necessary condition for phenomena. After all, you can't have anything appear if there is no one for it to appear to.
  \item \textcquote[17]{sartre}{Are we satisfied? We have found a transphenomenal being, but is that really the being to which the phenomenon of being points? Is [the transphenomenal being of consciousness] really the being of the phenomenon?}
  \item Oh boy. It's not, and Sartre will demonstrate in this chapter which is called the \textbf{Being of the Percipi}.
  \item This is because the percipi (i.e. the perceived thing, the phenomena) has it's own being which is \emph{irreducible to the being of the percipiens} (i.e. the perceiver, consciousness)
  \begin{enumerate}
    \item \textcquote[17]{sartre}{The known [thing] cannot be absorbed into our knowledge of it, we must recognise its \emph{being}. This \emph{being}, as we are told, is the \emph{percipi} \ldots\ The most we can say is that [the percipi] is relative to [the percipiens].}
  \end{enumerate}

  \subsection{Refuting Two Attempts To Derive Being-of-Phenomena From Consciousness}
  \item Sartre goes on to answer two other attempts to found the being-of-phenomena into the being-of-consciousness.
  \begin{enumerate}
    \item The first objection is that the phenomena is \emph{passive}, and hence phenomena must derive it's being from the consciousness to whom it appears (the consciousness is active).
    \begin{enumerate}
      \item But Sartre goes on to demonstrate that this passivity is an active action between two beings:
      \item \textcquote[18]{sartre}{Passivity does not involve the very being of the passive existent: it is a relation between one being and another being, and not being a being and a nothingness.}
      \item Hence even though phenomena is passive, it still has a being that comes from elsewhere than consciousness.
    \end{enumerate}
    \item The second objection is that phenomena is \emph{relative}, to the consciousness to which it appears. \textcquote[20]{sartre}{Is it conceivable that the being of the known [being-of-phenomena] should be relative to our knowledge of it?}
    \begin{enumerate}
      \item This also cannot be the case, being an existent (i.e. a phenomena) does not have a two-way relationship to the consciousness to whom it appears. An existent is relative to the consciousness, but a consciousness is \emph{not} relative to the existent. Hence, just because phenomena is relative, does not mean the being-of-phenomena stems from the being-of-consciousness.
      \item \textcquote[20]{sartre}{The perceived being stands before a consciousness that it cannot penetrate and that it cannot make contact with, and as it is cut off from consciousness, it exists cut off from its own existence [which is absurd].}
    \end{enumerate}
  \end{enumerate}
  \item \textcquote[20]{sartre}{Thus, there is no case in which either of the two determinations of \emph{relativity} or \emph{passivity} -- which concerns the ways of being -- is applicable to being itself. The esse [being] of the phenomena cannot be its percipi. The transphenomenal being-of-consciousness cannot provide the foundation for the phenomenon's transphenomenal being.}
\end{enumerate}

\section{The Ontological Proof}
\subsection*{Summary}
Sartre presents the ontological proof, which demonstrates that there exists such a thing which is a transphenomenal (outside of the phenomena) being for phenomenon. This is the being-of-phenomena (\emph{not} the phenomena-of-being!) which we have been searching for all along. He proves the existence of the transphenomenal being-for-phenomenon by using the being-of-consciousness as an instrument. Specifically, he looks at the way in which consciousness is positional towards objects (of consciousness).

First, he shows that the objects of consciousness must be related to the transphenomenal being of consciousness itself. They can only be related in two ways -- either the transphenomenal being of consciousness is located in the object, or the transphenomenal being of consciousness is related to the transphenomenal being of phenomena (which we seek).

The first possibility is false, because a concrete object cannot contain a transphenomenal consciousness. This I understand reasonably well. Sartre goes on to show that the second possibility is equally false, for we have already demonstrated that the transphenomenality of the being-of-phenomena cannot be derived from the transphenomenality of the being-of-consciousness.

However, Sartre than proceeds to show a third possibility. Which is that the transphenomenal being of phenomena is found in consciousness not through certain forms of \emph{presence}, but rather from \emph{absence}. What does this mean? We are conscious of an object, because it appears to us as a phenomena. However, any object can have an arbitrarily infinite amount of appearances, but we only see one. Hence, the being of the object is defined to us not as the presence of an appearance, but rather as the \emph{absence of all but one appearances}.

From this, Sartre concludes that \textbf{the being-of-phenomena is a \emph{non-being} of the being-of-consciousness.} Or in other words, the being-of-phenomena comes from the non-being which sets it apart as an object away from the being of our consciousness, hence making it \emph{objective}. This is my best understanding of this section so far.

\begin{enumerate}
  \item Sartre begins with the claim that \textcquote[20]{sartre}{the transphenomenality of consciousness actually requires the phenomenon's being to be transphenomenal,} to which he then goes to demonstrate using an \enquote{ontological proof.}
  \begin{enumerate}
    \item \textbf{transphenomenal}: the quality of being beyond the phenomena, i.e. transcendent.
  \end{enumerate}
  \item This ontological proof is very difficult to understand. It's logical structure seems to be of the following order:
  \begin{enumerate}
    \item We grant that consciousness must be consciousness \emph{of} something (such as an object). Now we flip the terms, and look at the above postulate from the perspective of the object. There are two possibilities:
    \begin{enumerate}
      \item Either consciousness is constitutive [i.e. the structure which describes] of it's object's being,
      \item Or consciousness is in its innermost nature related to a transcendent being.
    \end{enumerate}
    \item The first possibility is absurd, because if an object is structured by consciousness, it would be conscious itself.
    \item The second possibility is the one which Sartre follows.
  \end{enumerate}
  \subsection{The Object of Consciousness Derives Its Being Negatively}
  \item What does it mean for consciousness to be related to a transcendent being? It's not possible for consciousness to directly relate to a transcendent being, because remember -- consciousness is \enquote{real subjectivity,} and to posit anything transcendent beyond the conscious subject is against the very definition of consciousness.
  \item Sartre resolves this issue by discovering that the \emph{being of the object of consciousness comes from (i.e. is derived via) a negative act}:
  \begin{enumerate}
    \item \textcquote[21]{sartre}{The fact that: it is necessarily impossible for the infinite number of terms [appearances] in the series to stand before consciousness simultaneously, in conjunction with the fact that all but one of these terms is really absent, is the foundation of objectivity.}
    \item \textcquote[21]{sartre}{[For] if these impressions were present -- even if their number were infinite -- they would become merged into subjectivity; [hence it is the absence of the infinite term of impressions] which is what gives them [the object,] objective being.}
  \end{enumerate}
  \item Hence, the being-of-phenomena (i.e. the being of the object) comes from consciousness as a a \enquote{pure non-being}.

  \subsection{Summary on the Being-of-Consciousness}
  \item Sartre wraps things up here by giving some concluding remarks on the self-bootstrapping nature of consciousness.
  \item \textcquote[23]{sartre}{Consciousness is a being whose existence posits its essence and, inversely, it is conscious of a being whose essence implies its existence.}
  \begin{enumerate}
    \item If consciousness exists, it posits that it exists.
    \item If there is a phenomena, the essence of the phenomena implies the existence of a consciousness to which it appears.
  \end{enumerate}
  \item \textcquote[23]{sartre}{Consciousness is a being for whom in its being there is a question of its being, insofar as this being implies \textbf{a being other than itself}.}
  \begin{enumerate}
    \item What is this \textbf{being other than itself [consciousness]}? Sartre claims that this is the transphenomenal being of phenomena, the thing which we are looking for all along.
  \end{enumerate}
\end{enumerate}

\section{Being in Itself}
\subsection*{Summary}

\begin{enumerate}
  \item \textcquote[24]{sartre}{Consciousness is a revealed-revelation of existents, and these existents appear before consciousness on the foundation of their being.}
  \begin{enumerate}
    \item \textbf{revealed-relevation}: A relevation that is not only objective (in the sense that an \emph{object} has the status of a relevation), but it also has a subjective component, in the sense that the object's relevation is likewise revealed to \emph{some subject}.
    \item In other words: Consciousness is an appearing appearance of existents.
    \item It is an appearing appearance because we are conscious of our own consciousness. Our positional consciousness contains the posited \emph{content} of the existent (the relevation of the existent), but we are also aware of this relevation itself.
    \item These existents appear since they \emph{are} (they appear since they have being).
  \end{enumerate}
  \item Hence we have the fundamental law of existents (my words): \emph{For something to exist, it has to be.}
  \item Now Sartre presents a new, and important step in his reasoning. \textbf{Consciousness is ontico-ontological}: \textcquote[24]{sartre}{Consciousness [as an action can] always surpass an existent, not towards its being but towards this being's \emph{meaning}.}
  \begin{enumerate}
    \item What is the \emph{meaning} of a being? Sartre says that it is the phenomenon-of-being (?). \enquote{The meaning itself has a being,} and because meaning has a being in the first place, it can manifest itself as a phenomenon.
  \end{enumerate}
  \item I'm not entirely clear, but it seems to me that the ultimate aim which we seek to understand is this meaning of being. It is the a phenomenon, which can appear to our consciousness. However, we do not need to go on to find the meaning of the being of meaning anymore, because they are the same. Hence, we no longer have a vicious cycle.
  \item Sartre notes that we must make the distinction between two types of being. Right now we are investigating the being-of-phenomena, which comes from meaning. However, this is different from the being of consciousness, which comes from being-for-itself.
  \item Sartre says there are two distinct regions of being:
  \begin{enumerate}
    \item The being of the \emph{prereflective cogito}
    \item the being of the phenomenon (which we are investigating now).
  \end{enumerate}
\end{enumerate}


\section{Part I: The Problem of Nothingness}

\subsection{Chapter 1: The Origin of Negation}
Where does nothingness come from? This is the thematic question of chapter one. The general progression of this section's content and argument is as follows:

\begin{enumerate}
  \item First, we inquire about negation, which appears as a simple and rather direct phenomenon of nothingness.
  \begin{enumerate}
    \item After all, it's easy to posit positive being by making definite statements, e.g. "there is an apple." However, when we make a negation of a definite statements e.g. "there is no apple," it is clear that we are making a statement that's rooted in a certain conception of non-being.
  \end{enumerate}
  \item We ultimately conclude that it is not possible to derive negation from being, but instead negation must be derived from a certain definite non-being, i.e. \emph{nothingness}
  \item We try to investigate where this nothingness comes from. Sartre ultimately concludes that this \emph{nothingness cannot come from regular being, but it has to come from a being through which can be its own nothingness}
  \begin{enumerate}
    \item Sartre concludes that this being is the human being (i.e. the Daesin).
  \end{enumerate}
  \item Hence, in order to investigate nothingness, we have to investigate the human being.
  \item How is it possible for the human consciousness to experience nothingness? The answer to this is \emph{anguish}.
\end{enumerate}

\subsubsection{Questioning}
In this section, Sartre presents the necessity for non-being in all forms of questioning. Specifically:

\begin{enumerate}
  \item In every question, we confront a being that we interrogate.
  \item The answer to the interrogation can be either `Yes' (affirmative) or `No' (negative). To allow for an affirmative answer by necessity presupposes the possibility of a negative answer.
  \item After all, the very being of an affirmative answer is defined by its shadow, which is the negative part. When we say that X is Z, we are also simultaneously saying X is not A, B, C, \ldots\ Y.
  \item Hence, there is such a thing as non-being
\end{enumerate}

\noindent
With this, Sartre introduces the concept of \emph{non-being as a neccesary component of reality}, towards which we must investigate further throughout the rest of this section.

\subsubsection{Negations}
We begin our investigation into the nature of non-being by looking at \emph{negations}, which are easily observable manifestations (phenomenae?) of non-being.

\begin{enumerate}
  \item "It is not true that negation is merely a quality of judgement" \autocite[38]{sartre}
  \item In the process of questioning, we expect a being (which is the answer). But we can equally receive a non-being as a response. Sartre's analogy is that if a watch-maker questions a watch on why it's not working, it is perfectly plausible to receive a non-being as a response, e.g. the mainspring is missing.
  \item "A being is \emph{fragile} if it bears within its being a clear-cut possibility of non-being." \autocite[40]{sartre}
  \item It seems that nothingness is distinct from the process of thought which is negation. We need nothingness in order to separate beings from each other, but the thought process that is negation is more simple and less fundamental.
  \item Furthermore, Sartre claims that the thought process of negation must come from the being of nothingness.
  \item "If there is being everywhere, it is not only nothingness that becomes inconceivable: from being we can never derive negation" \autocite[44]{sartre}.
\end{enumerate}

\subsubsection{The Dialectical Conception of Nothingness}
This section seems to be a general review of a dialectical (i.e. Hegelian) conception of nothingness. To be completed.

\begin{enumerate}
  \item Being and non-being are not contemporaries.
\end{enumerate}

\subsubsection{The Phenomenological Conception of Nothingness}

Sartre demonstrates that \emph{it is impossible to dismiss nothingness} as either a shadow of being, nor as the absence or something before or after being. But rather nothingness has a definite existence as a special sort of being, which has the characteristic of its own negation (which he will elaborate in a later section). This crux of this demonstration lies in the neccessity of defining positive being using negations (in his example with distance and lengths), as well as with the existence of \textbf{negatities}.

\noindent
In conclusion, nothingness cannot be outside of being.

\begin{enumerate}
  \item Sartre presents a few examples here on how it is absolutely impossible to abstract away the being of nothingness into a simple quality of regular (i.e. \emph{positive}) being:
  \begin{enumerate}
    \item "Take for example the notion of distance \ldots\ it is easy to see the [distance] contains a negative moment: two points are distant when a specific length \emph{separates} them."
    \item "How might wish toreduce distance to being \emph{no more than} the length of the segments of which the [two points are] the limits \ldots\ [but] in this case we have switched the direction of our attention \ldots\ negation, expelled from the segment and its length, will take refuge in the two \emph{limits}" \autocite[55]{sartre}.
  \end{enumerate}
  \item This leads to the introduction of the existence of \textbf{negatities}, i.e. \emph{negative entities}. These things cannot be accounted for as positive being, but as beings which "inhabited in their internal structure by negation as a necessary condition of their existence" \autocite[56]{sartre}
  \item "Nothingness can only nihilate itself on the ground of being: if nothingness can be given, it is neither before being nor after being; nor is it, in a general way, outside being; rather, it is right inside being, in its heart, like a worm." \autocite[57]{sartre}
\end{enumerate}

\subsubsection{The Origin of Nothingness}

\begin{enumerate}
  \subsubsection*{The Origin of Nothingness in the Human Being}
  \item In this section, we try to find out where nothingness comes from, based on a set of restrictions which we demonstrated from the prior sections. I am going to rehash those prior conclusions again:
  \begin{enumerate}
    \item "Nothingness must be given in the heart of being"
    \item "But being-in-itself [i.e. the being-of-phenomena] is not able to produce this intraworldly [transphenomenal (?)] nothingness: the notion of being as a full positivity does not contain nothingness as one of its structures."
    \item "Nothingness cannot be conceived of, either outside being or on the basis of being" \autocite[57]{sartre}
    \item Nothingness must have the power to nihilate itself, i.e. be the source of its own negation.
  \end{enumerate}
  \item Based on the above restrictions, where does nothingness come from? Nothingness must be the result of some being, but it cannot be the being of the phenomena.
  \item "\emph{The being through which nothingness comes to the world must be its own nothingness.} And let us not construe this as an act of nihilation -- which would in turn require a foundation in being -- but as an ontological characteristic of the being which we are seeking."
  \begin{enumerate}
    \item \textbf{Nihilation}: The act in which an object with being is negated.
    \item This latter qualification is important. If the being through nothingness comes is something that is nihilated -- it would need another being to support it, which leads to an infinite regression. There has to be a being in which nihilation is the ontological requirement.
  \end{enumerate}
  \item What sort of being is able to nihilate itself? The clue towards this being comes from the ability to \emph{question}. For the act of questioning requires negation, as we have determined earlier already-- but more importantly:
  \begin{enumerate}
    \item "Every question posits, in its essence, the possibility of a negative answer. In a question we interrogate a being about its being, or its way of being. And this being, or way of being, is concealed; the possibility always remains open for it to be disclosed as nothingness."
    \item "But it follows, from the very fact of our envisaging that an existent can always disclose itself as nothing, that every question presupposes that \textbf{we have taken a nhilating step in relation to the given, which becomes mere \emph{presentation}, oscillating between being and nothingness}." \autocite[59]{sartre}
    \begin{enumerate}
      \item This is the "permanent possibility in which the questioner is able to detach himself from the causal series that constitute being," where:
    \end{enumerate}
    \item "In consequence, through a \emph{twofold movement of nihilation}, [the questioner] nihilates the thing he is questioning in relation to himself:"
    \begin{enumerate}
      \item "By placing it in a neutral state between being and non-being"
      \item "[And also] by separating himself from being in order to draw out from himself the possibility of a non-being". \autocite[59]{sartre}
    \end{enumerate}
    \item It seems clear to me that this act of questioning takes it's ontological characteristic as a \emph{twofold movement of nihilation}. Where to question, is to complete the neccessary two steps where:
  \end{enumerate}
  \item Sartre's conclusion is that \textbf{the being from which negatity (i.e. nothingness) comes from is the being of man, i.e. the being-of-consciousness.}
  \item Keep in mind that this process is not a process of \emph{generation}. \textbf{No being can generate \emph{non-being}, never!} But rather, it is \emph{a process of changing our being's relationship to another being}:
  \begin{enumerate}
    \item "To disconnect some particular existent, for human-reality (i.e. Daesin) is to disconnect [human reality] in relation to [the existent]. In this case, human-reality escapes the existent and cannot be acted on by it; it is out of reach, having withdrawn beyond a nothingness"
  \end{enumerate}
  \item Now that we conclude that the being which is the condition of negation and nothingness is the human being, i.e. the being-of-consciousness. But furthermore, we acknowledge that this is \emph{not} a process of generation, but rather one of changing our relationship to definite beings.
  \item \textbf{This process of bringing out nothingness in our being, is called \emph{freedom}.}

  \subsubsection*{Nothingness as Freedom, the Phenomena of Freedom as Angst}
  \item In this second half of the section, Sartre talks about the practical implications of the being-of-consciousness's ability to bring about nothingness as freedom. He first presents it more or less directly, but than he presents the consciousness-of-this-freedom as \emph{angst.}
  \item "If nihilating consciousness exists only as a consciousness of nihilation, it ought to be possible to define and describe a constant mode of consciousness, present \emph{as} consciousness, that is the consciousness of nhilation." \autocite[66]{sartre}.
  \begin{enumerate}
    \item This \textbf{consciousness-of-nihilation} is the human emotion of \emph{angst}.
    \item There are two types of angst, properly speaking. There is both angst for the future, as well as angst for the past.
  \end{enumerate}
  \item A question that comes up at this stage is to ask: \emph{"How is angst different from \emph{fear}?"}
  \begin{enumerate}
    \item Fear is the emotion of worrying of \emph{external existents.}
    \item Angusih is the emotion of worrying about one's own \emph{being.}
  \end{enumerate}
  \item These definitions will be developed by further examples. For instance, a key component of anguish is the \emph{nihilation} or \emph{nothingness}. It's an awareness of the nothingness which conditions the vast array of possibilities of one's own being. Sartre talks more about this from pages 66 and onwards \autocite[66]{sartre}, with a particularly definite example on \autocite[71]{sartre} and \autocite[77]{sartre}.
  \item In \autocite[80]{sartre} Sartre talks about the ways in which one flees away from this anguish.
  \item In the very end of this chapter, we look at 'bad faith' -- which Sartre describes as the collection of behaviours (consciousness(es) (?)) in which we flee away from \emph{anguish}. It's important to Sartre that we examine bad faith next in our inquiry, for the following reasons:
  \begin{enumerate}
    \item Bad faith is paradoxical, since in order to flee away from anguish, we must aim at anguish itself \autocite[86]{sartre}. This means that the content of bad faith contains anguish.
    \item As a result, bad faith serves as a very good and direct proxy to understand what this anguish is, which will allow us to go further in our question of nothingness.
  \end{enumerate}
  \item As a sort of final sketch in this part of our inquiry, we can summarise the digression as follows:
  \begin{enumerate}
    \item Nothingness must exist, but cannot come from or be generated upon, or be founded by being.
    \item Nothingness is a relationship between two beings, where one being negates the other (?).
    \item The only being that is capable of this action of nihilation is the human being, i.e. the Daesin or the being-of-consciousness.
    \item The way in which consciousness is conscious of this act of nihilation is in the phenomena of \emph{angst}.
    \item We try to flee from angst through the application of bad faith. However, bad faith must contain the content of angst.
    \item Hence, finally -- in order to understand where being and nothingness comes from, we must examine bad faith as our proxy.
  \end{enumerate}
\end{enumerate}

\subsection{Chapter 2: Bad Faith}

\subsubsection{Bad Faith and Lies}

\begin{enumerate}
  \item \textbf{New Working Definition of Consciousness:} "Consciousness is a being for whom in its being there is concsciousness of the nothingness of its being." \autocite[87]{sartre}
  \item "What must man be in  his being for it to be possible for him to negate himself?" \autocite[88]{sartre}. Where self-negation serves as the foundation of bad faith, it seems.
  \begin{enumerate}
    \item "We should choose and examine a specific attitude, essential to human-reality (i.e. Daesin), and in which, at the same time, consciousness, instead of directing its negation outward, turns it against itself. It has seemed to us that this attitude must be \emph{bad faith}." \autocite[88]{sartre}
  \end{enumerate}
  \item Bad faith is not simply lying, or even some extended or fundamental form of lying. For lies "requires no special ontological foundation." \autocite[89]{sartre}
  \begin{enumerate}
    \item In ordinary lying, there is the liar, and the deceived.
  \end{enumerate}
  \item "In bad faith it is from myself that I am concealing the truth. Thus the duality of the deceiver and the deceived is not present here. On the contrary, bad faith implies in its essence the unity of a single consciousness." \autocite[90]{sartre}
  \item Sartre goes on to test, criticise, and ultimately reject the Freudian explanation for the foundation of bad faith. The Freudian explanation posits a trinity of the consciousness as the \emph{id, ego, and superego} -- upon which there's an interference between one of the two.
  \item For reasons that are not ultimately too important, the Freudian explanation is shown to be an inaccurate one at best.
  \item the ultimate conclusion here is that \textbf{bad faith must know the thing which it denys, in order to actively act in denial of it}
\end{enumerate}

\subsubsection{Forms of Bad Faith}

\begin{enumerate}
  \item In order to explore properly what bad faith is, Sartre takes us on an examination of various everyday-examples of bad faith in action.
  \item In \autocite[98]{sartre} Sartre presents an \emph{amazing} example of a form of `bad faith' in practice -- the dance of flirtation. It's really cool and you should totally check it out.
  \item \textbf{Characteristics of Bad Faith}
  \begin{enumerate}
    \item Forming contradictory concepts, where an idea and the negation of the idea are united.
    \begin{enumerate}
      \item i.e. "I am not what I am"
    \end{enumerate}
    \item The method in which we generate these contradictions is through "the twofold property of human beings, of being a facticity and a transcendence." \autocite[99]{sartre}
    \begin{enumerate}
      \item This seems to mean we accept (acknowledge?) a facticity, but then escape it through our transcendence (?)
    \end{enumerate}
  \end{enumerate}
  \item but Sartre also goes on to say that this facticity-transcendence dichotomy is not the only way in which we generate bad faith, but there are other ways?
  \item \autocite[102]{sartre} What does it mean to \emph{play} at acting something? Sartre presents another marvellously beautiful example, that of the cafe-waiter. \textbf{I should examine this scene in more detail.}
  \item Around \autocite[109]{sartre} Sartre goes on a digression about sincerity, in an attempt to understand bad faith through its contrast.
  \item "At the same time, [through sincerity] the malice is defused, since if it only exists deterministically it is nothing, and since, by acknowledging it, I posit my freedom in relation to it; my future is virgin, so everything is permitted. In this way, sincerity's essential structure does not differ from that of bad faith, since the sincere man constitutes himself as what he is \emph{in order to not be it.}" \autocite[110]{sartre}
  \item Sincerity and Bad faith seems to be both two sides of the same coin, for they both require one to objectify oneself -- and to look on the self as an external object.
  \begin{enumerate}
    \item "Sincerity [and bad faith] does not assign a particular quality or way of being to me, but in relation to the quality at issue, it aims to move me from one mode of being into another mode of being." \autocite[111]{sartre}
  \end{enumerate}
\end{enumerate}

\subsubsection{The "Faith" of Bad Faith}

\begin{enumerate}
  \item Bad faith requires a specific position to be held on our epistemology. To be in the state of bad faith is to be in a state where we are willing to accept \emph{non-persuasive} evidence, since if the evidence was persuasive in the first place, we wouldn't be in bad faith.
  \begin{enumerate}
    \item "This primary project of bad faith is a decision, in bad faith, about the nature of faith." \autocite[114]{sartre}
  \end{enumerate}
  \item "There is no cynical lie in bad faith, or any knowning preparation of misleading concepts. But \textbf{bad faith's most basic act is to flee from something that is impossible to flee from: to flee from what one is.}" \autocite[117]{sartre}
\end{enumerate}


\section{Part II: Being-For-Itself}

A quick definitional reference:

\begin{enumerate}
  \item \textbf{Being-for-itself}: The being \emph{for-itself} is the kind of being of consciousness. To quote Oxford Dictionary, being for-itself \textcquote{oxford}{is the mode of existence of consciousness, consisting in its own activity and purposive nature.}
  \item \textbf{Being-in-itself:} is the existence of ordinary, non-conscious objects, like tables or chairs.
\end{enumerate}

\subsection{Chapter 1: The Immediate Structures of the For-Itself}

\subsubsection{Self-Presence}

\begin{enumerate}
  \item \textbf{The In-Itself:} Something that has an infinite density of being, a plentitude. I think when Sartre talks about the (Being-)In-Itself, he is talking directly about the being of the existent (perhaps analogous to the being-of-the-phenomena?).
  \item \textcquote[123]{sartre}{Identity is the limiting concept of unification \ldots\ at its extreme limit, unity vanishes and passes over into identity.}
  \item \textcquote[123]{sartre}{Consciousness is characterised, on the contrary, by its \emph{decompression of being.} Indeed, it is impossible to define it as self-coincident.}
  \begin{enumerate}
    \item What this means is that the being of consciousness is not a being that's in-itself. The example that Sartre gives is that when we talk about my belief, I cannot say that my consciousness \emph{is} by belief. But rather only that \textcquote[123]{sartre}{my belief is a consciousness (of) my belief.}
    \item See how in this base, the being-of-consciousness is not infinitely dense?
  \end{enumerate}
  \item \textbf{Self-Presence as the foundation for self-consciousness:} I'm not too certain in my understanding of this right now, but Sartre elaborates on this in \autocite[126]{sartre}. The gist of it seems to be:
  \begin{enumerate}
    \item The being-in-itself is the regular being of existents, i.e. the being-of-the-phenomena.
    \item However, what is the being-for-itself? The very use of the word \emph{for} implies a strong reflective action.
    \item We cannot qualify the being-\emph{for}-itself using any regular conception of the being-in-itself.
    \item The key difference in the being-for-itself is that there's a separation which makes the being not its own coincidence, but still requires its own unity (bottom of \autocite[126]{sartre})
    \item Hence this self-presence must have some sort of separation, which will be shown to be \emph{nothing}.
    \item \textcquote[126]{sartre}{The law of being of the for-itself as the ontological foundation of consciousness is to be itself in the form of self-presence.}
  \end{enumerate}
  \item \textbf{Self-Presence as an act of separation from the self:} Sartre elaborates this at \autocite[127]{sartre}. Self-Presence is taken as something different, or apart from identity -- which is the dense plentitude of being, as we have explained above. In fact:
  \begin{enumerate}
    \item \enquote{The principle of identity is the negation of any type of relation within the being-in-itself.}
    \item \enquote{On the contrary, self-presence presupposes that an intangible fissure has slipped inside being. If it is present to itself, that is because it is not completely itself. Presence is an immediate degradation of coincidence, because it presupposes separation.}
    \item \textcquote[127]{sartre}{But if we ask now \emph{what} separates the subject fro himself, we are forced to admit that it is \emph{nothing.}}
  \end{enumerate}
  \item So once again, we are back at the discovery that nothing is essential for being -- in this case, nothing is essential for self-presence.
\end{enumerate}

\subsubsection{The For-Itself's Facticity}

\begin{enumerate}
  \item It seems like this section Sartre ties down the being-for-itself into the being-in-itself. Essentially, there has to be some sort of foundation for the for-itself.
  \item \textcquote[133]{sartre}{Thus the for-itself is supported by a constant contingency that it takes up, and assimilates, without ever being able to get rid of it. We may call this constantly evanescent contingency of the in-itself -- which haunts the for-itself and ties it to being in-itself without ever allowing itself to b e grasped -- the for-itself's \emph{facticity.}}
  \item It seems to me that this distinction -- the idea that the being-for-itself is founded upon a factual circumstance -- is important to avoid the illusion and absurdity of solipsism.
  \begin{enumerate}
    \item \textcquote[134]{sartre}{The for-itsef, even while it chooses the \emph{meaning} of its situation and constitutes itself in situation as its own foundation, does \emph{does not choose} its position.}
  \end{enumerate}
\end{enumerate}

\subsubsection{The For-Itself and the Being of Value}

\begin{enumerate}
  \item \textbf{The Lack:} The lack is the form of negation which \textcquote[137]{sartre}{most deeply establishes an internal relation between what we negate and what our negation applies to \ldots\ [the lack is the form of negation] which penetrates most deeply into being -- the one that constitutes \emph{in its being} the being to which its negation applies with the being that it negates.}
  \begin{enumerate}
    \item \textcquote[138]{sartre}{The lack does not belong to the nature of the in-itself, which is entirely positive. It appears within the world only when human-reality arises.}
    \item For example, given an unfinished circle -- it is technically an open curve that is complete in its being as an open curve. It is only through the realm of human-reality, specifically human desire, to which we give to it the lack -- the lack in which it is not a circle. \autocite[139]{sartre}
  \end{enumerate}
  \item Human reality itself must be a lack, because only through a lack can we derive lacks. Sartre talks about this very definitively in \autocite[139]{sartre}:
  \begin{enumerate}
    \item \textcquote[139]{sartre}{A psychological state whose existence had the sufficiency of that curve [i.e. the unfinished circle] could not in addition make the slightest `call' for anything else: it would be itself, without any relation to anything other than itself}
    \item \textcquote[139]{sartre}{In order to constitute it as a hunger or thirst, an external transcendence would be required.}
    \item \textcquote[139]{sartre}{No recourse to psychophysiological parallelism [i.e. the doctrine that the psychological is a direct parallel to the physiological] can enable us to escape these difficulties:}
    \begin{enumerate}
      \item Any physiological signs of a lack of water in an organism only posits a positive being of the state of the organism, refering to itself. Sartre presents this in vivid detail in \autocite[139]{sartre}
      \item \textcquote[139]{sartre}{[Any] exact correspondence between the mental and the physiological [requires] that correspondance [to be] established only on the basis of an ontological identity.}
    \end{enumerate}
  \end{enumerate}
  \item \textbf{Desire is a \emph{lack} of being:} \textcquote[140]{sartre}{and is haunted in its [desire's] innermost being by the being that it desires [i.e. lacks].}
  \item \textbf{Lack is a trinity:} When we lack something, there are three components to the act of lacking \autocite[138]{sartre}:
  \begin{enumerate}
    \item \textbf{The \emph{manqué} (i.e. the lack):} The item that is missing
    \item \textbf{The incomplete existent:} That from which [the item] is missing [i.e. the existent].
    \item \textbf{The hypothetical whole:} A totality that is broken apart by the lack, and could which be restored by the synthesis of the missing item with the existent.
  \end{enumerate}
  \item In \autocite[139]{sartre} talks about how \textbf{value} comes from this lack. I need to investigate this further.

  \subsubsection*{The Being to Which Consciousness Aims For}
  \item Sartre presents a rather tricky, but essential understanding on \textbf{the emergence of value from lack and desire}. It's essential that we understand what value is, and where does it come from. Right now, take value in this case to mean ethical/personal value, i.e. what is important to us, or what we aim for. Sartre's presentation goes as follows:
  \begin{enumerate}
    \item Recall that lack is a trinity.
    \item Further recall that \textbf{the human-reality is a lack} (since otherwise, the being of the human condition would be positive, and there would be no such thing as lacking).
    \item Hence, \emph{if the human-reality is a lack, what are the components of the lack's trinity?} Sartre answers this in \autocite[140]{sartre}, where he states:
    \begin{enumerate}
      \item \textbf{The \emph{manqué} (i.e. the lack):} \textcquote[141]{sartre}{The \emph{itself-as-being-in-itself}.}
      \item \textbf{The incomplete existent:} \textcquote[139]{sartre}{The element that plays the role of the existent is given to the \emph{cogito} as the immediacy of the \emph{desire.}}
      \item \textbf{The hypothetical whole:} \ldots
    \end{enumerate}
    \item From the incomplete trinity above, Sartre asks: what is this hypothetical whole from which the lack of the human-reality presupposes?
    \item It seems to me that this hypothetical whole is a transcendence towards a better whole, a better version of the being [i.e. self] (?)
    \item \textcquote[142]{sartre}{This constantly absent being which haunts the for-itself is itself -- but frozen in the in-itself [i.e. as an object].}
    \begin{enumerate}
      \item My intepretation of this sentence is essentially thus: Our human-reality is defined by a negative thing, a lacking. But a lacking must presuppose first a thing that is lacking (e.g. the missing puzzle piece), which Sartre cals the manqué -- as well as the incomplete existent (e.g. the puzzle-hole) and the hypothetical whole (e.g. the complete puzzle.)
      \item The incomplete existent is manifest as desire.
      \item But the thing which we are lacking in our human-reality is another state of human-reality or being, which is the object of our consciousness. Our consciousness wishes to be something else, to be another consciousness -- which it is not.
      \item Hence the ultimate, hypothetical, and unachievable synthesis of what we lack from the lacking is where \emph{value comes from}.
    \end{enumerate}
    \item \textbf{Value: a transcendent thing which our current being lacks, which eludes our being.} \autocite[146]{sartre}
    \item \textcquote[147]{sartre}{Value arrives to the world through human-reality.}
    \item \textcquote[148]{sartre}{Value haunts being insofar as it founds itself and not insofar as it is: it haunts \emph{freedom}. So value's relation to the for-itself is quite distinctive: it is the being that the for-itself has to be, insofar as it is the foundation of the nothingness of its being.}
  \end{enumerate}
\end{enumerate}

\noindent
As a sort of parting remark on this section, it seems that Sartre's ontology places an important role on the idea of a \emph{transcendence}, or a \emph{transcendent} thing. Whenever we are looking for something (i.e. some being, or quality of being) which does not exist in the thing (i.e. the being) itself, but comes from something which is beyond the given thing (i.e. being), we are looking for a transcendent thing. The transcendent thing is like a higher object to which a shadow is cast.

\subsubsection{The For-Itself and the Being of Possibles}

\begin{enumerate}
  \item In this section, Sartre takes the concept of \emph{lacking} and relates it to the concept or being of \emph{possibles.} He derives possibility from lacking through a similar transcendental meditation.
  \item The being of possibility is not in the being of any existents, but rather comes from the human-reality.
  \item However, possibility is also not subjective!
  \item Possibility is not within the being of the human-reality, but it is also transendent. It seems to be something outside of human reality. \autocite[158]{sartre}
  \item \textcquote[158]{sartre}{Let us call the for-itself's relation to the possible that it is the \enquote{circuit of ipseity} -- and the totality of being, insofar as it is traversed by the \emph{circuit of ipseity}, the \enquote{world.}}
  \item I'm not entirely certain at this point, but it sounds like that the world is the totality of possibility (in the context of human-reality), while the self traverses a subset of that as the circuit of ipseity.
\end{enumerate}

\subsubsection{My Self and the Circuit of Ipseity}

This seems to be a summary of the above sections and the chapter in general. I should revisit it sometime, in particular \autocite[161]{sartre}.

\subsection{Chapter 2: Temporality}

\subsubsection{Phenomenology of the Three Temporal Dimensions}

In this part, Sartre wishes to examine the past, the present, and the future -- without the explanation of time being a simple series of `nows' or moments, since this naive approach yields Xeno's paradox. In this discussion, he presents what he later refers to as the \emph{three temporal ecstasies}, which are acts of unification. \marginnote{My understandings of Sartre's temporal ecstasies is not very clear at the moment. How can I deepen my understanding of their them?}

\begin{enumerate}
  \subsubsection*{The Past}
  \item Sartre rejects the naively materialist (or in his terms, the psychophysiological parallelism) of the theory of `memory traces,' where the past is seen as something that is departed, and hence every memory is merely a physical, present trace in the mind.
  \item In the next pages, he presents a few non-materialist approaches to understanding where the past derives its being, and goes on to reject all of them.
  \item His conclusion is that the past must derive its being from the person to whom the past is for. He elaborates most keenly on this conclusion at \autocite[169]{sartre}. He presents an example with Pierre:
  \begin{enumerate}
    \item \textcquote[170]{sartre}{Of \emph{whom} is this past-Pierre the past? It cannot be in relation with an universal Present which purely affirms being; it is therefore the past of \emph{my actuality.} And as a matter of fact Pierre has been for-me and I have been for-him.}
  \end{enumerate}
  \item \textcquote[170]{sartre}{There are therefore beings that \enquote{have} pasts.} However, this \emph{does not} mean all beings have pasts! Rather, \emph{only a specific type of being} has a past, which Sartre elaborates in \autocite[172]{sartre}:
  \begin{enumerate}
    \item \textcquote[172]{sartre}{There is a past only for a present that cannot exist without being its past `over there,' behind it. In other words, \emph{the only beings that have a past are those beings for whom there is a question, in their being, of their past being} -- beings that \emph{have} their past \emph{to be}.}
    \item In my own words, the only types of beings that have a past, are the sort of beings which contain a question of their own being.
    \item or in other words, the only beings that have a past, are the beings that are beings \emph{for-itself,} (i.e. the being-of-consciousness). Beings that are only \emph{in-itself} (i.e. the being-of-phenomena) do not have pasts!
  \end{enumerate}
  \item Sartre then proceeds to learnedly make the important nuanced qualification that this \textcquote[172]{sartre}{\emph{does not settle the question of the past of living things}.}
  \begin{enumerate}
    \item Remember how we defined that the only beings which have a past, are the beings which are for-itself? In a more vulgar manner of speaking, we're talking about beings that are conscious.
    \item There are of course, plenty of living things like moss or algae which obviously do not fulfill this definition.
  \end{enumerate}
  \item In \autocite[174]{sartre} Sartre talks about the relationship between the past and death. There are some particularly memorable quotes (you should revisit the cited page):
  \begin{enumerate}
    \item \textcquote[174]{sartre}{Ultimately, at the infinitisimal instant of my death, I will no longer be anything but my past. It alone will define me.}
    \item \textcquote[174]{sartre}{Through death, the for-itself [being of consciousness] changes for eternity into in-itself [being of phenomena], to the extent to which it has entirely slipped into the past. Thus the past is the \emph{ever-increasing totality of the in-itself that we are.}}
  \end{enumerate}

  \item \textcquote[176]{sartre}{To explain the world in terms of becoming, conceived as a synthesis of being and non-being, is easily done. But has anyone considered that no being that becomes could be such a synthesis unless it were, in relation to itself, \emph{an act that founded its own nothingness?}}
  \item \textcquote[180]{sartre}{To sum it up, [the past] is an inversion of value, the for-itself reclaimed by the in-itself, thickened by the in-itself to the point at which it can no longer exist as a reflection for the reflecting, or as a reflecting for the reflection, but merely as an in-itself sign of the reflecting-reflection pair.}
  \begin{enumerate}
    \item This is an important summary on Sartre's conclusion on the nature of the past. To put into more simple words, the past is the \emph{being-for-itself} (i.e. the being of consciousness) which has became the \emph{being-in-itself}, the mere being of the phenomenon.
  \end{enumerate}
  \subsubsection*{The Present}
  \item \textcquote[181]{sartre}{[Any] strict analysis that aimed to rid the present of everything it is not -- i.e. its immediate past and future -- would in fact find nothing more than an infinitesimal instant \ldots\ the ideal term of an infinitely pursued division: a nothingness.}
  \begin{enumerate}
    \item With this opening passage, Sartre presents the fundamental problem of \emph{the present}, and relates it thematically to the earlier conceptions of nothingness which we discovered in the past.
    \item The first realisation that Sartre presents, is that the idea of the present -- or formally speaking, the attribute of \emph{presence} -- is a quality that only exists between two beings.
    \begin{enumerate}
      \item \textbf{Presence:} the quality of an object being \emph{present}.
    \end{enumerate}
    \item \textcquote[181]{sartre}{The in-itself cannot be present, any more than it can be past; it \emph{is}, quite simply. There can be no question of any one in-itself existing in some kind of simultaneity alongside another in-itself -- other than from the point of view of a being who was co-present to the two in-itselfs, and who had its own capacity for presence.}
  \end{enumerate}
  \item \textcquote[181]{sartre}{Therefore \textbf{the present can only be the for-itself's presence to being-in-itself.}}.
  \begin{enumerate}
    \item If I am understanding this argument properly, essentially the present is a quality which is only shared by a being which has consciousness in the first place. Objects (beings-in-itself) are present to a being-for-itself. But in a world without beings-for-itself (conscious beings), there would be no such thing as a present, or objects present to it.
  \end{enumerate}
  \item Now Sartre segways to a new section, where we investigate \textcquote[182]{sartre}{to which being does the for-itself make itself a presence?}
  \item \textcquote[183]{sartre}{Our presence to any being implies that we are linked to that being by an internal-connection; otherwise no link between the present and being would be possible. But this internal connection is negative: it denies, with respect to a present b eing, that it is that being to which it is present. Otherwise the internal connection would isappear into a straightforward identification.}
  \item I'm not sure how to quite summmarise this section, but it seems to me the goal of this is for Sartre to present the neccesity of negation in all of it's forms within the being of consciousness (i.e. the being-for-itself). Negation is neccessary for the past, for the present, and as we shall soon see, for the future as well.

  \subsubsection*{The Future}
  As a quick summary, Sartre's conception of the future likewise derives its ontological foundation from the negative element present within the being-for-itself (i.e. consciousness). Where the future cannot and does not come from neither a simple material relation or quality, nor does it come from a simple quality of the being-for-itself. But rather, it is that negative aspect, an \emph{lack}. The best way I can understand this argument is that just as the being of the for-itself flees from the past (because the past is what it's not), the for-itself has to flee \emph{towrards} something -- and it would not be inaccurate to call that thing to which it flees toward the \emph{future}.

  \item \textcquote[184]{sartre}{Let us note first that the in-itself cannot be the future, and nor can it contain any part of the future. When I look at this crescent moon, the full moon is in the future only \enquote{within the world} that is disclosed to human reality: it is through human reality that the future arrives in the world. In itself, this quarter of the moon is what it is. Nothing in it as potentiality. It is in actuality.}
  \item \textcquote[185]{sartre}{Even were we to accept, as Laplace does, a complete determinism that would enable us \emph{to predict} a future state, this future circumstance would still need to be profiled against an antecedent disclosure of the future as such, a being-to-come of the world.}
  \item With the first two statements, it seems clear to me that the future has to have a distinct, unique ontological existance -- that cannot be dismissed away as a simple property of the material world. This understanding is developed more concretely in Sartre's subsequent sections.
  \item \textcquote[185]{sartre}{Only a being who has to be its being, rather than merely being it, can have a future}
  \begin{enumerate}
    \item By \emph{a being who has to be its being}, Sartre is talking about the specific, ontologically distinct kind of being whose own being \emph{possesses the question of its being}. In other words, this being is the \emph{Daesin}, or perhaps more generally, a conscious being (i.e. the being for-itself). This is in contrast to beings that are merely objects, i.e. beings of-itself.
  \end{enumerate}
  \item Sartre states in the following paragraphs that the future is not merely \enquote{representation}, nor is it mere \enquote{a futurising intention}. This excludes the more popular and common ontologies of future-ness.
  \begin{enumerate}
    \item In fact, Sartre is quite clear and unequivocal about how future-ness cannot be derived as a mere property of the material world, of mere beings-in-itself.
    \item Likewise, the future is not a simple property of just consciousness alone -- this is a more nuanced thesis:
    \item \textcquote[186]{sartre}{The for-itself can neither be \enquote{pregnant with the future}, nor an \enquote{awaiting of the future}, except against the ground of an original and prejudicative relation of the self to itself.}
  \end{enumerate}
  \item So what is the future, under Sartre's phenomenological conception of ontology? There's a specific argument that Sartre makes, where the future derives it's being from a certain negative presence of the being-for-itself. I'll try to present this argument to the best of my understanding:
  \marginnote{This inversion of causes present in his conception of the future is very interesting. In fact, Sartre calls it \enquote{casuality in reverse \ldots\ the efficient power of a future state.} Whaat does this mean for causes in general?}
  \begin{enumerate}
    \item \textcquote[186]{sartre}{Let us take a simple example: this position which I keenly take up on the [tennis] court has meaning only through the movement I will make next, with my racket, to send the ball back over the net. But I am obeying neither my \enquote{clear representation} of the future movement, nor my \enquote{firm resolution} to accomplish it \ldots\ it is my future movement which, without even being thematically presented, turns backward to the positions I adopt, in order to illuminate, to connect, and to modify them.}
    \item \textbf{\textcquote[186]{sartre}{There is not a moment of my consciousness that is not similarly defined by an internal relation to a future; whether I write, I smoke, I drink, or I rest, the meaning of my [acts of] consciousness is alwyas at a distance, over there, outside.}}
    \item \textcquote[187]{sartre}{The future is \emph{what I have to be} insofar as I cannot be it.}
  \end{enumerate}
  \item \textcquote[187]{sartre}{Recall that the for-itself, confronted with being, presentifies itself as not being that being, and as having been its past. This presence is flight, because, in fleeing from the being that it is not [i.e. the past], presence flees from the being that it was. \emph{What} does it flee toward? Let us not forget that the for-itself, insofar as it presentifies itself to being in order to flee from it, is a lack \ldots\ From this we can grasp the meaning of the flight involved in presence: it is a flight towards \emph{its being}.}
  \item \textcquote[191]{sartre}{[The for-itself reaches the future] in vain: the for-itself can only ever be its future problematically, because it is separated from it by the nothingness that it is. In brief, the for-itself is free, and its freedom sets its own limit to itself. To be free is to be condemned to be free. \emph{Thus the future, insofar as it is the future, has no being.} It is not \emph{in itself} and nor does it have the for-itself's mode of being either, since it is the for-itself's \emph{meaning}. The future is not; it \emph{possibilises} itself.}
\end{enumerate}

\subsubsection{The Ontology of Temporality}

After examining the tripartite division of \emph{the past}, \emph{the present}, and \emph{the future,} Sartre turns to examine the ontology of temporality itself. In this section, he begins with the following dichotomy of \emph{static temporality} and \emph{dynamic temporality}, where:

\begin{enumerate}
    \item \textbf{Static Temporality:} The elements of \emph{before} and \emph{after}.

    \textcquote[193]{sartre}{What Kant calls the \emph{order} of time.}

    \item \textbf{Dynamic Temporality:} The fact of succession, the motion of how every after becomes a before.

    \textcquote[193]{sartre}{What Kant calls the \emph{course} of time.}
\end{enumerate}

\noindent
Sartre separates the two and begins an examination of each individually. We will begin with static temporality.

\begin{enumerate}
  \subsubsection*{Static Temporality}
  \item \textcquote[193]{sartre}{The \enquote{before-after} order [of static temporality] is defined in the first place by its irreversibility. We call a series of \enquote{successive} if its terms can only be considered one by one, and in only one direction.}
  \item \textcquote[193]{sartre}{Without the succession of \enquote{others} I could be what I want to be straightaway, and there would no longer be any distance between me and myself, or any separation between an action and a dream.}
  \item It is this very atomic separate-ness of the temporality of \emph{instants} that yields the ontological problem of temporality. After all, by reducing every moment to an instant, the casual order between instants seem to disappear. Sartre summarises this problem by stating:
  \item \textcquote[194]{sartre}{Thus, when we consider in isolation temporality's power to dissolve, we are forced to admit that \emph{the fact of having existed at any given instant does not constitute the right to exist at the folowing instant}, nor even mortage or an option on the future.}
  \item At this point, Sartre exams three competing solutions to this problem of succession and order in atomic temporal ontology:
  \begin{enumerate}
    \item \textbf{Kant}: Tries to resolve this by making the witness of time (i.e. the being who experiences time) temporal, and by having time come from a transcendental relationship of the witness towards God.
    \item \textbf{Descartes}: Same as Kant, except the ultimate unifying act of the witness with the temporal object is the \emph{I think} of reason.
    \item \textbf{Leibniz}: Rejects Kant and Descarte, and attempts to view all time as \enquote{pure relation of immanence and cohesion}, where time is continuous and not atomic at all.
  \end{enumerate}
  \item Ultimately, he finds all three approaches to be lacking and/or inadequate in some way.
  \item \textcquote[198]{sartre}{How can a timeless being, having to unify timeless elements, conceive of the kind of unification that belongs to succession? And if -- as we would need to agree in that case -- the \emph{esse} of time is a \emph{percipi}, how will the \emph{percipitur} be constituted? \ldots\ Thus, insofar as [time] is at the same time a form of separation and a form of synthesis, temporality will not permit us either to derive it from something timeless or to impose it \emph{from outside} on timeless things.}
  \item In that vein of questioning, both Sartre and the reader asks: \enquote{Who \emph{draws} time?}
  \item \textcquote[200]{sartre}{What may we conclude, at the end of this discussion? In the first place, this: temporality is a force that dissolves, but it does so within an act of unification; it is not so much a real multiplicity [but] as a quasi-multiplicity, the first draft of a dissociation within unity}
  \marginnote{Once again, we see the theme of nihilation in this act of dissolution. How does this relate to the broader theme of being arising from nothingness?}
  \begin{enumerate}
    \item \textcquote[200]{sartre}{Time cannot be a real multiplicity for it could not subsequently receive any unity and could not, in consequence, even exist in the form of real multiplicity}
    \item \textcquote[200]{sartre}{If we start by positing temporal unity, we are at risk of no longer even being able to understand anything about the irreversible succession as the \emph{meaning} of this unity.}
    \item \textcquote[200]{sartre}{We must conceive [temporality] as a unity that multiplies \emph{itself}, which means temporality can only be a relation of being \emph{within the same being.}}
  \end{enumerate}
  \item \textcquote[200]{sartre}{\textbf{Temporality is \emph{not}.} Only a being with a specific structure of being can, in the unity of its being, be temporal. \enquote{Before} and \enquote{After} is intelligible only as being what is \emph{before} itself.}
  \begin{enumerate}
    \item \textcquote[201]{sartre}{Rather, the for-itself, in existing, temporalises itself.}
  \end{enumerate}
  \marginnote{What sort of being does Sartre refer to? Is this likewise the conscious being, the being-for-itself which contains negation \enquote{in its heart like a worm?} Likewise, how does this relate to, or differ from ipseity?}
  \item \textcquote[201]{sartre}{Temporality must have the structure of ipseity.}

  \subsubsection*{The Birth/Emergence of Consciousness in Temporality}

  Around pages \autocite[204]{sartre} of the preceding section, Sartre goes on a parallel, but \emph{deeply} fascinating tangent on the absolute \emph{neccessity} of temporality for the being-for-itself. It begins with the question of \enquote{How can temporal things have a definite \emph{beginning point}?}, otherwise called the \enquote*{Problem of Birth} and ends up as a deeper investigation on the ontology of emergence.
  \marginnote{What parallels can we draw from the emergence of consciousness in temporal ontology, to the artificial creation of conscious minds?}

  \item \textcquote[203]{sartre}{In effect, it strikes us as scandelous that consciousness should come at some moment \enquote{appear} and should come to \enquote{inhabit} the embryo, or in short, that there should be one moment in which the living thing, as it develops, lacks any consciousness and another moment in which a consciousness without any past becomes imprisoned within it.}
  \item In order to resolve this paradox, Sartre takes his previous statements about the nature of temporality in being, and posits that absolutely it is impossible for any conscious being (being-for-itself) to lack a past.
  \item \textcquote[204]{sartre}{The for-itself's being is originally constituted by this relation to a being that is \emph{not} consciousness, existing within the complete night of identity, that the for-itself is, however, outside itself, behind itself}
  \begin{enumerate}
    \item \textcquote[204]{sartre}{The in-itself is what the for-itself was \emph{before}. In consequence, it makes perfect sense that our past does not appear to us as if it were limited by a clean line, with no smudges}
  \end{enumerate}
  \item \textcquote[204]{sartre}{There is no ontological problem: we do not have to ask ourselves how a consciousness can be born, because consciousness can appear to itself only as the nihilation of in-itself, i.e. as \emph{having already been born.}}

  \subsubsection*{The Temporal Dynamic}
  The central theme of this section seems to be to answer the question: \enquote{where does the dynamicism of temporality as a continuous progression/succession come from?} I'm not sure if I understand the entirety of the answer, but Sartre does present an interesting approach to this question from the perspective of change.

  \item In the first segment of Sartre's rhetorical progression, he explores the common idea of temporal progression as \emph{change}, either as a symptom of change, or as something which emerges from change. Specifically, he responds to the idea that \textcquote[208]{sartre}{temporality is reduced to being no more than the measure and order of change. Without change there would be no temporality, since time would have no purchase on the permanent and identical}. Sartre \emph{rejects} this conception, calling it one \enquote{based on many mistakes}.
  \item \textcquote[209]{sartre}{In brief, the change's \emph{unity} with the permanent is necessary for the constitution of a change as such.}
  \item \textcquote[209]{sartre}{The appeal to permanence in order to found change is, moreover, utterly useless. The idea is to show that any absolute change is, strictly speaking, no longer a change, since \emph{nothing} remains that is changing -- or in relation to which there could be change.}
  \item \textcquote[209]{sartre}{But in addition, when we are dealing with human-reality, what is necessary is pure and absolute change, which moreover is perfectly able to be a change while \emph{nothing} changes -- and which is duration itself. Even if we allowed that a for-itself could be an absolutely empty presence to a permanent in-itself-- \emph{the very existence of that consciousness would imply temporality,} since it would have to be, without changing, what it is, in the form of \enquote{having been it.}}

  \noindent
  Now at the start of \autocite[211]{sartre} Sartre begins his presentation on the main body of his argument, which is that dynamic temporality derives its motive force from the process of the being-for-itself fleeing from a in-itself past towards the future. He uses the image of \enquote{a hole constantly being filled}.
  \item \textcquote[210]{sartre}{The present cannot \emph{pass} [into the past] except by becoming the \enquote{before} of a for-itself that thereby constitutes itself as \enquote{after.} There is therefore just one phenomenon: the arising of a new present that \enquote{pastifies} the present that it \emph{was} and in the wake of the past-ification of a present, the appearing of a for-itself for whom that present will become the past.}
  \item We must take especial care and attention towards the generation of a \enquote{for-itself for whom that present will become the past.} It seems intimately related to the specific ontological being of past-pasts, and past-futures -- which Sartre presents shortly following:
  \begin{enumerate}
    \item \textcquote[211]{sartre}{The past of the present that has undergone in its past-ification becomes the past of a past -- or the \emph{pluperfect.} In relation to it, the present's heterogeneity with the past is immediately eliminated.}
    \item \textcquote[211]{sartre}{On the other hand, although the future is equally affected by the metamorphosis it does not cease to be the future -- which means it remains outside the for-itself, in front of it, beyond being -- but it becomes the future of a past, or the \emph{future-perfect}.}
  \end{enumerate}
  \marginnote{What is the exact relationship or ontological difference between the \enquote*{past of a past} (i.e. the pluperfect), and the \enquote*{future of a past} (i.e. the future-perfect)?}
  \item It's clear that the ontology of a past-of-a-past is essentially different from the ontology of a future-of-a-past, where the future-of-a-past still possesses some transcendental nature.
  \item  \textcquote[211]{sartre}{The connection between the past and the pluperfect is a connection in the mode of the in-itself and it appears upon the foundation of the present for-itself}
  \item At this point I am at able to understand the following thematic idea: As the future moves in this dynamic motion into the past, the future loses the transcendental property (which I admittedly still cannot properly define) but becomes a matter of the \emph{in-itself}. The past is absolutely fixed, and \emph{objective} -- it takes no part in the being of the living \emph{being-for-itself}, but is merely the mundane \emph{being-in-itself}.
  \item  \textcquote[213]{sartre}{The past is a backward fatality: the for-itself makes itself what it wants, but [the for-itself] cannot escape the necessity that a new for-itself will be, irremediably, what it wanted to be.}
  \item  \textcquote[211]{sartre}{The past, therefore, is a for-itself \emph{that has ceased to be a transcending presence to the in-itself. As itself \emph{in itself}, it has fallen \emph{into the midst of the world.}}}
  \item \textcquote[213]{sartre}{In the past the world hems me in and I become lost within a universal determinism, but I radically transcend my past toward my future, to just the extent to which I \enquote{was} that past.}
  \item \textcquote[213]{sartre}{What is the meaning of this arising of the for-itself? We must be careful not to regard it as the appearance of a new being. It is as if the present were a constant hole in being which, the moment it is filled in, constantly reappears: as if the present were in constant flight from the threat of becoming bogged down in \enquote{in itself,} a threat that continues until the in-itself's final victory, which drags it into a past that is no longer any for-itself's past. This victory is death, because death puts a radical stop to temporality, by past-ifying the entire system, or alternatively, by the in-itself's seizing back of the human totality.}
  \item \textcquote[216]{sartre}{The time of consciousness, therefore, is human-reality temporalising itself as a totality that is its own unfinished task; it is nothingness, sliding into a totality like a detotalising enzyme. This totality is simultaneously chasing after, and rejecting itself; it is unable to find any final term within itself for its surpassing, because it is its own surpassing, and surpasses itself towards itself; such a totality cannot, in any case, exist within the limits of an instant.}
  \end{enumerate}

\subsubsection{Original Temporality and Psychological Temporality: Reflection}

This is a great section, which serves both as a synthesis of the previous tri-partite division of temporality, as well as a means of deriving psychological reality from temporality. In summary, the following process seems to occur. First, Sartre asks the question of \enquote{How does the being-for-itself actually perceive the passage of time?,} and in order to answer this question he makes the distinction between \emph{original temporality} and \emph{psychological temporality}.

Original temporality is the abstract, ontological synthesis of the past, present, and the future.
\marginnote{Once again, I would like to have a better understanding on the process of the ecstatic synthesis of the three temporal ecstasies.}
It is the thing that we talk about when we are operating purely in the level of meaning and being, insofar the nature of the being-for-itself is concerned. Original temporality is concerned purely with the non-thetic nature of the being-for-itself (i.e. the un-conscious conscious). He talks about this division starting at \autocite[217]{sartre}

However, there is an impure derivative of original temporality, which comes as soon as the relation between the being-for-itself and temporality becomes \emph{thetic}. When the positional consciousness of the mind \emph{posits} about time, and the passage of time, all of a sudden the process becomes different. This is akin to moving one step down on the ladder of metaphysics. In order to answer what is psychological temporality, and indeed even distinguish the difference between the two -- Sartre has to go on a slight detour to talk about the nature of \emph{reflection} itself. Because remember -- if original temporality is purely a component of the being-for-itself, then in order for there to be any positional (i.e. psychological) awareness of temporality, the positional consciousness has to have \emph{itself} (i.e. the consciousness) as its object of reflection. Hence we must first understand reflection, in order to properly talk about the two.

\begin{enumerate}
  \item \textcquote[217]{sartre}{The for-itself endures in the form of a non-thetic consciousness (of) enduring. But I am a ble to \enquote{feel time passing} and to apprehend myself as a unity of scucession. In this case I am conscious \emph{of} enduring. This consciousness is thetic and closely resembles knowledge, just as duration being temporalised before my eyes comes close to being an object of consciousness. \emph{What kind of relation exists between original temporality and this psychological temporality,} that I encounter as soon as I apprehend myself \enquote{in the process of enduring?}}
  \item \textcquote[218]{sartre}{Reflection is the for-itself as conscious \emph{of} itself.}
  \item \textcquote[220]{sartre}{For a consciousness, to become reflected is to undergo a deep modification in its being and precisely to lose the \emph{selbstständigkeit} it possessed.}
  \item \textcquote[217]{sartre}{The person who reflects on me is not some kind of  pure, timeless gaze: it is myself, an enduring me, who is commited within the circuit of my ipseity, in danger within the world, and with my historicity.}
  \item \textcquote[224]{sartre}{Reflection is knowledge; that is beyond doubt; it possesses a positional character, and it affirms the consciousness it reflects on.}
  \item \textcquote[224]{sartre}{[Reflection's] knowledge is totalising: it is a lightning intuition without any contrasts, or any point of departure, or any point of arrival. Everything is given at once in a kind of absolute proximity.}
  \item \textcquote[224]{sartre}{But if reflective consciousness \emph{is} what it reflects on -- if this unity of being founds and limits the authority of reflection -- we ought to add that what we reflect on \emph{is} itself its past and its future.}
  \begin{enumerate}
    \item \textcquote[224]{sartre}{We can reach this conclusion, moreover, from the fact that \emph{thinking} is an act that commits the past, and is sketched out in advance by the future. \enquote{I doubt, therefore I am,} said Descartes.}
    \marginnote{The object of reflection is fundamentally temporal in nature, since the result of it's discernment requires the separation of the past and the future.}
    \item \textcquote[225]{sartre}{For there to be doubt, it is necessary for this suspension to be motivated by an insufficiency of reasons either to assert or to deny -- which refers to the past -- and that should be deliberately maintained until new elements intervene -- which is already a project of the future.}
  \end{enumerate}
  \item \textcquote[225]{sartre}{Now, if our findings are correct, reflection is a for-itself seeking to reclaim itself as a totality that is constantly in a state of incompleteness.}
  \begin{enumerate}
    \item \textcquote[225]{sartre}{Reflection, as a mode of being of the for-itself, must be in the form of temporalisation and that it is, itself, its past and its future.}
    \item \textcquote[255]{sartre}{That, by virtue of its nature, its authority and certainty extend as far as the possibilities that \emph{I am} and to the past that \emph{I was}.}
  \end{enumerate}
  \item \textcquote[226]{sartre}{Thus reflection is a consciousness \emph{of the three} ecstatic dimensions. It is a \emph{non-thetic consciousness} of flowing, and a \emph{thetic consciousness} of duration. For it, the past and the present of what it reflects on begin to exist as \emph{quasi-outsides}, in the sense that they are not only held within the unity of a for-itself that exhausts their being by having it to be, but also \emph{for} a for-itself that is separated from them by a nothingness.}
  \item \textcquote[227]{sartre}{Reflection, therefore, grasps temporality insofar as it is disclosed as the unique and incomparable mode of being of an ipseity, i.e., as historicity.}
\end{enumerate}

It is after having a good understanding of reflection do we come to understand original temporality, and then proceed to talk about pyschological temporality. I am very excited about Sartre's presentation of psychological temporality, because it is metaphysically interesting. Psychological temporality is based upon the original temporality that is a part of the being of the for-itself -- however, it is by neccessity derivative. I'm not sure if I understand the complete difference right now, but it seems like a pretty important concept behind psychological temporality is that our reflection on temporal states is in-complete -- there's some sort of \emph{nothingness} which seperates our being from the reflected being. In some sense, the objects of psycholgoical temporality are strictly beings \emph{in-itself}. Which does make sense, because whenever we conduct an act of reflection, we are naturally not creating a whole new consciousness from the mere act of reflection.

\begin{enumerate}
  \item \textcquote[228]{sartre}{This psychological duration constituted by the concrete flow of autonomous structures -- or, in other words, by the succession of psychological \emph{facts}, of \emph{facts} of consciousness -- cannot be called an illusion: \textbf{indeed their reality provides the object of psychology}; in practical terms, it is at the level of psychological fact that concrete relations between men \ldots\ are established.}
  \item \textcquote[229]{sartre}{We find ourselves therefore in the presence of two temporalities: original temporality, of which we \emph{are} the temporalisation; and psychological temporality, which exists at the same time both as something incompatible with our being's mode of being, and as an \emph{intersubjective reality}, an object of science, a goal of human actions.}
  \item \textcquote[229]{sartre}{At this point we need to distinguish pure reflection from impure or constituting reflection, \emph{because it is impure reflection that constitutes the succession of psychological facts, or psychè.} And in daily life, impure or constituting reflection is given first, even though it incorporates pure reflection within it as its original structure.}
  \item \textcquote[232]{sartre}{Thus, \textbf{reflection is impure when it presents itself as an \enquote{intuition of the for-itself in the in-itself;}} what is disclosed to it is not the temporal and insubstantial historicity of what it reflects on but -- beyond what it reflects on -- the actual substantiality of organised forms within the flow. The unity of these virtual beings is called \emph{a psychological life,} or the \emph{psychè,} a virtual and transcendent in-itself that subtends the for-itself]s temporalisation. Pure reflection is only ever a quasi-knowledge, but the only thing of which reflective knowledge is possible is the \emph{Psychè}.}
\end{enumerate}

\noindent
It is this degree of \emph{objectivity} which creates a whole new \emph{psychological world}, which contains its own facts and happenings.

\begin{enumerate}
  \item \textcquote[233]{sartre}{We should understand \emph{acts} as all of a person's synthetic activity, i.e., every ordering of means in view of ends -- not insofar as the for-itself is its own possibilities but insofar as the act represents a transcendent psychological synthesis that the for-itself is obliged to live.}
  \begin{enumerate}
    \item \textcquote[235]{sartre}{In short, the only way of presentifying these qualities, states, or acts is by apprehending them through a reflected consciousness, whose shadow they cast into the in-itself, in which they are objectified.}
    \item \textcquote[236]{sartre}{The psychological object, as the shadow cast by the reflected for-itself, possesses the characteristics of consciousness in degraded form.}
  \end{enumerate}
  \item \textcquote[233]{sartre}{The term \enquote{psychological} applies exclusively to a special category of  cognitive acts: the acts of the reflective for-itself.}
  \item \textcquote[242]{sartre}{In this way, reflective consciousness is constituted as a consciousness \emph{of} duration, and in consequence, psychological duration appears to consciousness. This psychological temporality, as the projection of original temporality into the in-itself, is a virtual being whose phantom flowing endlessly accompanies the for-itself's ecstatic temporalisation, insofar as reflection grasps this latter.}
  \item \textcquote[243]{sartre}{As soon as one takes up the standpoint of impure reflection -- the kind of reflection that seeks to determine the being that I am -- an entire world appears, to populate this temporality. This world -- a virtual presence, and the probable object of my reflective intention -- is the psychological world, or \emph{psychè}.}
  \marginnote{What is the relationship and neccessity of there being an \enquote{Other,} which allows the actualisation of an \enquote{Outside?}}
  \item \textcquote[243]{sartre}{And with this transcendent world, which takes up residence in the infinite becoming of antihistorical indifference, the temporality that we refer to as \enquote{internal} or \enquote{qualitative} -- which is an objectification of original temporality into in-itself -- is constituted, precisely, as a virtual unity of being. Here we find the first draft of an \enquote{outside} on itself, in its own eyes: but this \enquote{outside} is purely virtual. Later on we will see how being-for-the-Other \emph{actualises} the first draft of this \enquote{outside.}}
\end{enumerate}

\subsection{Chapter 3: Transcendence}

\subsubsection{Knowledge as a Type of Relation Between the For-Itself and the In-Itself}

\begin{enumerate}
  \item To be completed later.
\end{enumerate}

\subsubsection{On Determination as Negation}

\begin{enumerate}
  \item To be completed later.
\end{enumerate}

\subsubsection{Quality and Quantity, Potentiality and Equipmentality}

\begin{enumerate}
  \item To be completed later.
\end{enumerate}

\subsubsection{World-Time}

\begin{enumerate}
  \item To be completed later.
\end{enumerate}

\subsubsection{Knowledge}

\begin{enumerate}
  \item To be completed later.
\end{enumerate}


\section{Part III: Being-For-The-Other}

\subsection{Chapter 1: The Other's Existence}

\subsubsection{The Problem}

\begin{enumerate}
  \item To be completed later.
\end{enumerate}

\subsubsection{The Reef of Solipsism}

\begin{enumerate}
  \item To be completed later.
\end{enumerate}

\subsubsection{Husserl Hegel, Heidegger}

\begin{enumerate}
  \item To be completed later.
\end{enumerate}

\subsubsection{The Look}

\begin{enumerate}
  \item To be completed later.
\end{enumerate}

\subsection{Chapter 2: The Body}

\subsubsection{The body as Being-For-Itself: Facticity}

\begin{enumerate}
  \item To be completed later.
\end{enumerate}

\subsubsection{The Body-for-the-Other}

\begin{enumerate}
  \item To be completed later.
\end{enumerate}

\subsubsection{The Third Ontological Dimension of the Body}

\begin{enumerate}
  \item To be completed later.
\end{enumerate}

\subsection{Chapter 3: Concrete Relations with the Other}

\subsubsection{Our First Attitude Toward the Other: Love, Language, Masochism}

\begin{enumerate}
  \item To be completed later.
\end{enumerate}

\subsubsection{Our Second Attitude Toward the Other: Indifference, Desire, Hatred, and Sadism}

\begin{enumerate}
  \item To be completed later.
\end{enumerate}

\subsubsection{\enquote{Being-With} (\emph{Mitsen}) and the \enquote{We}}

\begin{enumerate}
  \item To be completed later.
\end{enumerate}


\chapter{Part IV: To Have, To Do, and To Be}

So what is \textsc{Part IV} about? In the introduction and the three parts, we've explored the ontology of the in-itself (i.e. the being-of-phenomena), the ontology of the for-itself, and finally the ontology of the Other -- as well as all their interleaving relationships such as the relationship of the for-itself to the in-itself (knowledge) or the relationship of the for-itself to the Other (the being-for-the-Other). With the above ontology established, we are finally able to explore the nature of the being-for-itself as an agent that \emph{acts}. It is this exploration of action that serves as the theme of Part IV. This part serves as the final third of Sartre's monograph on Phenomenological Ontology: \emph{Being and Nothingness}. Sartre introduces this part to us with the following opening-question:

\textcquote[567]{sartre}{Having, doing, and being are the fundamental categories of human reality. Every type of human behaviour can be subsumed within them. \ldots\ Is the supreme value of the human action \emph{to do} or \emph{to be}?}

\section{Chapter 1: Being and Doing: Freedom}

In this chapter, we are interested in the ontology of freedom. To begin this investigation, we first look at \emph{actions} -- which are things that our being partakes (i.e. our being \emph{acts}). This first chapter of \textsc{Part IV} is actually the much longer (and in my opinion, more important) chapter of the two chapters in this Part.

\subsection{The First Condition of Action is Freedom}

From where does an action derive its being? Sartre rejects the naively rationalist perspective that actions simply emerge from a series of contingent reasons (even erroneous reasons). Rather, he claims that no material contingency can bring about, or generate the impetus for action -- this is because any material contingency will simply be an objective description of the world in its facticity -- there will be no \emph{lack} that yields the space for action \autocite[574]{sartre}. Instead, action has to come from a \emph{nothingness} -- and that nothingness, which is the same nothingness that founds our lack and desire -- is the source of our freedom. Hence, in order to understand action, we must understand freedom -- the first condition of action.

\begin{enumerate}
  \item \textcquote[569]{sartre}{The point we should note at the outset is that an action is, by definition, \emph{intentional}.}
  \item \textcquote[570]{sartre}{An action necessarily implies, as its condition, some recognised \enquote{desideratum} i.e., an objective lack or even a negatity.}
  \item \textcquote[575]{sartre}{It is the act that determines its end and its motives, and the act is the expression of freedom.}
  \item \textbf{Relationship between Freedom, Unfreedom, For-itself, and In-Itself:}
  \begin{enumerate}
    \item \textcquote[577]{sartre}{The innermost meaning of determinism is to establish within us an unfailing continuity of existence in itself \ldots\ Thus, the rejection of freedom can be conceived only as an attempt to apprehend oneself as a being-in-itself}
    \item \textcquote[578]{sartre}{Freedom coincides with the nothingness that lies at man's heart. It is because human-reality \emph{is not enough} that it is free, because it is constantly separated from itself, and because a nothingness that separates what it has been from what is, and from what it will be.}
    \item \textcquote[578]{sartre}{Man is free because he is not an [in-] itself but self-presence. \emph{A being that is what it is cannot be free.}}
  \end{enumerate}
  \item \textcquote[579]{sartre}{Human-reality is entirely abandoned, without any help of any kind, to the unbearable necessity of making itself be, right down to the last detail: In this way freedom is not \emph{a} being: it is man's being, i.e. his nothingness of being.}
  \item \textcquote[579]{sartre}{Man cannot be sometimes free and sometimes a slave: he is free in his entirety and always, or he is not.}
\end{enumerate}

What is this understanding of freedom? Freedom comes from nothingness -- not just any nothingness, but the very nothingness that lies inside our being-for-itself, like a worm in its heart. This is a \emph{very} strong claim for Sartre to make, because all of a sudden freedom is not a contingent, material property of a human -- but rather, our freedom is founded on our very being, on our ontology. Human beings are not born free, only to lose it later on. Human beings -- so far as we can \emph{be} in any case -- have the ontology of freedom. As this freedom comes from the same nothingness which serves as a foundation of our being as a being-for-itself, it means that consciousness implies freedom. Anything else would be objectivity (a being-in-itself), and hence absolute determinism.

Sartre goes on to defend and elaborate on this audacious claim in the subsequent pages, with a special focus on the idea of \enquote{passions} \autocite[581 -- 585]{sartre} encroaching or affecting our freedom. He rejects passions as an external force upon our freedom -- because any admission of a passion being beyond freedom will make our freedom deterministic to it: \textcquote[581]{sartre}{This discussion shows that only two solutions, and only two are possible: either man is entirely determined \ldots\ or, indeed, man is entirely free.}

Likewise, with this rejection of passions having any special hold over our freedom, Sartre also rejects reason as having any special hold over our freedom. Sartre makes the distinction between reasons and motives, but ultimately both are equally non-essential to our freedom. In fact, Sartre states that \textcquote[585]{sartre}{In relation to freedom, no psychological phenomenon [either passion nor reason] is favoured. All my \enquote{ways of being} manifest it equally, since they are all ways of being my own nothingness.}

\begin{enumerate}
  \item \textcquote[591]{sartre}{In fact, reasons and motives only have the weight that my project -- i.e., the free production of the end, and of the act as having to be actualised -- confers on them. When I deliberate, the die is already cast. And if I must come to deliberate, it is simply because it is a part of my original project to take account of my motives \emph{by the means of deliberation} rather than through this or that other means of discovery.}
  \item Essentially, reasons and motives only bind our action, insofar as we ascribe their importance to them (roughly speaking).
\end{enumerate}

\noindent
Once again, freedom comes from the core of our being. Sartre illustrates this quite well using an image involving a hike with friends at \autocite[595]{sartre} that deserves revisiting:

\begin{enumerate}
  \item \textcquote[596]{sartre}{[When giving up to fatigue] is not a contemplative apprehension of my fatigue [i.e. \enquote{thought}]; rather -- as we saw in relation to pain -- I suffer my fatigue [i.e. a relation of being].}
  \item Essentially, Sartre is telling us that we \emph{act} in a certain way, because we \emph{be} in a certain way. This line of reasoning is much clearer on an examination of \autocite[596 -- 600]{sartre}.
  \item He presents another example of this in effect, by looking at a person with an \enquote{inferiority complex} which manifests in certain ways:
  \item \textcquote[601]{sartre}{This inferiority, which I struggle against and yet recognise, was \emph{chosen} by me at the outset \ldots\ to give in to fatigue, for example, is to transcend the path still to be covered by constituting it with the meaning \enquote{the path that is too difficult to follow.}}
  \item \textcquote[602]{sartre}{Thus the inferiority complex is a free and global project of myself, as inferior next to another; it is the way in which I choose to take on my being-for-the-Other, the free solution that I find for the insurmountable scandal of the other's existence.}
\end{enumerate}

\noindent
On how we are free to change our being:

\begin{enumerate}
  \item \textcquote[607]{sartre}{[all phenomena] which is in the end of the world which I am constantly conscious -- at least as the meaning implied by the object that I am looking at or using -- everything teaches me, myself, about my choice, i.e. about my [choice of] being.}
  \item \textcquote[607]{sartre}{Earlier we raised a question: I gave in to fatigue [in the example of the hike with friends], we said, and probably \emph{I could have done} otherwise, but at \emph{what cost}?} Or in other words, how are we free to change our choice, if our choice comes from our being?
  \begin{enumerate}
    \item \textcquote[607]{sartre}{We are now in a position to answer it Our analysis has just shown us, in effect that this act was not \emph{gratuitous.} Of course, it could not be explained by a motive or reason conceived as the content of an earlier \enquote{state} of consciousness; but it needed \emph{to be interpreted on the basis of an original project of which it formed an integral part}.}
    \item \textcquote[607]{sartre}{In consequence it becomes clear that we cannot suppose the action could have been modified without at the same time supposing \emph{a fundamental modification in my original choice of myself.}}
    \item \textcquote[607 -- 608]{sartre}{[Hence] I can refuse to stop only though a \emph{radical conversion of my being-in-the-world}, which is to say by a sudden metamorphosis of my initial project, which is to say by a different choice of myself and my ends.}
    \item \textcquote[608]{sartre}{Moreover, this modification is always possible. The anguish which, when it is disclosed, manifests our freedom to our consciousness testifies to this constant alterability of our initial project. In anguish, we do not simply grasp the fact that the possibles we are projecting are constantly eaten into by the freedom still to come; in addition, we apprehend our choice -- which is to say ourselves -- as being \emph{unjustifiable} [i.e. we can always change our being].}
  \end{enumerate}
  \item \textcquote[608]{sartre}{In this way we are constantly engaged in our choice, and constantly conscious of the fact that we ourselves can suddenly reverse this choice and change course, because we project the future through our very being, and we constantly eat away at it through our own existential freedom, declaring to ourselves by the means of the future what we are, and lacking any grip on this future, which remains always \emph{possible} without ever passing into the ranks of the \emph{real} Thus we are constantly \emph{threatened} with the nihilation of our current choice, constantly threatened with choosing ourselves -- and in consequence with becoming -- other than we are. Just because our choice is absolute, it is \emph{fragile}, which is to say that, by positing our freedom through it, we posit at the same time the constant possibility of becoming something that is \enquote{on this side} and pastified, in relation to an \enquote{over on that side} that I will be.}
\end{enumerate}

\noindent
For the remainder of the section, Sartre examines the various nuances and manifestations of freedom and choice being an element of our ontology. There's a particularly interesting section on bad faith and choice in \autocite[620]{sartre}, and he finally reiterates and summarises the  conclusions of this chapter in an eight-point summary at \autocite[622 -- 628]{sartre}. Having completed his elucidation on freedom, we move on to the next section.

\subsection{Freedom and Facticity: the Situation}

\subsubsection{My Place}

Pages \autocite[639 -- 646]{sartre}.

\begin{enumerate}
  \item To be completed later.
\end{enumerate}

\subsubsection{My Past}

Pages \autocite[646 -- 657]{sartre}.

\begin{enumerate}
  \item To be completed later.
\end{enumerate}

\subsubsection{My Surroundings}

Pages \autocite[657 -- 663]{sartre}.

\begin{enumerate}
  \item To be completed later.
\end{enumerate}

\subsubsection{My Fellow Man}

Pages \autocite[663 -- 689]{sartre}.

\begin{enumerate}
  \item To be completed later.
\end{enumerate}

\subsubsection{My Death}

Pages \autocite[689 -- 719]{sartre}.

\begin{enumerate}
  \item To be completed later.
\end{enumerate}

\subsection{Freedom and Responsibility}

Pages \autocite[719 -- 722]{sartre}.

\begin{enumerate}
  \item To be completed later.
\end{enumerate}

\section{Chapter 2: To Do and To Have}

\subsection{Existential Psychoanalysis}

\begin{enumerate}
  \item To be completed later.
\end{enumerate}

\subsection{To Do and To Have: Possession}

\begin{enumerate}
  \item To be completed later.
\end{enumerate}

\subsection{The Revelation of Being Through Qualities}

\begin{enumerate}
  \item To be completed later.
\end{enumerate}

\section{Conclusion}

\subsection{In-Itself and For-Itself: Some Metaphysical Observations}

\begin{enumerate}
  \item To be completed later.
\end{enumerate}

\subsection{Moral Perspectives}

\begin{enumerate}
  \item To be completed later.
\end{enumerate}


\section{Other Notes}


\appendix
% Uncomment for bibliography using biblatex (see includes/formatting.tex)
% \section*{Bibliography}
% Print every citation in citations.bib, even if unused by \autocite
% \nocite{*}
% \printbibliography[heading=none]

\section*{Technical Notes}
This essay is typeset using \LaTeX, an Open Source document typesetting language
by Donald Knuth, and version-controlled via Git. The git repository containing notes, source code, and revision history is available upon request.

% Optional: Include github URL here
% \url{https://github.com/ShenZhouHong/}

\noindent
This essay is written using the EssayTemplate, an open source \LaTeX\ essay
template designed for the Humanities by Shen Zhou Hong. It is available at:

\url{https://github.com/ShenZhouHong/EssayTemplate}

\vfill
\begin{center}
This \LaTeX\ essay is also available in Microsoft Word, OpenOffice, HTML, and \mbox{plain text} upon request.
\end{center}


\end{document}
