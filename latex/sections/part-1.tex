\section{Part I: The Problem of Nothingness}

\subsection{Chapter 1: The Origin of Negation}
Where does nothingness come from? This is the thematic question of chapter one. The general progression of this section's content and argument is as follows:

\begin{enumerate}
  \item First, we inquire about negation, which appears as a simple and rather direct phenomenon of nothingness.
  \begin{enumerate}
    \item After all, it's easy to posit positive being by making definite statements, e.g. \enquote{there is an apple.} However, when we make a negation of a definite statements e.g. \enquote{there is no apple,} it is clear that we are making a statement that's rooted in a certain conception of non-being.
  \end{enumerate}
  \item We ultimately conclude that it is not possible to derive negation from being, but instead negation must be derived from a certain definite non-being, i.e. \emph{nothingness}
  \item We try to investigate where this nothingness comes from. Sartre ultimately concludes that this \emph{nothingness cannot come from regular being, but it has to come from a being through which can be its own nothingness}
  \begin{enumerate}
    \item Sartre concludes that this being is the human being (i.e. the Daesin).
  \end{enumerate}
  \item Hence, in order to investigate nothingness, we have to investigate the human being.
  \item How is it possible for the human consciousness to experience nothingness? The answer to this is \emph{anguish}.
\end{enumerate}

\subsubsection{Questioning}
In this section, Sartre presents the necessity for non-being in all forms of questioning. Specifically:

\begin{enumerate}
  \item In every question, we confront a being that we interrogate.
  \item The answer to the interrogation can be either `Yes' (affirmative) or `No' (negative). To allow for an affirmative answer by necessity presupposes the possibility of a negative answer.
  \item After all, the very being of an affirmative answer is defined by its shadow, which is the negative part. When we say that X is Z, we are also simultaneously saying X is not A, B, C, \ldots\ Y.
  \item Hence, there is such a thing as non-being
\end{enumerate}

\noindent
With this, Sartre introduces the concept of \emph{non-being as a neccesary component of reality}, towards which we must investigate further throughout the rest of this section.

\subsubsection{Negations}
We begin our investigation into the nature of non-being by looking at \emph{negations}, which are easily observable manifestations (phenomenae?) of non-being.

\begin{enumerate}
  \item \textcquote[38]{sartre}{It is not true that negation is merely a quality of judgement.}
  \item In the process of questioning, we expect a being (which is the answer). But we can equally receive a non-being as a response. Sartre's analogy is that if a watch-maker questions a watch on why it's not working, it is perfectly plausible to receive a non-being as a response, e.g. the mainspring is missing.
  \item \textcquote[40]{sartre}{A being is \emph{fragile} if it bears within its being a clear-cut possibility of non-being.}
  \item It seems that nothingness is distinct from the process of thought which is negation. We need nothingness in order to separate beings from each other, but the thought process that is negation is more simple and less fundamental.
  \item Furthermore, Sartre claims that the thought process of negation must come from the being of nothingness.
  \item \textcquote[44]{sartre}{If there is being everywhere, it is not only nothingness that becomes inconceivable: from being we can never derive negation.}
\end{enumerate}

\subsubsection{The Dialectical Conception of Nothingness}
This section seems to be a general review of a dialectical (i.e. Hegelian) conception of nothingness. To be completed.

\begin{enumerate}
  \item Being and non-being are not contemporaries.
\end{enumerate}

\subsubsection{The Phenomenological Conception of Nothingness}

Sartre demonstrates that \emph{it is impossible to dismiss nothingness} as either a shadow of being, nor as the absence or something before or after being. But rather nothingness has a definite existence as a special sort of being, which has the characteristic of its own negation (which he will elaborate in a later section). This crux of this demonstration lies in the neccessity of defining positive being using negations (in his example with distance and lengths), as well as with the existence of \textbf{negatities}.

\noindent
In conclusion, nothingness cannot be outside of being.

\begin{enumerate}
  \item Sartre presents a few examples here on how it is absolutely impossible to abstract away the being of nothingness into a simple quality of regular (i.e. \emph{positive}) being:
  \begin{enumerate}
    \item \textcquote[55]{sartre}{Take for example the notion of distance \ldots\ it is easy to see the [distance] contains a negative moment: two points are distant when a specific length \emph{separates} them.}
    \item \textcquote[55]{sartre}{How might wish toreduce distance to being \emph{no more than} the length of the segments of which the [two points are] the limits \ldots\ [but] in this case we have switched the direction of our attention \ldots\ negation, expelled from the segment and its length, will take refuge in the two \emph{limits}.}
  \end{enumerate}
  \item This leads to the introduction of the existence of \textbf{negatities}, i.e. \emph{negative entities}. These things cannot be accounted for as positive being, but as beings which \textcquote[56]{sartre}{inhabited in their internal structure by negation as a necessary condition of their existence.}
  \item \textcquote[57]{sartre}{Nothingness can only nihilate itself on the ground of being: if nothingness can be given, it is neither before being nor after being; nor is it, in a general way, outside being; rather, it is right inside being, in its heart, like a worm.}
\end{enumerate}

\subsubsection{The Origin of Nothingness}

\begin{enumerate}
  \subsubsection*{The Origin of Nothingness in the Human Being}
  \item In this section, we try to find out where nothingness comes from, based on a set of restrictions which we demonstrated from the prior sections. I am going to rehash those prior conclusions again:
  \begin{enumerate}
    \item \enquote{Nothingness must be given in the heart of being}
    \item \textcquote[57]{sartre}{But being-in-itself [i.e. the being-of-phenomena] is not able to produce this intraworldly [transphenomenal (?)] nothingness: the notion of being as a full positivity does not contain nothingness as one of its structures.}
    \item \textcquote[57]{sartre}{Nothingness cannot be conceived of, either outside being or on the basis of being}
    \item Nothingness must have the power to nihilate itself, i.e. be the source of its own negation.
  \end{enumerate}
  \item Based on the above restrictions, where does nothingness come from? Nothingness must be the result of some being, but it cannot be the being of the phenomena.
  \item \textcquote[57]{sartre}{\emph{The being through which nothingness comes to the world must be its own nothingness.} And let us not construe this as an act of nihilation -- which would in turn require a foundation in being -- but as an ontological characteristic of the being which we are seeking.}
  \begin{enumerate}
    \item \textbf{Nihilation}: The act in which an object with being is negated.
    \item This latter qualification is important. If the being through nothingness comes is something that is nihilated -- it would need another being to support it, which leads to an infinite regression. There has to be a being in which nihilation is the ontological requirement.
  \end{enumerate}
  \item What sort of being is able to nihilate itself? The clue towards this being comes from the ability to \emph{question}. For the act of questioning requires negation, as we have determined earlier already-- but more importantly:
  \begin{enumerate}
    \item \textcquote[59]{sartre}{Every question posits, in its essence, the possibility of a negative answer. In a question we interrogate a being about its being, or its way of being. And this being, or way of being, is concealed; the possibility always remains open for it to be disclosed as nothingness.}
    \item \textcquote[59]{sartre}{But it follows, from the very fact of our envisaging that an existent can always disclose itself as nothing, that every question presupposes that \textbf{we have taken a nhilating step in relation to the given, which becomes mere \emph{presentation}, oscillating between being and nothingness}.}
    \begin{enumerate}
      \item This is the \enquote{permanent possibility in which the questioner is able to detach himself from the causal series that constitute being,} where:
    \end{enumerate}
    \item \enquote{In consequence, through a \emph{twofold movement of nihilation}, [the questioner] nihilates the thing he is questioning in relation to himself:}
    \begin{enumerate}
      \item \enquote{By placing it in a neutral state between being and non-being}
      \item \textcquote[59]{sartre}{[And also] by separating himself from being in order to draw out from himself the possibility of a non-being.}
    \end{enumerate}
    \item It seems clear to me that this act of questioning takes it's ontological characteristic as a \emph{twofold movement of nihilation}. Where to question, is to complete the neccessary two steps where:
  \end{enumerate}
  \item Sartre's conclusion is that \textbf{the being from which negatity (i.e. nothingness) comes from is the being of man, i.e. the being-of-consciousness.}
  \item Keep in mind that this process is not a process of \emph{generation}. \textbf{No being can generate \emph{non-being}, never!} But rather, it is \emph{a process of changing our being's relationship to another being}:
  \begin{enumerate}
    \item \enquote{To disconnect some particular existent, for human-reality (i.e. Daesin) is to disconnect [human reality] in relation to [the existent]. In this case, human-reality escapes the existent and cannot be acted on by it; it is out of reach, having withdrawn beyond a nothingness}
  \end{enumerate}
  \item Now that we conclude that the being which is the condition of negation and nothingness is the human being, i.e. the being-of-consciousness. But furthermore, we acknowledge that this is \emph{not} a process of generation, but rather one of changing our relationship to definite beings.
  \item \textbf{This process of bringing out nothingness in our being, is called \emph{freedom}.}

  \subsubsection*{Nothingness as Freedom, the Phenomena of Freedom as Angst}
  \item In this second half of the section, Sartre talks about the practical implications of the being-of-consciousness's ability to bring about nothingness as freedom. He first presents it more or less directly, but than he presents the consciousness-of-this-freedom as \emph{angst.}
  \item \textcquote[66]{sartre}{If nihilating consciousness exists only as a consciousness of nihilation, it ought to be possible to define and describe a constant mode of consciousness, present \emph{as} consciousness, that is the consciousness of nhilation.}
  \begin{enumerate}
    \item This \textbf{consciousness-of-nihilation} is the human emotion of \emph{angst}.
    \item There are two types of angst, properly speaking. There is both angst for the future, as well as angst for the past.
  \end{enumerate}
  \item A question that comes up at this stage is to ask: \emph{\enquote{How is angst different from \emph{fear}?}}
  \begin{enumerate}
    \item Fear is the emotion of worrying of \emph{external existents.}
    \item Angusih is the emotion of worrying about one's own \emph{being.}
  \end{enumerate}
  \item These definitions will be developed by further examples. For instance, a key component of anguish is the \emph{nihilation} or \emph{nothingness}. It's an awareness of the nothingness which conditions the vast array of possibilities of one's own being. Sartre talks more about this from pages 66 and onwards \autocite[66]{sartre}, with a particularly definite example on \autocite[71]{sartre} and \autocite[77]{sartre}.
  \item In \autocite[80]{sartre} Sartre talks about the ways in which one flees away from this anguish.
  \item In the very end of this chapter, we look at \enquote*{bad faith} -- which Sartre describes as the collection of behaviours (consciousness(es) (?)) in which we flee away from \emph{anguish}. It's important to Sartre that we examine bad faith next in our inquiry, for the following reasons:
  \begin{enumerate}
    \item Bad faith is paradoxical, since in order to flee away from anguish, we must aim at anguish itself \autocite[86]{sartre}. This means that the content of bad faith contains anguish.
    \item As a result, bad faith serves as a very good and direct proxy to understand what this anguish is, which will allow us to go further in our question of nothingness.
  \end{enumerate}
  \item As a sort of final sketch in this part of our inquiry, we can summarise the digression as follows:
  \begin{enumerate}
    \item Nothingness must exist, but cannot come from or be generated upon, or be founded by being.
    \item Nothingness is a relationship between two beings, where one being negates the other (?).
    \item The only being that is capable of this action of nihilation is the human being, i.e. the Daesin or the being-of-consciousness.
    \item The way in which consciousness is conscious of this act of nihilation is in the phenomena of \emph{angst}.
    \item We try to flee from angst through the application of bad faith. However, bad faith must contain the content of angst.
    \item Hence, finally -- in order to understand where being and nothingness comes from, we must examine bad faith as our proxy.
  \end{enumerate}
\end{enumerate}

\subsection{Chapter 2: Bad Faith}

\subsubsection{Bad Faith and Lies}

\begin{enumerate}
  \item \textbf{New Working Definition of Consciousness:} \textcquote[87]{sartre}{Consciousness is a being for whom in its being there is concsciousness of the nothingness of its being.}
  \item \textcquote[88]{sartre}{What must man be in  his being for it to be possible for him to negate himself?} Where self-negation serves as the foundation of bad faith, it seems.
  \begin{enumerate}
    \item \textcquote[88]{sartre}{We should choose and examine a specific attitude, essential to human-reality (i.e. Daesin), and in which, at the same time, consciousness, instead of directing its negation outward, turns it against itself. It has seemed to us that this attitude must be \emph{bad faith}.}
  \end{enumerate}
  \item Bad faith is not simply lying, or even some extended or fundamental form of lying. For lies \textcquote[89]{sartre}{requires no special ontological foundation.}
  \begin{enumerate}
    \item In ordinary lying, there is the liar, and the deceived.
  \end{enumerate}
  \item \textcquote[90]{sartre}{In bad faith it is from myself that I am concealing the truth. Thus the duality of the deceiver and the deceived is not present here. On the contrary, bad faith implies in its essence the unity of a single consciousness.}
  \item Sartre goes on to test, criticise, and ultimately reject the Freudian explanation for the foundation of bad faith. The Freudian explanation posits a trinity of the consciousness as the \emph{id, ego, and superego} -- upon which there's an interference between one of the two.
  \item For reasons that are not ultimately too important, the Freudian explanation is shown to be an inaccurate one at best.
  \item the ultimate conclusion here is that \textbf{bad faith must know the thing which it denys, in order to actively act in denial of it}
\end{enumerate}

\subsubsection{Forms of Bad Faith}

\begin{enumerate}
  \item In order to explore properly what bad faith is, Sartre takes us on an examination of various everyday-examples of bad faith in action.
  \item In \autocite[98]{sartre} Sartre presents an \emph{amazing} example of a form of `bad faith' in practice -- the dance of flirtation. It's really cool and you should totally check it out.
  \item \textbf{Characteristics of Bad Faith}
  \begin{enumerate}
    \item Forming contradictory concepts, where an idea and the negation of the idea are united.
    \begin{enumerate}
      \item i.e. \enquote{I am not what I am}
    \end{enumerate}
    \item The method in which we generate these contradictions is through \textcquote[99]{sartre}{the twofold property of human beings, of being a facticity and a transcendence.}
    \begin{enumerate}
      \item This seems to mean we accept (acknowledge?) a facticity, but then escape it through our transcendence (?)
    \end{enumerate}
  \end{enumerate}
  \item but Sartre also goes on to say that this facticity-transcendence dichotomy is not the only way in which we generate bad faith, but there are other ways?
  \item \autocite[102]{sartre} What does it mean to \emph{play} at acting something? Sartre presents another marvellously beautiful example, that of the cafe-waiter. \textbf{I should examine this scene in more detail.}
  \item Around \autocite[109]{sartre} Sartre goes on a digression about sincerity, in an attempt to understand bad faith through its contrast.
  \item \textcquote[110]{sartre}{At the same time, [through sincerity] the malice is defused, since if it only exists deterministically it is nothing, and since, by acknowledging it, I posit my freedom in relation to it; my future is virgin, so everything is permitted. In this way, sincerity's essential structure does not differ from that of bad faith, since the sincere man constitutes himself as what he is \emph{in order to not be it.}}
  \item Sincerity and Bad faith seems to be both two sides of the same coin, for they both require one to objectify oneself -- and to look on the self as an external object.
  \begin{enumerate}
    \item \textcquote[111]{sartre}{Sincerity [and bad faith] does not assign a particular quality or way of being to me, but in relation to the quality at issue, it aims to move me from one mode of being into another mode of being.}
  \end{enumerate}
\end{enumerate}

\subsubsection{The \enquote{Faith} of Bad Faith}

\begin{enumerate}
  \item Bad faith requires a specific position to be held on our epistemology. To be in the state of bad faith is to be in a state where we are willing to accept \emph{non-persuasive} evidence, since if the evidence was persuasive in the first place, we wouldn't be in bad faith.
  \begin{enumerate}
    \item \textcquote[114]{sartre}{This primary project of bad faith is a decision, in bad faith, about the nature of faith.}
  \end{enumerate}
  \item \textcquote[117]{sartre}{There is no cynical lie in bad faith, or any knowning preparation of misleading concepts. But \textbf{bad faith's most basic act is to flee from something that is impossible to flee from: to flee from what one is.}}
\end{enumerate}
