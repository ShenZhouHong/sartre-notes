\chapter{Part IV: To Have, To Do, and To Be}

\textcquote[567]{sartre}{Having, doing, and being are the fundamental categories of human reality. very type of human behaviour can be subsumed within them. \ldots\ Is the supreme value of the human action \emph{to do} or \emph{to be}?}

\section{Chapter 1: Being and Doing: Freedom}

\subsection{The First Condition of Action is Freedom}

\begin{enumerate}
  \item \textcquote[569]{sartre}{The point we should note at the outset is that an action is, by definition, \emph{intentional}.}
  \item \textcquote[570]{sartre}{An action necessarily implies, as its condition, some recognised \enquote{desideratum} i.e., an objective lack or even a negatity.}
\end{enumerate}

\subsection{Freedom and Facticity: the Situation}

\begin{enumerate}
  \item To be completed later.
\end{enumerate}

\subsection{Freedom and Responsibility}

\begin{enumerate}
  \item To be completed later.
\end{enumerate}

\section{Chapter 2: To Do and To Have}

\subsection{Existential Psychoanalysis}

\begin{enumerate}
  \item To be completed later.
\end{enumerate}

\subsection{To Do and To Have: Possession}

\begin{enumerate}
  \item To be completed later.
\end{enumerate}

\subsection{The Relevation of Being Through Qualities}

\begin{enumerate}
  \item To be completed later.
\end{enumerate}

\section{Conclusion}

\subsection{In-Itself and For-Itself: Some Metaphysical Observations}

\begin{enumerate}
  \item To be completed later.
\end{enumerate}

\subsection{Moral Perspectives}

\begin{enumerate}
  \item To be completed later.
\end{enumerate}
