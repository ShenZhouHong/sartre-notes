\chapter{Part IV: To Have, To Do, and To Be}

So what is \textsc{Part IV} about? In the introduction and the three parts, we've explored the ontology of the in-itself (i.e. the being-of-phenomena), the ontology of the for-itself, and finally the ontology of the Other -- as well as all their interleaving relationships such as the relationship of the for-itself to the in-itself (knowledge) or the relationship of the for-itself to the Other (the being-for-the-Other). With the above ontology established, we are finally able to explore the nature of the being-for-itself as an agent that \emph{acts}. It is this exploration of action that serves as the theme of Part IV. This part serves as the final third of Sartre's monograph on Phenomenological Ontology: \emph{Being and Nothingness}. Sartre introduces this part to us with the following opening-question:

\textcquote[567]{sartre}{Having, doing, and being are the fundamental categories of human reality. Every type of human behaviour can be subsumed within them. \ldots\ Is the supreme value of the human action \emph{to do} or \emph{to be}?}

\section{Chapter 1: Being and Doing: Freedom}

In this chapter, we are interested in the ontology of freedom. To begin this investigation, we first look at \emph{actions} -- which are things that our being partakes (i.e. our being \emph{acts}). This first chapter of \textsc{Part IV} is actually the much longer (and in my opinion, more important) chapter of the two chapters in this Part.

\subsection{The First Condition of Action is Freedom}

From where does an action derive its being? Sartre rejects the naively rationalist perspective that actions simply emerge from a series of contingent reasons (even erroneous reasons). Rather, he claims that no material contingency can bring about, or generate the impetus for action -- this is because any material contingency will simply be an objective description of the world in its facticity -- there will be no \emph{lack} that yields the space for action \autocite[574]{sartre}. Instead, action has to come from a \emph{nothingness} -- and that nothingness, which is the same nothingness that founds our lack and desire -- is the source of our freedom. Hence, in order to understand action, we must understand freedom -- the first condition of action.

\begin{enumerate}
  \item \textcquote[569]{sartre}{The point we should note at the outset is that an action is, by definition, \emph{intentional}.}
  \item \textcquote[570]{sartre}{An action necessarily implies, as its condition, some recognised \enquote{desideratum} i.e., an objective lack or even a negatity.}
  \item \textcquote[575]{sartre}{It is the act that determines its end and its motives, and the act is the expression of freedom.}
  \item \textbf{Relationship between Freedom, Unfreedom, For-itself, and In-Itself:}
  \begin{enumerate}
    \item \textcquote[577]{sartre}{The innermost meaning of determinism is to establish within us an unfailing continuity of existence in itself \ldots\ Thus, the rejection of freedom can be conceived only as an attempt to apprehend oneself as a being-in-itself}
    \item \textcquote[578]{sartre}{Freedom coincides with the nothingness that lies at man's heart. It is because human-reality \emph{is not enough} that it is free, because it is constantly separated from itself, and because a nothingness that separates what it has been from what is, and from what it will be.}
    \item \textcquote[578]{sartre}{Man is free because he is not an [in-] itself but self-presence. \emph{A being that is what it is cannot be free.}}
  \end{enumerate}
  \item \textcquote[579]{sartre}{Human-reality is entirely abandoned, without any help of any kind, to the unbearable necessity of making itself be, right down to the last detail: In this way freedom is not \emph{a} being: it is man's being, i.e. his nothingness of being.}
  \item \textcquote[579]{sartre}{Man cannot be sometimes free and sometimes a slave: he is free in his entirety and always, or he is not.}
\end{enumerate}

What is this understanding of freedom? Freedom comes from nothingness -- not just any nothingness, but the very nothingness that lies inside our being-for-itself, like a worm in its heart. This is a \emph{very} strong claim for Sartre to make, because all of a sudden freedom is not a contingent, material property of a human -- but rather, our freedom is founded on our very being, on our ontology. Human beings are not born free, only to lose it later on. Human beings -- so far as we can \emph{be} in any case -- have the ontology of freedom. As this freedom comes from the same nothingness which serves as a foundation of our being as a being-for-itself, it means that consciousness implies freedom. Anything else would be objectivity (a being-in-itself), and hence absolute determinism.

Sartre goes on to defend and elaborate on this audacious claim in the subsequent pages, with a special focus on the idea of \enquote{passions} \autocite[581 -- 585]{sartre} encroaching or affecting our freedom. He rejects passions as an external force upon our freedom -- because any admission of a passion being beyond freedom will make our freedom deterministic to it: \textcquote[581]{sartre}{This discussion shows that only two solutions, and only two are possible: either man is entirely determined \ldots\ or, indeed, man is entirely free.}

Likewise, with this rejection of passions having any special hold over our freedom, Sartre also rejects reason as having any special hold over our freedom. Sartre makes the distinction between reasons and motives, but ultimately both are equally non-essential to our freedom. In fact, Sartre states that \textcquote[585]{sartre}{In relation to freedom, no psychological phenomenon [either passion nor reason] is favoured. All my \enquote{ways of being} manifest it equally, since they are all ways of being my own nothingness.}

\begin{enumerate}
  \item \textcquote[591]{sartre}{In fact, reasons and motives only have the weight that my project -- i.e., the free production of the end, and of the act as having to be actualised -- confers on them. When I deliberate, the die is already cast. And if I must come to deliberate, it is simply because it is a part of my original project to take account of my motives \emph{by the means of deliberation} rather than through this or that other means of discovery.}
  \item Essentially, reasons and motives only bind our action, insofar as we ascribe their importance to them (roughly speaking).
\end{enumerate}

\noindent
Once again, freedom comes from the core of our being. Sartre illustrates this quite well using an image involving a hike with friends at \autocite[595]{sartre} that deserves revisiting:

\begin{enumerate}
  \item \textcquote[596]{sartre}{[When giving up to fatigue] is not a contemplative apprehension of my fatigue [i.e. \enquote{thought}]; rather -- as we saw in relation to pain -- I suffer my fatigue [i.e. a relation of being].}
  \item Essentially, Sartre is telling us that we \emph{act} in a certain way, because we \emph{be} in a certain way. This line of reasoning is much clearer on an examination of \autocite[596 -- 600]{sartre}.
  \item He presents another example of this in effect, by looking at a person with an \enquote{inferiority complex} which manifests in certain ways:
  \item \textcquote[601]{sartre}{This inferiority, which I struggle against and yet recognise, was \emph{chosen} by me at the outset \ldots\ to give in to fatigue, for example, is to transcend the path still to be covered by constituting it with the meaning \enquote{the path that is too difficult to follow.}}
  \item \textcquote[602]{sartre}{Thus the inferiority complex is a free and global project of myself, as inferior next to another; it is the way in which I choose to take on my being-for-the-Other, the free solution that I find for the insurmountable scandal of the other's existence.}
\end{enumerate}

\noindent
On how we are free to change our being:

\begin{enumerate}
  \item \textcquote[607]{sartre}{[all phenomena] which is in the end of the world which I am constantly conscious -- at least as the meaning implied by the object that I am looking at or using -- everything teaches me, myself, about my choice, i.e. about my [choice of] being.}
  \item \textcquote[607]{sartre}{Earlier we raised a question: I gave in to fatigue [in the example of the hike with friends], we said, and probably \emph{I could have done} otherwise, but at \emph{what cost}?} Or in other words, how are we free to change our choice, if our choice comes from our being?
  \begin{enumerate}
    \item \textcquote[607]{sartre}{We are now in a position to answer it Our analysis has just shown us, in effect that this act was not \emph{gratuitous.} Of course, it could not be explained by a motive or reason conceived as the content of an earlier \enquote{state} of consciousness; but it needed \emph{to be interpreted on the basis of an original project of which it formed an integral part}.}
    \item \textcquote[607]{sartre}{In consequence it becomes clear that we cannot suppose the action could have been modified without at the same time supposing \emph{a fundamental modification in my original choice of myself.}}
    \item \textcquote[607 -- 608]{sartre}{[Hence] I can refuse to stop only though a \emph{radical conversion of my being-in-the-world}, which is to say by a sudden metamorphosis of my initial project, which is to say by a different choice of myself and my ends.}
    \item \textcquote[608]{sartre}{Moreover, this modification is always possible. The anguish which, when it is disclosed, manifests our freedom to our consciousness testifies to this constant alterability of our initial project. In anguish, we do not simply grasp the fact that the possibles we are projecting are constantly eaten into by the freedom still to come; in addition, we apprehend our choice -- which is to say ourselves -- as being \emph{unjustifiable} [i.e. we can always change our being].}
  \end{enumerate}
  \item \textcquote[608]{sartre}{In this way we are constantly engaged in our choice, and constantly conscious of the fact that we ourselves can suddenly reverse this choice and change course, because we project the future through our very being, and we constantly eat away at it through our own existential freedom, declaring to ourselves by the means of the future what we are, and lacking any grip on this future, which remains always \emph{possible} without ever passing into the ranks of the \emph{real} Thus we are constantly \emph{threatened} with the nihilation of our current choice, constantly threatened with choosing ourselves -- and in consequence with becoming -- other than we are. Just because our choice is absolute, it is \emph{fragile}, which is to say that, by positing our freedom through it, we posit at the same time the constant possibility of becoming something that is \enquote{on this side} and pastified, in relation to an \enquote{over on that side} that I will be.}
\end{enumerate}

\noindent
For the remainder of the section, Sartre examines the various nuances and manifestations of freedom and choice being an element of our ontology. There's a particularly interesting section on bad faith and choice in \autocite[620]{sartre}, and he finally reiterates and summarises the  conclusions of this chapter in an eight-point summary at \autocite[622 -- 628]{sartre}. Having completed his elucidation on freedom, we move on to the next section.

\subsection{Freedom and Facticity: the Situation}

This section of the chapter is both essential as it is illuminating. After coming to an understanding that our freedom is an essential part of our ontology, it is easy for one to throw one's hands up and say: \enquote{But in practice, how can we be free if we have homework/debt/obligations?} To quote sartre's own introduction of this chapter: \textcquote[629]{sartre}{The decisive argument brought by \enquote{good sense} against freedom is a reminder of our powerlessness. Far from being able to change our situation at will, it seems that we are unable to change ourselves. I am not \enquote{free} to escape the destiny of my class, my nation, my family, or even to build up my power or my wealth, or to overcome my most trivial appetites or habits.} This section is entirely dedicated to understanding the relationship between our freedom, which is ontological -- and our facticity, which is a sole in-itself of the world.

Sartre dedicates this section towards addressing five broad manifestations of facticity that our freedom encounters. These are \emph{My Place, My Past, My Surroundings, My Fellow Man,} and \emph{My Death}. Note the significance of using the personal possessive pronoun \enquote{my} -- these are fundamentally \enquote{my} facticities.

\begin{enumerate}
  \item \textcquote[629]{sartre}{In particular, the coefficient of adversity of things cannot be an argument against our freedom, because it is \emph{through us}, which is to say by means of an end that we have posited beforehand, that this coefficient of adversity arises.}
  \item \textcquote[630]{sartre}{Thus, although brute things may limit our freedom of action from the outset, it is our freedom itself which must previously have constituted the framework, the technique, and the ends, in relation to which these things will show themselves to be limits \ldots\ it is our freedom therefore which constitutes the limits it will thereafter encounter.}
  \item \textcquote[630]{sartre}{Of course, after these remarks, an unnameable and unthinkable \emph{residuum} remains, which belongs to the in-itself in question \ldots\ But this \emph{residue} is far from being an original limit to freedom; rather, it is thanks to it -- i.e., thanks to the brute in-itself, as such -- that our freedom can arise as freedom.}
  \begin{enumerate}
    \item \textcquote[630]{sartre}{If it is sufficient to conceive of something for it to be actualised, I will find myself suddenly plunged into a world resembling the dream-world, where the possible is no longer in any way distinct from the real. I am condemned, then, to see the world changing in accordance with the changes \emph{of} my consciousness.}
    \item \textcquote[630]{sartre}{With the abolition of the distinction between a mere \emph{wish}, a \emph{representation} that I might be able to choose, and the \emph{choice}, freedom will disappear too.}
  \end{enumerate}
  \item \textcquote[631]{sartre}{There can only be a free for-itself if it is committed within a resisting world. Outside this commitment, the notions of freedom, determinism, and necessity lose all of their meaning.}
\end{enumerate}

\noindent
The key realisation here is that \emph{we need facticity in order to have freedom.} The material opposition which facticity offers us is like the friction of a tarmac which allows a car to move in the first place. A car upon a frictionless surface is like a being-for-itself in a dream world -- there cannot be any motion, nor freedom at all without facticity. 

\begin{enumerate}
  \item \textcquote[634]{sartre}{Thus freedom is a lesser being, which presupposes being, in order to subtract from it. It is free neither to not exist nor not to be free. We can grasp the connection between these two structures immediately; in effect, because freedom is an escape from being, it cannot occur \emph{besides} being \ldots\ \emph{one cannot escape from a jail in which one is not locked up/}}
  \item \textcquote[635]{sartre}{To exist -- as the \emph{fact} of freedom -- or to have, in the midst of the world, a being to be is one and the same thing, which means that freedom is from the outset a \emph{relation to the given}.}
  \item \textcquote[637]{sartre}{The given in itself in the form of \emph{resistance} or \emph{aide} is revealed only in the light of the pro-jecting freedom. \ldots\ Therefore it is only in and through freedom's free arising that the world can develop and reveal the resistances that may make the projected end impossible to achieve. A man can encounter an obstacle only within the field of his freedom.}
\end{enumerate}

\noindent
In the remainder of this chapter, Sartre explores the five different forms of facticity that our freedom encounters.

\subsubsection{My Place}

Pages \autocite[639 -- 646]{sartre}. This is the facticity of my geography -- but more generally, the immediate world of the in-itselfs that surrounds us.

\begin{enumerate}
  \item To be completed later.
\end{enumerate}

\subsubsection{My Past}

Pages \autocite[646 -- 657]{sartre}. This is the facticity of my past -- which while does not determine our actions, still forms as an inescapable part of our being. We cannot not have a past -- even to deny one's past is to be in a certain relationship with the past. Generally the meaning of the past changes depending on the project of our present \autocite[649]{sartre}.

\begin{enumerate}
  \item To be completed later.
\end{enumerate}

\subsubsection{My Surroundings}

Pages \autocite[657 -- 663]{sartre}. Our surroundings are not the same as our place -- but rather, it is a specific term for the \enquote{implement-things that surround me, with their own coefficient of adversity and their equipmentality}. It's sort of like the solution space of our surrounding which lets us evaluate how easy our project is.

\begin{enumerate}
  \item To be completed later.
\end{enumerate}

\subsubsection{My Fellow Man}

Pages \autocite[663 -- 689]{sartre}. \marginnote{I should make sure to revisit this section, as it is the most interesting out of the series.} This is one of the more interesting sections in this series. My Fellow Man is a distinct category of facticity that we must examine, because we live in a world that is filled with the Other. Specifically, the \emph{meaning} of an object (i.e. its equipmentality) is not always determined by ourselves, by often by other people. This section explores our relationship with objects that already have a meaning -- i.e. the imposition of another's freedom upon our own. Out of all the sections, it is this section that deserves a closer re-visit. In particular, there is an interesting examination on luxury.

\begin{enumerate}
  \item To be completed later.
\end{enumerate}

\subsubsection{My Death}

Pages \autocite[689 -- 719]{sartre}. Death is the ultimate absence of being, and what Sartre calls an \emph{absurdity}. It is also the section that I least understand. I think at least one of the more important philosophical takeaways from this section is Sartre's insistence that death itself has no meaning, and cannot have any meaning: \textcquote[710]{sartre}{Thus death is in no way an ontological structure of my being, at least not insofar as it is \emph{for-itself}; it is \emph{the Other} who is mortal in his being. There is no place for death in being-for-itself; it can neither await it nor actualise it, nor project itself towards it. It is in no way the foundation of its finitude.}

\begin{enumerate}
  \item To be completed later.
\end{enumerate}

\subsection{Freedom and Responsibility}

Pages \autocite[719 -- 722]{sartre}. In this last, but very short section on freedom and responsibility, Sartre presents the double-edged sword of this absolute, ontological freedom. Just as we are always free, we are likewise always responsible. We cannot dodge responsibility, just as we can't dodge our freedom. We maintain the ultimate authorship of all our actions through our very being. \textcquote[718]{sartre}{The essential consequence of our previous remarks is that man, being condemned to be free, carries the weight of the whole world on his shoulders: he is responsible for the world and for himself, as a way of being.} 

To end this chapter on a concluding thought from Sartre: \marginnote{These are very powerful quotes, which gives our life a lot of brilliance. I should definitely revisit them in time.} \textcquote[718]{sartre}{In this sense, the for-itself's responsibility is overwhelming, since he is the one who makes it the case that \emph{there is} a world, and since it is he, too, who \emph{makes himself be;} whatever the situation in which he finds himself, the for-itself has to accept this situation in its entirety, with its own coefficient of adversity, even if it is unsustainable. He has to accept it with the proud consciousness of being its author because the worst disadvantages, or the worst threats from which my person is in danger, can have meaning only through my project, and they appear against the ground of the commitment that I am.}

\section{Chapter 2: To Do and To Have}

\subsection{Existential Psychoanalysis}

\begin{enumerate}
  \item To be completed later.
\end{enumerate}

\subsection{To Do and To Have: Possession}

\begin{enumerate}
  \item To be completed later.
\end{enumerate}

\subsection{The Revelation of Being Through Qualities}

\begin{enumerate}
  \item To be completed later.
\end{enumerate}

\section{Conclusion}

\subsection{In-Itself and For-Itself: Some Metaphysical Observations}

\begin{enumerate}
  \item To be completed later.
\end{enumerate}

\subsection{Moral Perspectives}

\begin{enumerate}
  \item To be completed later.
\end{enumerate}
