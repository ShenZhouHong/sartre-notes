\chapter{Part IV: To Have, To Do, and To Be}

So what is \textsc{Part IV} about? In the introduction and the three parts, we've explored the ontology of the in-itself (i.e. the being-of-phenomena), the ontology of the for-itself, and finally the ontology of the Other -- as well as all their interleaving relationships such as the relationship of the for-itself to the in-itself (knowledge) or the relationship of the for-itself to the Other (the being-for-the-Other). With the above ontology established, we are finally able to explore the nature of the being-for-itself as an agent that \emph{acts}. It is this exploration of action that serves as the theme of Part IV. This part serves as the final third of Sartre's monograph on Phenomenological Ontology: \emph{Being and Nothingness}. Sartre introduces this part to us with the following opening-question:

\textcquote[567]{sartre}{Having, doing, and being are the fundamental categories of human reality. Every type of human behaviour can be subsumed within them. \ldots\ Is the supreme value of the human action \emph{to do} or \emph{to be}?}

\section{Chapter 1: Being and Doing: Freedom}

In this chapter, we are interested in the ontology of freedom. To begin this investigation, we first look at \emph{actions} -- which are things that our being partakes (i.e. our being \emph{acts}). This first chapter of \textsc{Part IV} is actually the much longer (and in my opinion, more important) chapter of the two chapters in this Part.

\subsection{The First Condition of Action is Freedom}

From where does an action derive its being? Sartre rejects the naively rationalist perspective that actions simply emerge from a series of contingent reasons (even erroneous reasons). Rather, he claims that no material contingency can bring about, or generate the impetus for action -- this is because any material contingency will simply be an objective description of the world in its facticity -- there will be no \emph{lack} that yields the space for action \autocite[574]{sartre}. Instead, action has to come from a \emph{nothingness} -- and that nothingness, which is the same nothingness that founds our lack and desire -- is the source of our freedom. Hence, in order to understand action, we must understand freedom -- the first condition of action.

\begin{enumerate}
  \item \textcquote[569]{sartre}{The point we should note at the outset is that an action is, by definition, \emph{intentional}.}
  \item \textcquote[570]{sartre}{An action necessarily implies, as its condition, some recognised \enquote{desideratum} i.e., an objective lack or even a negatity.}
  \item \textcquote[575]{sartre}{It is the act that determines its end and its motives, and the act is the expression of freedom.}
  \item \textbf{Relationship between Freedom, Unfreedom, For-itself, and In-Itself:}
  \begin{enumerate}
    \item \textcquote[577]{sartre}{The innermost meaning of determinism is to establish within us an unfailing continuity of existence in itself \ldots\ Thus, the rejection of freedom can be conceived only as an attempt to apprehend oneself as a being-in-itself}
    \item \textcquote[578]{sartre}{Freedom coincides with the nothingness that lies at man's heart. It is because human-reality \emph{is not enough} that it is free, because it is constantly separated from itself, and because a nothingness that separates what it has been from what is, and from what it will be.}
    \item \textcquote[578]{sartre}{Man is free because he is not an [in-] itself but self-presence. \emph{A being that is what it is cannot be free.}}
  \end{enumerate}
  \item \textcquote[579]{sartre}{Human-reality is entirely abandoned, without any help of any kind, to the unbearable necessity of making itself be, right down to the last detail: In this way freedom is not \emph{a} being: it is man's being, i.e. his nothingness of being.}
  \item \textcquote[579]{sartre}{Man cannot be sometimes free and sometimes a slave: he is free in his entirety and always, or he is not.}
\end{enumerate}

What is this understanding of freedom? Freedom comes from nothingness -- not just any nothingness, but the very nothingness that lies inside our being-for-itself, like a worm in its heart. This is a \emph{very} strong claim for Sartre to make, because all of a sudden freedom is not a contingent, material property of a human -- but rather, our freedom is founded on our very being, on our ontology. Human beings are not born free, only to lose it later on. Human beings -- so far as we can \emph{be} in any case -- have the ontology of freedom. As this freedom comes from the same nothingness which serves as a foundation of our being as a being-for-itself, it means that consciousness implies freedom. Anything else would be objectivity (a being-in-itself), and hence absolute determinism.

Sartre goes on to defend and elaborate on this audacious claim in the subsequent pages, with a special focus on the idea of \enquote{passions} \autocite[581 -- 585]{sartre} encroaching or affecting our freedom. He rejects passions as an external force upon our freedom -- because any admission of a passion being beyond freedom will make our freedom deterministic to it: \textcquote[581]{sartre}{This discussion shows that only two solutions, and only two are possible: either man is entirely determined \ldots\ or, indeed, man is entirely free.}

Likewise, with this rejection of passions having any special hold over our freedom, Sartre also rejects reason as having any special hold over our freedom. Sartre makes the distinction between reasons and motives, but ultimately both are equally non-essential to our freedom. In fact, Sartre states that \textcquote[585]{sartre}{In relation to freedom, no psychological phenomenon [either passion nor reason] is favoured. All my \enquote{ways of being} manifest it equally, since they are all ways of being my own nothingness.}

\begin{enumerate}
  \item \textcquote[591]{sartre}{In fact, reasons and motives only have the weight that my project -- i.e., the free production of the end, and of the act as having to be actualised -- confers on them. When I deliberate, the die is already cast. And if I must come to deliberate, it is simply because it is a part of my original project to take account of my motives \emph{by the means of deliberation} rather than through this or that other means of discovery.}
  \item Essentially, reasons and motives only bind our action, insofar as we ascribe their importance to them (roughly speaking).
\end{enumerate}

\noindent
Once again, freedom comes from the core of our being. Sartre illustrates this quite well using an image involving a hike with friends at \autocite[595]{sartre} that deserves revisiting:

\begin{enumerate}
  \item \textcquote[596]{sartre}{[When giving up to fatigue] is not a contemplative apprehension of my fatigue [i.e. \enquote{thought}]; rather -- as we saw in relation to pain -- I suffer my fatigue [i.e. a relation of being].}
  \item Essentially, Sartre is telling us that we \emph{act} in a certain way, because we \emph{be} in a certain way. This line of reasoning is much clearer on an examination of \autocite[596 -- 600]{sartre}.
  \item He presents another example of this in effect, by looking at a person with an \enquote{inferiority complex} which manifests in certain ways:
  \item \textcquote[601]{sartre}{This inferiority, which I struggle against and yet recognise, was \emph{chosen} by me at the outset \ldots\ to give in to fatigue, for example, is to transcend the path still to be covered by constituting it with the meaning \enquote{the path that is too difficult to follow.}}
  \item \textcquote[602]{sartre}{Thus the inferiority complex is a free and global project of myself, as inferior next to another; it is the way in which I choose to take on my being-for-the-Other, the free solution that I find for the insurmountable scandal of the other's existence.}
\end{enumerate}

\noindent
On how we are free to change our being:

\begin{enumerate}
  \item \textcquote[607]{sartre}{[all phenomena] which is in the end of the world which I am constantly conscious -- at least as the meaning implied by the object that I am looking at or using -- everything teaches me, myself, about my choice, i.e. about my [choice of] being.}
  \item \textcquote[607]{sartre}{Earlier we raised a question: I gave in to fatigue [in the example of the hike with friends], we said, and probably \emph{I could have done} otherwise, but at \emph{what cost}?} Or in other words, how are we free to change our choice, if our choice comes from our being?
  \begin{enumerate}
    \item \textcquote[607]{sartre}{We are now in a position to answer it Our analysis has just shown us, in effect that this act was not \emph{gratuitous.} Of course, it could not be explained by a motive or reason conceived as the content of an earlier \enquote{state} of consciousness; but it needed \emph{to be interpreted on the basis of an original project of which it formed an integral part}.}
    \item \textcquote[607]{sartre}{In consequence it becomes clear that we cannot suppose the action could have been modified without at the same time supposing \emph{a fundamental modification in my original choice of myself.}}
    \item \textcquote[607 -- 608]{sartre}{[Hence] I can refuse to stop only though a \emph{radical conversion of my being-in-the-world}, which is to say by a sudden metamorphosis of my initial project, which is to say by a different choice of myself and my ends.}
    \item \textcquote[608]{sartre}{Moreover, this modification is always possible. The anguish which, when it is disclosed, manifests our freedom to our consciousness testifies to this constant alterability of our initial project. In anguish, we do not simply grasp the fact that the possibles we are projecting are constantly eaten into by the freedom still to come; in addition, we apprehend our choice -- which is to say ourselves -- as being \emph{unjustifiable} [i.e. we can always change our being].}
  \end{enumerate}
  \item \textcquote[608]{sartre}{In this way we are constantly engaged in our choice, and constantly conscious of the fact that we ourselves can suddenly reverse this choice and change course, because we project the future through our very being, and we constantly eat away at it through our own existential freedom, declaring to ourselves by the means of the future what we are, and lacking any grip on this future, which remains always \emph{possible} without ever passing into the ranks of the \emph{real} Thus we are constantly \emph{threatened} with the nihilation of our current choice, constantly threatened with choosing ourselves -- and in consequence with becoming -- other than we are. Just because our choice is absolute, it is \emph{fragile}, which is to say that, by positing our freedom through it, we posit at the same time the constant possibility of becoming something that is \enquote{on this side} and pastified, in relation to an \enquote{over on that side} that I will be.}
\end{enumerate}

\noindent
For the remainder of the section, Sartre examines the various nuances and manifestations of freedom and choice being an element of our ontology. There's a particularly interesting section on bad faith and choice in \autocite[620]{sartre}, and he finally reiterates and summarises the  conclusions of this chapter in an eight-point summary at \autocite[622 -- 628]{sartre}. Having completed his elucidation on freedom, we move on to the next section.

\subsection{Freedom and Facticity: the Situation}

This section of the chapter is both essential as it is illuminating. After coming to an understanding that our freedom is an essential part of our ontology, it is easy for one to throw one's hands up and say: \enquote{But in practice, how can we be free if we have homework/debt/obligations?} To quote sartre's own introduction of this chapter: \textcquote[629]{sartre}{The decisive argument brought by \enquote{good sense} against freedom is a reminder of our powerlessness. Far from being able to change our situation at will, it seems that we are unable to change ourselves. I am not \enquote{free} to escape the destiny of my class, my nation, my family, or even to build up my power or my wealth, or to overcome my most trivial appetites or habits.} This section is entirely dedicated to understanding the relationship between our freedom, which is ontological -- and our facticity, which is a sole in-itself of the world.

Sartre dedicates this section towards addressing five broad manifestations of facticity that our freedom encounters. These are \emph{My Place, My Past, My Surroundings, My Fellow Man,} and \emph{My Death}. Note the significance of using the personal possessive pronoun \enquote{my} -- these are fundamentally \enquote{my} facticities.

\begin{enumerate}
  \item \textcquote[629]{sartre}{In particular, the coefficient of adversity of things cannot be an argument against our freedom, because it is \emph{through us}, which is to say by means of an end that we have posited beforehand, that this coefficient of adversity arises.}
  \item \textcquote[630]{sartre}{Thus, although brute things may limit our freedom of action from the outset, it is our freedom itself which must previously have constituted the framework, the technique, and the ends, in relation to which these things will show themselves to be limits \ldots\ it is our freedom therefore which constitutes the limits it will thereafter encounter.}
  \item \textcquote[630]{sartre}{Of course, after these remarks, an unnameable and unthinkable \emph{residuum} remains, which belongs to the in-itself in question \ldots\ But this \emph{residue} is far from being an original limit to freedom; rather, it is thanks to it -- i.e., thanks to the brute in-itself, as such -- that our freedom can arise as freedom.}
  \begin{enumerate}
    \item \textcquote[630]{sartre}{If it is sufficient to conceive of something for it to be actualised, I will find myself suddenly plunged into a world resembling the dream-world, where the possible is no longer in any way distinct from the real. I am condemned, then, to see the world changing in accordance with the changes \emph{of} my consciousness.}
    \item \textcquote[630]{sartre}{With the abolition of the distinction between a mere \emph{wish}, a \emph{representation} that I might be able to choose, and the \emph{choice}, freedom will disappear too.}
  \end{enumerate}
  \item \textcquote[631]{sartre}{There can only be a free for-itself if it is committed within a resisting world. Outside this commitment, the notions of freedom, determinism, and necessity lose all of their meaning.}
\end{enumerate}

\noindent
The key realisation here is that \emph{we need facticity in order to have freedom.} The material opposition which facticity offers us is like the friction of a tarmac which allows a car to move in the first place. A car upon a frictionless surface is like a being-for-itself in a dream world -- there cannot be any motion, nor freedom at all without facticity. 

\begin{enumerate}
  \item \textcquote[634]{sartre}{Thus freedom is a lesser being, which presupposes being, in order to subtract from it. It is free neither to not exist nor not to be free. We can grasp the connection between these two structures immediately; in effect, because freedom is an escape from being, it cannot occur \emph{besides} being \ldots\ \emph{one cannot escape from a jail in which one is not locked up/}}
  \item \textcquote[635]{sartre}{To exist -- as the \emph{fact} of freedom -- or to have, in the midst of the world, a being to be is one and the same thing, which means that freedom is from the outset a \emph{relation to the given}.}
  \item \textcquote[637]{sartre}{The given in itself in the form of \emph{resistance} or \emph{aide} is revealed only in the light of the pro-jecting freedom. \ldots\ Therefore it is only in and through freedom's free arising that the world can develop and reveal the resistances that may make the projected end impossible to achieve. A man can encounter an obstacle only within the field of his freedom.}
\end{enumerate}

\noindent
In the remainder of this chapter, Sartre explores the five different forms of facticity that our freedom encounters.

\subsubsection{My Place}

Pages \autocite[639 -- 646]{sartre}. This is the facticity of my geography -- but more generally, the immediate world of the in-itselfs that surrounds us.

\begin{enumerate}
  \item To be completed later.
\end{enumerate}

\subsubsection{My Past}

Pages \autocite[646 -- 657]{sartre}. This is the facticity of my past -- which while does not determine our actions, still forms as an inescapable part of our being. We cannot not have a past -- even to deny one's past is to be in a certain relationship with the past. Generally the meaning of the past changes depending on the project of our present \autocite[649]{sartre}.

\begin{enumerate}
  \item To be completed later.
\end{enumerate}

\subsubsection{My Surroundings}

Pages \autocite[657 -- 663]{sartre}. Our surroundings are not the same as our place -- but rather, it is a specific term for the \enquote{implement-things that surround me, with their own coefficient of adversity and their equipmentality}. It's sort of like the solution space of our surrounding which lets us evaluate how easy our project is.

\begin{enumerate}
  \item To be completed later.
\end{enumerate}

\subsubsection{My Fellow Man}

Pages \autocite[663 -- 689]{sartre}. \marginnote{I should make sure to revisit this section, as it is the most interesting out of the series.} This is one of the more interesting sections in this series. My Fellow Man is a distinct category of facticity that we must examine, because we live in a world that is filled with the Other. Specifically, the \emph{meaning} of an object (i.e. its equipmentality) is not always determined by ourselves, by often by other people. This section explores our relationship with objects that already have a meaning -- i.e. the imposition of another's freedom upon our own. Out of all the sections, it is this section that deserves a closer re-visit.

\begin{enumerate}
  \item To be completed later.
\end{enumerate}

\subsubsection{My Death}

Pages \autocite[689 -- 719]{sartre}. Death is the ultimate absence of being, and what Sartre calls an \emph{absurdity}. It is also the section that I least understand. I think at least one of the more important philosophical takeaways from this section is Sartre's insistence that death itself has no meaning, and cannot have any meaning: \textcquote[710]{sartre}{Thus death is in no way an ontological structure of my being, at least not insofar as it is \emph{for-itself}; it is \emph{the Other} who is mortal in his being. There is no place for death in being-for-itself; it can neither await it nor actualise it, nor project itself towards it. It is in no way the foundation of its finitude.}

\begin{enumerate}
  \item To be completed later.
\end{enumerate}

\subsection{Freedom and Responsibility}

Pages \autocite[719 -- 722]{sartre}. In this last, but very short section on freedom and responsibility, Sartre presents the double-edged sword of this absolute, ontological freedom. Just as we are always free, we are likewise always responsible. We cannot dodge responsibility, just as we can't dodge our freedom. We maintain the ultimate authorship of all our actions through our very being. \textcquote[718]{sartre}{The essential consequence of our previous remarks is that man, being condemned to be free, carries the weight of the whole world on his shoulders: he is responsible for the world and for himself, as a way of being.} 

To end this chapter on a concluding thought from Sartre: \marginnote{These are very powerful quotes, which gives our life a lot of brilliance. I should definitely revisit them in time.} \textcquote[718]{sartre}{In this sense, the for-itself's responsibility is overwhelming, since he is the one who makes it the case that \emph{there is} a world, and since it is he, too, who \emph{makes himself be;} whatever the situation in which he finds himself, the for-itself has to accept this situation in its entirety, with its own coefficient of adversity, even if it is unsustainable. He has to accept it with the proud consciousness of being its author because the worst disadvantages, or the worst threats from which my person is in danger, can have meaning only through my project, and they appear against the ground of the commitment that I am.}

\section{Chapter 2: To Do and To Have}

There are two main parts in this chapter -- the first part is the first section entitled \emph{Existential Psychoanalysis}, and the second part are the next two sections (which are respectively To Do and To Have: Possession, and the Revelation of Being Through Qualities). Of these two sections, the most interesting is the first one on Existential Psychoanalysis -- and it is for a specific reason. In Existential Psychoanalysis, Sartre first presents the \emph{aim or object} of existential psychoanalysis -- which has a pretty strong effect on our understanding of a human being's ontology. The aim of existential psychoanalysis (i.e. the object hidden in our being which this psychoanalysis aims to uncover and reveal) is the fundamental \emph{project} of our being. Recall how our freedom comes from our being -- any of the more material and superficial projects that we pursue in life ultimately draw their meaning from our \emph{being}. What is our being? Our being is the question of its own being -- hence our being is this continued flowing into a being that we wish to become. What is that being? That is the aim of existential psychoanalysis.

\subsection{Existential Psychoanalysis}

Pages \autocite[723 -- 746]{sartre}. In the first ten pages \autocite[723 -- 732]{sartre} of this section, Sartre begins with a more mundane preliminary on what \emph{existential psychoanalysis} is as a field, and he compares it with regular Freudian psychoanalysis. This is just scene-setting, and not quite interesting on its own. It is after setting out the preliminaries do we move on to the ontologically interesting questions on the ultimate aim of our being. \marginnote{On the teleology of human apotheosis.}

\begin{enumerate}
  \item \textcquote[732]{sartre}{[Any superficial projects] as the totality of my being, expresses my original choice [of being] in some particular circumstances; it is nothing but the choice of myself \emph{as a totality} in these circumstances [which manifests as the project].}
  \item \textcquote[732]{sartre}{It is therefore by means of a \emph{comparison} of a subject's various empirical tendencies that we may attempt to discover and to isolate the fundamental project that they have in common}
  \item \textcquote[733]{sartre}{Man is fundamentally the \emph{desire to be}, and the existence of this desire is not to be established through empirical induction; it comes out of an \emph{a priori} description of the for-itself's being, since desire is a lack and the for-itself is the being that is, in relation to itself, its own lack of being.}
  \item \textcquote[733]{sartre}{The original project that is expressed in each of our empirically observable tendencies is therefore the \emph{project of being.}}
  \item \textcquote[734]{sartre}{As for the being that is the object of this desire, we know \emph{a priori} what it is. The for-itself is the being that is, in relation to itself, its own lack of being. And the being that is missing from the for-itself is the in-itself.}
  \begin{enumerate}
    \item \textcquote[734]{sartre}{The for-itself arises as the nihilation of the in-itself, and this nihilation is defined as a pro-ject toward the in-itself: between the nihilated in-itself and the projected in-itself, the for-itself is a nothingness.}
    \item \textcquote[734]{sartre}{Thus the goal and the end of the nihilation that I am is the \emph{in-itself}. Thus, human-reality is the desire to-be-in-itself.}
    \item \textcquote[734]{sartre}{But the in-itself that it desires cannot be pure in-itself, contingent and absurd, and comparable in every respect to the in-itself that it encounters and nihilates.}
    \item \textcquote[734]{sartre}{In fact, as we have seen, we can understand nihilation in terms of a rebellion on the part of the in-itself, which nihilates itself in opposition to its contingency.}
    \item \textcquote[734]{sartre}{\ldots\ nihilation [itself] is a being's futile endeavour to found its own being and that its withdrawal, in order to found it, is the source of the infinitesimal gap through which nothingness enters being.}
  \end{enumerate}
  \item \textcquote[734]{sartre}{The being that is the object of the for-itself's desire is, therefore, \emph{an in-itself that might relate to itself as its own foundation, i.e.,} an in-itself \emph{whose relation to its facticity would be like the for-itself's relation to its motivations.}}
\end{enumerate}

It's that very last sentence here that presents a very big and important idea, so we'll take care to unpack it here. Every human pursuit (or project, in Sartre's terminology) -- is an action taken through freedom, the freedom that comes from our being. But what is our being? Our being is the nihilation of the in-itself, which yields the for-itself. The for-itself is a \emph{lack} -- so what does our for-itself want \emph{to be}? Or in Sartre's words: \enquote{What is the object of the for-itself's desire?} What our for-itself ultimately lacks is the in-itself -- the in-itself which we continuously nihilate in order to derive our being as a for-itself. 

However, the ultimate pursuit of our being cannot be a simple in-itself. We do not desire to become unconscious Rather, the being that we desire is ultimately an in-itself which will allow us to e our own foundation -- without needing to nihilate it. Recall how our being is always \emph{endangered} and at risk -- we are endangered by contingency, by the gaze of the Other. All of these \emph{ontological dangers} come from how our being is not self-supporting -- we are not our own foundation. Hence, in Sartre's words -- the being that we seek, is \enquote{an in-itself whose relation to its facticity would be like the for-itself's relation to its motivations.} Just like how our motivations completely come from our freedom (which completely comes from our being) -- we want an in-itself those \emph{facticity} completely comes from our freedom -- and is not an arbitrary child of contingency:

\begin{enumerate}
  \item \textcquote[735]{sartre}{The for-itself's project is \emph{to be as for-itself} a being that is what it is:}
  \begin{enumerate}
    \item \enquote{[The for-itself right now] is as a being that \emph{is} what it is not [i.e. the nihilation]}
    \item \enquote{and that [contingent existence] is not what it is}
    \item \enquote{that the for-itself has the project to be what it is [the better, self-standing being]}
  \end{enumerate}
  \item \enquote{[The for-itself] is as a consciousness that it wants to have the in-itself's impermeability and infinite density}
  \item \textcquote[735]{sartre}{That is why the possible is generally pro-jected as what is missing from the for-itself in order for it to become in-itself-for-itself; and the fundamental value presiding over this project is precisely, the in-itself for itself, i.e. the ideal of a consciousness that could be the foundation of its own being-in-itself \emph{purely by the means of its own being conscious of itself}}
  \item \textcquote[735]{sartre}{To this ideal, we can give the name \enquote{God.} So we can say that the best way to conceive of the human-reality's fundamental project is to regard man as the being whose project is to be God}
  \item \textcquote[735]{sartre}{To be a man is to aim to be God; or, alternatively, man is fundamentally the desire to be God}
\end{enumerate}

This audacious conclusion, of the ultimate aim of man's being to be God -- serves as the ontological conclusion of this section. Following this, Sartre presents the \emph{existential psychoanalysis} which serves as a practical toolkit for finding a man's ultimate project of being, from the beginning point of his external projects. \textcquote[739]{sartre}{Existential psychoanalysis seeks to determine the \emph{original choice} [of being].}

\noindent
More specifically: \textcquote[739]{sartre}{This original choice, which is made in the face of the world and is the choice of a position within the world, is totalising. Like the complex, this original choice is prior to logic; it \emph{chooses} the person's attitude in relation to logic and principles.} This section concludes with a further discussion on the method, the practical application, and the limitations of existential psychoanalysis as a practice \autocite[739 -- 746]{sartre}.

\subsection{To Do and To Have: Possession}
\marginnote{This is the section where Sartre presents an interesting examination on luxury as a type of possession. I should make sure to revisit it in further detail.}
Pages \autocite[746 -- 777]{sartre}. In this section, Sartre begins with the opening question: \textcquote[746]{sartre}{What therefore does ontology teach us about desire, insofar as desire is the being of human reality?} Remember how human-reality, as the reality of the being-for-itself, is characterised by the lack, which manifests as desire. Having explored desire implicitly in the previous section from a more abstract, ontological perspective (abstract as in, the object of our desire is to become God -- which is pretty impractical). Hence, it makes sense to revisit the object of desire -- or rather, the act of desire itself, to see how it manifests more practically.

\begin{enumerate}
  \item \textcquote[746]{sartre}{Desire is a lack in being, as we have seen. As such, it is immediately \emph{directed on} the being that it lacks This being, as we have seen, is [the] in-itself-for-itself, a consciousness that has become substance, a substance that has become its own cause, a Man-God.}\
  \item \textcquote[747]{sartre}{Human-reality is purely the endeavour to become God, an endeavour that is without any given substratum, in which there is \emph{nothing} to make this effort. Desire expresses this endeavour. Nonetheless, desire is not defined only in relation to the in-itself-that-is-its-own-cause. It [desire] \emph{is also relative to a brute and concrete existent, commonly known as the desire's object.}}
  \item \textcquote[747]{sartre}{In this way desire expresses in its very structure man's relation with one or several objects in the world; it is one of the aspects of the being-in-the-world.}
\end{enumerate}

Here, Sartre presents an interesting point. He makes the claim that all forms of desire can be reduced to an act of possession. Beginning with a cursory examination of the objects of various practical desires, Sartre condenses them to manifestations of three \enquote{great categories of human existence,} namely the categories of \emph{to do, to have,} and \emph{to be}, From these three categories, Sartre ultimately concludes that they are fundamental because they are all a certain modality of our relationship to the in-itself -- and that this fundamental relationship is ultimately the relationship of the \emph{to have}, i.e. \emph{possession}.

\begin{enumerate}
  \item \textcquote[748]{sartre}{It may seem to be gratuitous, as in the case of scientific research, sport, and aesthetic creation. Yet the \emph{doing}, in these various cases, is not irreducible either. If I create a painting, a play, a melody, it is in order to be at the origin of some particular existence.}
  \item \textcquote[748]{sartre}{It is not only that this painting, for which I have had the idea, should exist: it must, further, exist \emph{through me.}}
  \item \textcquote[748]{sartre}{[Any creation of mine] is also necessary that it should exist \emph{in itself} [as opposed to just being thought], i.e. that it should constantly renew its existence \emph{by itself}. Thereafter my work will appear as a creation that is continuous but frozen into the in-itself it forever bears my \enquote{mark}.}
  \item \textcquote[748]{sartre}{I am therefore in the twofold relationship with it [the object of my creation] of a consciousness that \emph{conceives} it and a consciousness that \emph{encounters} it. It is precisely this twofold relationship that I express when I say that it [the object] is \emph{mine}.}
  \item \textcquote[748]{sartre}{And it is in order to maintain this twofold relationship within the synthesis of appropriation that I create my work.}
\end{enumerate}

\subsubsection{Knowing as an act of (Sexual) Possession}

From this, it seems that the most fundamental (and ontologically pure and authentic) mode of possession is solely the possession of something that we have created ourselves. Because it is only through an act of pure, independent creation, do we maintain the twofold relationship of renewing our existence through an in-itself -- where the being of the in-itself owes its existence to the being of our for-itself. This act is transcendental -- it bears \emph{a lot} of resemblance to our (ultimately fruitless) attempts at supporting the \enquote{intolerable contingency} of our own existence through the Other. This is why possession in its purest form of creation is so satisfying: \textcquote[749]{sartre}{We find the unity of a single project, from the case of artistic creation to that of the cigarette which \enquote{is better when you have rolled it yourself.}} \marginnote{Is this why acts of pure creation (scientific, aesthetic) are somehow so satisfying to our being?} Sartre subsumes \emph{knowing} as the possession of knowledge within this dynamic. Specifically, he is keen to point out the ontological similarity between the possession of knowledge, and the possession of the sexual act:

\begin{enumerate}
  \item \textcquote[750]{sartre}{Within the very idea of discovery, of revelation, an idea of the appropriation involved in using something is included. Sight is an \emph{enjoyment}\footnote{According to the translator, the use of the French verb \emph{jouissance} has strong and unmistakable sexual connotations.} of something; to see something is to \emph{deflower} it.}
  \item \textcquote[750]{sartre}{All the images [of discovery] emphasise the object's ignorance of the investigations and instruments targeted at it: unaware that it is known, it goes about its business without noticing the spying eyes, like a woman surprised by a passer-by as she was washing.}
  \item \textcquote[750]{sartre}{Knowledge is a hunt. [Francis] Bacon calls it \enquote{Pan's Hunt.} The scholar is a hunter who surprises a naked white figure, and violates her through his gaze.}
\end{enumerate}

\noindent
The remainder of the section on knowledge \autocite[750 -- 752]{sartre} examines knowledge as a sexual, and even consumptive act in greater detail. It is uncomfortable in its Hellenistic depictions of knowing as a defloration -- but the section remains of ontological interest.

\subsubsection{On Play and Seriousness}

In pages \autocite[752 -- 760]{sartre}, Sartre takes us on a detour examining seriousness, and play. He uses an extended metaphor of snow and skiing -- the act of gliding -- as a way to examine the ontology of these concepts. Somehow playfulness is more closely related to our transcendental existence -- whereas seriousness is to root oneself within the mundane nature of the in-itself. This section takes up a surprisingly large amount of space, and deserves revisiting.

\begin{enumerate}
  \item To be completed later.
\end{enumerate}

\subsubsection{On Luxury and Capitalistic Possession}

Moving on in pages \autocite[764 -- 768]{sartre}, Sartre also embarks on an interesting examination on the possessions of luxuries, which he takes as a degenerate form of this more original, ontologically authentic possession of creation: \textcquote[749]{sartre}{In a short while we will return to this project in relation to a special type of property, akin to a degraded version of it [the ontologically purer act of creative possession], which goes by the name of \emph{luxury}}. In this section, he also segues into a discussion of possession through monetary means, which seems to be an implicit examination of possession under capitalism specifically -- the ontology of our relationship to objects that we acquire by the act of purchasing them. \marginnote{What connections and parallels can we draw from this existential conception of the ontology of money, with Karl Marx's own monetary metaphysics, that of \enquote{natural relations between commodities, and unnatural relationships between men?}}

\begin{enumerate}
  \item \textcquote[764]{sartre}{Luxury is a degraded version of [creative possession]; in luxury's primitive form, I possess an object that I have \emph{had made} for me, by people who are \emph{mine} (slaves, servants born in the house). Luxury is therefore the form of property that comes closest to primitive property [which comes solely from self-creation] and which throws the next best light, after it, on the relation of \emph{creation} by which appropriation is originally constituted.}
  \item \textcquote[765]{sartre}{Money becomes effaced by its \emph{buying} power: it is evanescent, made in order to disclose the object, the concrete thing; it has only a transitive being. But \emph{to me,} it appears as a creative force: to buy an object is a symbolic act equivalent to creating the object.}
  \item \textcquote[765]{sartre}{In this way money establishes a connection of appropriation between the for-itself and the total collection of objects in the world.}
\end{enumerate}

\subsubsection{On Possession through Destruction}

In pages \autocite[769 -- 771]{sartre}, Sartre takes us on an examination of \emph{possession through destruction}, which later on develops into a more general description of the consumptive aspects of possession. From my understanding, destruction is the not quite the opposite of creation, but more of a certain antimatter of creation -- namely whereas with creation I serve as the foundation of the object's being, with destruction I serve as the foundation of the object's non-being.

\begin{enumerate}
  \item \textcquote[769]{sartre}{It is precisely the for-itself's recognition of the impossibility of \emph{possessing} an object that brings in its wake a violent wish, on his part, to \emph{destroy} it. To destroy something is to absorb it into me, and to maintain a relation with the destroyed object's being-in-itself as profound as that of creation.}
  \item \textcquote[769]{sartre}{Here I rediscover the relation of being that characterises creation, but in reverse: I \emph{am} the foundation of the burning barn [the destroyed object]; I \emph{am} this barn, since I am destroying its being.}
  \item \textcquote[769]{sartre}{To destroy something is to re-create it by assuming sole responsibility for the being of that which had existed \emph{for everyone}.}
\end{enumerate}

\noindent
Sartre concludes this section with a summary at page \autocite[776]{sartre}.

\subsection{The Revelation of Being Through Qualities}

In pages \autocite[777 -- 798]{sartre}, Sartre wants to take a step to the side -- and look at being from a more orthogonal angle -- namely, by examining qualities. A quality is something that we attribute to being -- and Sartre quite rightfully calls this section the \emph{psychoanalysis of things}. This section begins a little obscurely, but Sartre latches on to the quality of \emph{viscousness} starting around \autocite[784]{sartre}. Viscousness seems to be a form of \emph{ontological horror} -- our being abhors at being \emph{viscous}. This horror is expressed more by Sartre through images and feelings than through reason -- but it is quite powerful and compelling -- somehow viscousness derives its horror from being a quality that reminds us of our material nature. Our natural state as a transcendent being is the state of the \emph{flow} -- of this constant motion and movement. Viscousness is an impediment to this flow. It makes it quite difficult to summarise, but here are some key presentations:

\begin{enumerate}
  \item \textcquote[790]{sartre}{To touch the viscous is to risk becoming diluted into viscosity Now this dilution is, taken alone, already terrifying, because it is the for-itself's absorption by the in-itself. But it is additionally terrifying, if one is to be metamorphosed into a thing, that this precise metamorphosis should be into something viscous.}
  \item \textcquote[790]{sartre}{A consciousness that \emph{became viscous} would therefore be transformed by the thickening-out of its ideas.}
  \item \textcquote[791]{sartre}{The horror of the viscous is a horror of time's becoming viscous, of facticity's continual and imperceptible progression, and that it might suck up the for-itself which \emph{exists} it. It is the fear neither of death, nor of the pure-in-itself, nor of nothingness but of a particular type of being, which has no more existence than the in-itself-for-itself and which the viscous merely \emph{represents}.}
\end{enumerate}

\noindent
Sartre ends \textsc{Part IV} with this final parting thought: \textcquote[796]{sartre}{\textbf{Every human-reality is a Passion}, in that its project is to lose itself in order to found being and at the same time to constitute the in-itself as escaping contingency by being its own foundation, the \emph{Ens causa sui} that the religions know as God In this way man's Passion is the opposite of Christ's, because man loses himself as man in order that God should be born. But the idea of God is contradictory, and we lose ourselves in vain; \textbf{man is a useless Passion.}}

\noindent
Thus, we conclude Jean-Paul Sartre's \emph{Being and Nothingness}: an Essay on Phenomenological Ontology.