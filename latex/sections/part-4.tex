\chapter{Part IV: To Have, To Do, and To Be}

So what is \textsc{Part IV} about? In the introduction and the three parts, we've explored the ontology of the in-itself (i.e. the being-of-phenomena), the ontology of the for-itself, and finally the ontology of the Other -- as well as all their interleaving relationships such as the relationship of the for-itself to the in-itself (knowledge) or the relationship of the for-itself to the Other (the being-for-the-Other). With the above ontology established, we are finally able to explore the nature of the being-for-itself as an agent that \emph{acts}. It is this exploration of action that serves as the theme of Part IV. This part serves as the final third of Sartre's monograph on Phenomenological Ontology: \emph{Being and Nothingness}. Sartre introduces this part to us with the following opening-question:

\textcquote[567]{sartre}{Having, doing, and being are the fundamental categories of human reality. Every type of human behaviour can be subsumed within them. \ldots\ Is the supreme value of the human action \emph{to do} or \emph{to be}?}

\section{Chapter 1: Being and Doing: Freedom}

\subsection{The First Condition of Action is Freedom}

In this chapter, we are interested in the ontology of freedom. To begin this investigation, we first look at \emph{actions} -- which are things that our being partakes (i.e. our being \emph{acts}). From where does an action derive its being? Sartre rejects the naively rationalist perspective that actions simply emerge from a series of contingent reasons (even erroneous reasons). Rather, he claims that no material contingency can bring about, or generate the impetus for action -- this is because any material contingency will simply be an objective description of the world in its facticity -- there will be no \emph{lack} that yields the space for action \autocite[574]{sartre}. Instead, action has to come from a \emph{nothingness} -- and that nothingness, which is the same nothingness that founds our lack and desire -- is the source of our freedom. Hence, in order to understand action, we must understand freedom -- the first condition of action.

\begin{enumerate}
  \item \textcquote[569]{sartre}{The point we should note at the outset is that an action is, by definition, \emph{intentional}.}
  \item \textcquote[570]{sartre}{An action necessarily implies, as its condition, some recognised \enquote{desideratum} i.e., an objective lack or even a negatity.}
  \item \textcquote[575]{sartre}{It is the act that determines its end and its motives, and the act is the expression of freedom.}
\end{enumerate}

\subsection{Freedom and Facticity: the Situation}

\begin{enumerate}
  \item To be completed later.
\end{enumerate}

\subsection{Freedom and Responsibility}

\begin{enumerate}
  \item To be completed later.
\end{enumerate}

\section{Chapter 2: To Do and To Have}

\subsection{Existential Psychoanalysis}

\begin{enumerate}
  \item To be completed later.
\end{enumerate}

\subsection{To Do and To Have: Possession}

\begin{enumerate}
  \item To be completed later.
\end{enumerate}

\subsection{The Revelation of Being Through Qualities}

\begin{enumerate}
  \item To be completed later.
\end{enumerate}

\section{Conclusion}

\subsection{In-Itself and For-Itself: Some Metaphysical Observations}

\begin{enumerate}
  \item To be completed later.
\end{enumerate}

\subsection{Moral Perspectives}

\begin{enumerate}
  \item To be completed later.
\end{enumerate}
