\chapter{Epitome of Sartrean Ontology}

% If you must read five pages about Sartre, and not a single more -- this is the place to begin.

What is Sartre's \emph{Being and Nothingness} about? What is a \enquote{\emph{phenomenological ontology?}} What is Sartre's conception of \emph{nothingness}, and why does he place it central to being? This epitome (lit: to cut short) is a summary and a primer for Sartre's unique ontology -- which we now call \emph{Existentialism}. This summary is the text that you want, if you need to read five pages about Sartre but not a single page more. 

\subsubsection{Introduction}

\emph{Being and Nothingness} aims to solve the problem of \emph{being}. What is the \emph{being} of something? Where does it come from? What is its foundation? In the \textsc{Introduction}, Sartre presents these problems, and lays the foundation for their inquiry. We begin with the acknowledgement that we live in a \emph{phenomenal} world -- a world that's full of appearances (i.e. phenomena). Can we use these readily apparent phenomena to account for being? Not all philosophers necessarily believes this is possible -- but Sartre rejects the possibility of a hidden \enquote*{interior} behind phenomena, as an unnecessary \enquote{troubling dualism}. Hence, we will use phenomena to account for being. But what is this \emph{being-of-phenomena}?

Here Sartre makes the nuanced, but important distinction between the \emph{phenomena-of-being} and the \emph{being-of-phenomena}. The phenomena of a being (the first one) is only what we perceive of a being, in other words, an appearance. Presumably there are all kinds of phenomena for a being, and that's not really important or metaphysical. What we're truly trying to find in this inquiry is the \emph{being-of-phenomena} -- we're trying to find out what is the \emph{being} that all phenomena shares. What is this being-of-phenomena, and where does it come from? 

Well, what is a phenomenon in the first place? It is something that \emph{appears}. For there to be a phenomenon in the first place, there has to be \emph{someone} to \emph{perceive} it! This someone is a \emph{being} -- but it cannot be just any being (like a table, or chair). It has to be a \emph{conscious} being. What is this \emph{being-of-consciousness}?\footnote{There is strictly speaking, two \enquote{kinds} of consciousness. There is the \emph{thetic}, positional consciousness -- as well as a \emph{non-thetic} pre-reflective \emph{cogito} which is essential for our consciousness to be \emph{self-conscious}. This distinction is fleshed out by Sartre in \autocite[11]{sartre}.} Here, Sartre asserts that consciousness is both \emph{contentless} and \emph{positional}. This means that consciousness does not have content in itself -- but rather, it is like a spotlight that always shines upon (i.e. \enquote*{posits}) an object of consciousness. You can be conscious of heat, of sensation -- which are all objects -- but the consciousness itself is not filled with anything. But does this mean we can derive the being-of-phenomena from the being of consciousness itself? 

No, we cannot! Sartre rejects this absolutely. Consciousness is contentless, and it cannot give being to anything. The relationships of passivity and relativity do not allow consciousness to give its being to phenomena at all. Consciousness has its being, and phenomena has its being too. Phenomena cannot derive its being from the being of consciousness. But how does this make sense, especially when phenomena (as a perceived thing) can only exist if there is a perceiver? 

This is the crux of Sartre's ontological innovation. \emph{The being of phenomena derives its being only via an act of negation by the consciousness}. What does this mean? Well, let us first consider the being-of-consciousness. What is it? The being-of-consciousness is \emph{pure subjectivity}. Everything within a consciousness is itself. It is a pure plentitude of selfness. It's a mantra that chants \enquote{\ldots\ I am I am I am I am \ldots} without limit. How can any object (i.e. phenomena) enter this pure and absolute self-ness? A phenomena cannot derive its being from this, because if it does so -- it would simply become a part of consciousness. It would be one with the ceaseless chant of \enquote{\ldots I am I am I am I am \ldots} until there is no more distinction between the subject and the object, and hence the phenomena would have no being at all. How can \emph{anything} enter consciousness in the first place?

It would have to enter as a negation -- as a \emph{nothingness} of consciousness. Or in Sartre's words, \emph{the being-of-phenomena is the non-being of the being-of-consciousness}. When a phenomena enters consciousness, the mantra of pure and absolute subjectivity (my image, not Sartre's) is rudely interrupted all of a sudden by something that is \emph{not} the self. The mantra goes: \enquote{\ldots I am I am I am I AM NOT \ldots} -- and all of a sudden, we have a being that is not the being-of-consciousness. This is the act of ontological baryogenesis that creates the being-of-the-phenomena. Where its very nature, is this negation. Hence the title of Sartre's work: \emph{Being and Nothingness}. Now we know what the being-of-phenomena is. As a final question, what is the being-of-consciousness? Sartre gives us the answer as well: \textcquote[23]{sartre}{Consciousness is a being for whom in its being there is a question of its being, insofar as this being implies \textbf{a being other than itself}.} Keep this thought in mind, as we proceed on to \textsc{Part I}.

\subsubsection{Part I: The Problem of Nothingness}

So now that we discover that the being-of-phenomena comes not from any positive act of the being-of-consciousness, but rather from a negative act -- namely, a \emph{nothingness}, we are faced with another question. Where does this nothingness come from? Where does this negation originate? Can nothingness simply come from being -- as through an act of parthenogenesis? Or is there something special about the kinds of being which contain or generate nothingness? This is the subject of \textsc{Part I} of Sartre's work. We begin our investigation gently -- looking first at the phenomena of \emph{questioning} -- since a question is a simple, direct, and practical example of \emph{negation}. 

We discover that not only the act of questioning allows for negation, but within the being of every question there must necessarily be an act of negation. What does this mean? Consider the simplest question: the yes-or-no question. If we allow \enquote{yes} as an answer, we must by necessity presuppose the possibility of \enquote{no} as an answer -- otherwise this wouldn't be a question at all. More fundamentally, behind every question \emph{we confront a being whom we interrogate}. We ask yes-or-no towards a being. And the possibility of a \enquote{no} therefore also \emph{implies the possibility of a non-being}. With this, Sartre introduces \emph{non-being as a necessary aspect of reality}.\footnote{Rather, human-reality -- but we don't get into this distinction until Part III.}

So where does non-being come from? Non-being is a nothingness. But can we dismiss nothingness as a simple \emph{absence} of being? No, we cannot! To quote Sartre: \textcquote[57]{sartre}{Nothingness can only nihilate itself on the ground of being: if nothingness can be given, it is neither before being nor after being; nor is it, in a general way, outside being; rather, it is right inside being, in its heart, like a worm.} Hence nothingness can neither be before, after, nor outside being. So it can only come from \emph{inside} being. What sort of being does \emph{nothingness} come from? Do all beings generate nothingness on the tap? No! No being can \emph{generate} nothingness -- you can never derive non-being from being. Nothingness can only come from as a special \emph{relationship} of a being to another being -- a relationship where we question the being.

There is only one kind of being that possesses the unique ontology capable of his nihilating relationship (remember, not generation). This is the being that possesses within its ontology, the very \emph{question of its own being}. This is the \emph{being-of-consciousness}. Hence, only conscious beings possess this nothingness capable of negation.

After deducing this important ontological insight about where nothingness comes from, Sartre spends the remainder of \textsc{Part I} examining its effects in practice. He talks about how our own, human awareness of this nothingness manifests itself as the feeling of \emph{angst} -- the process of us extending our negation outwards. Likewise, he looks at what happens when we withdraw this negation \emph{inwards}, which manifests itself as \emph{bad faith}. After elucidating these practical manifestations of our being (as a being which contains the question (i.e. nothingness) of its own being), Sartre proceeds on to \textsc{Part II}.

\subsubsection{Part II: Being-for-Itself}

In \textsc{Part II}, Sartre takes us on an examination of the \emph{being-for-itself}.\footnote{There's also the \emph{being-in-itself}, which denotes regular, non-conscious beings like tables, chairs, and rocks. I gloss over it in this summary.} But wait, what is a being-for-itself? So far, we talked about the being-of-phenomena, and the being-of-consciousness. Where did this being-for-itself suddenly come from? Well, recall that consciousness has two components -- there's the \emph{thetic}, positional consciousness -- which is the being-of-consciousness that we talked about earlier. But there's only the \emph{non-thetic}, pre-reflective consciousness, that serves as a necessary condition for our self-awareness. This section is an examination of this non-thetic pre-reflective \emph{cogito} -- that is the being-for-itself. We will be talking about the being-for-itself from now on for the rest of Sartre's text.

So what is a being-for-itself? It is the foundation of our self-awareness -- of how our consciousness is not only positional, but we are (self-) aware of what we are positing. Without the being-for-itself, we would be an unconscious consciousness (which Sartre calls an \enquote{absurdity}). What is the \emph{being}, of this being-for-itself? The key property of the being-for-itself is \emph{self-presence} -- it's what I called self-awareness earlier. What is self-presence? The being-for-itself has a reflective nature -- after all, the very word \emph{for} implies a certain degree of reflectivity. However, this reflectiveness is \emph{not} simply identity! Our being-for-itself is \emph{not} simply our being. It is not a co-incident of being, or an identity of being. But rather, our being-for-itself (the self-presence) is a \emph{separation} from being. \emph{Self presence is an act of separation from the self} \autocite[127]{sartre}.

What separates our being-for-itself from our simple being (the itself)? It is \emph{nothing}. Nothingness is the foundation of our being-for-itself. The fundamental internal relationship of our self-presence (the being-for-itself) to our being is one of \emph{negation}. 

From this foundation in our understanding of the being-for-itself, Sartre takes us on a slight, but pedagogically signifiant detour, on an examination of \emph{lack} and \emph{value}. You'll see why an elucidation of these more derivative concepts is important, once we reconvene on our inquiry on the being-for-itself. So far, we have discussed negation solely in the context of our being-for-itself in a relation with other beings-in-itself (things) -- where the act of negation is our being-for-itself placing external in-itselfs in negation. But in this detour, we will look at how negation affects our being-for-itself.

Sartre begins with an exposition on the \emph{lack}. The lack is the form of negation which \textcquote[137]{sartre}{most deeply establishes an internal relation between what we negate and what our negation applies to \ldots\ [the lack is the form of negation] which penetrates most deeply into being -- the one that constitutes \emph{in its being} the being to which its negation applies with the being that it negates.} Essentially, when we lack something, we state our very being is inadequate for the (lack) of something. Lack leads on to \emph{desire} -- which is a lack of being. For example, in more concrete terms. If I have a house, there is house-being. If the house doesn't have a roof, the lack of the house is the roof which it doesn't have. Its \enquote{desire} would be a roof -- the desire is the hypothetical house-being which has a roof. This example is a flawed one, for houses are beings-in-itself, not for-itself. But humans are beings-for-itself!

This is where Sartre concludes the detour on lack and desire, and takes us on towards an important realisation of our being-for-itself. Human-reality -- our being-for-itself -- is a lack. That's because we are the source of our own negation. What is our desire? It is a being that has what our current being-for-itself lacks. From this, Sartre takes us onwards towards radical, ontologically interesting discussions on temporality and the ontology of the being-for-itself as a temporal being.

\subsubsection{Part III: Being-for-the-Other}

\subsubsection{Part IV: To Have, To Do, To Be}