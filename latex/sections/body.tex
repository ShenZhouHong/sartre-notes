\section{Introduction}

\subsection{I: The Idea of the Phenomenon}
In this section, Sartre introduces us to the problem of modern philosophy, which is its usage of incompatible dualisms. He shows how Phenomenology seemingly resolves these dualisms, but then introduces a dualism of its own: that of the finite and infinite, insofar appearances are concerned. He defines phenomenology as a philosophy in which the \emph{being} of existents are not somehow `behind' the appearances of said existents (e.g. as Immanuel Kant's ontology posits), but rather \emph{the being of an existent is in its appearances}. Sartre then concludes that in order to properly know the nature of such a being, we must investigate the being of appearances in further detail.

\begin{enumerate}
  \item As a preface, Sartre states that one of the accomplishments of modern philosophy was to reduce `existents' to merely the "series of appearances that manifest it" \autocite[1]{sartre}.
  \item This is done in order to eliminate certain `troublesome dualisms.'
  \begin{enumerate}
    \item Such as the dualism of the internal, versus the external.
  \end{enumerate}
  \item "An appearance refers to the total series of appearances, not to some hidden reality that siphons off all the existent's \emph{being} for itself." \autocite[2]{sartre}
  \item \textbf{Noumenal world} versus \textbf{Phenomenal world}: by noumenal, we mean the material (and potentially inaccessible) world outside of us, versus the phenomenal world which consists of what appears to us. The phenomenal world can be understood as our mental world, in a sense.
  \item Sartre proceeds to give us \textbf{a working definition of Phenomenology:}
  \begin{enumerate}
    \item The being of an existent \emph{is in it's appearances}. "For the being of an existent is precisely the way which it \emph{appears}" \autocite[2]{sartre}
    \item The phenomenon is the \emph{absolute-relative}. It remains relative, because the phenomenon has to \emph{appear} to someone. However, it is absolute because the appearance is not merely our perception of a deeper, transcendent being.
    \item But rather, the \emph{appearance is being}.
  \end{enumerate}
  \item The benefit of phenomenology as opposed to other competing ontologies (e.g. Immanuel Kant's transcendental metaphysics) is that it resolves the `troublesome dualisms' of an internal versus external being. However, it brings out new problems:
  \begin{enumerate}
    \item The \textbf{dualism of the finite versus the infinite}: An object (i.e. existent) has an infinity of appearances, as there are an unlimited amount of phenomenal subjects to which the object can appear.
    \item This infinity is necessary for an \emph{objective phenomenology}. "The reality of this cup is that it is there, and that it is \emph{not} me." \autocite[4]{sartre}
    \begin{enumerate}
      \item "We can express this by saying that the series of its appearances is connected by a \emph{principle} that does not depend on my whim." \autocite[4]{sartre}
    \end{enumerate}
    \item However, keep in mind that as subjects, \emph{we only see a finite set of appearances for any given existent at any given time}.
    \begin{enumerate}
      \item And yet, the existent must have an infinite series of appearances.
      \item Hence, there is now \textbf{a dualism between finite and infinite appearances}.
      \item By resolving the dualism of the internal and the external, phenomenology has seemingly introduced a new dualism.
    \end{enumerate}
    \item "Thus, a \emph{finite} appearance indicates itself in its finitude, but at the same time in order to be grasped as an appearance-of-that-which-appears, it demands to be surpassed towards the infinite." \autocite[4]{sartre}.
    \item Hence, the nature of the phenomenon's being has certain \emph{transcendent} properties. What are these properties?
  \end{enumerate}
  \item Recall that the essence, or being of an existent is now completely in its appearance. Hence, in order to properly ground phenomenology, we must investigate the \emph{being} of the appearance itself.
\end{enumerate}

\subsection{II: The Phenomenon of Being and the Being of the Phenomenon}
Sartre presents an important distinction between the \emph{phenomenon-of-being}, and the \emph{being-of-phenomena}. We wish to study the latter, not the former -- even though strictly speaking we only have access to the former (right now, at any rate). Sartre asks us whether or not the phenomenon-of-being is reducible to the being-of-phenomena, to which he concludes this reduction is not possible. This is because any phenomena (appearance) is founded on being: specifically, Sartre says that "being is the condition of a phenomena's disclosure" \autocite[7]{sartre}. Hence, the being-of-phenomena is not in the phenomena itself, but rather has a \emph{transphenomenal foundation}. That is the ultimate conclusion of this section.

\begin{enumerate}
  \item \textbf{Eidetic reduction}: \emph{"a technique in the study of essences in phenomenology whose goal is to identify the basic components of phenomena. Eidetic reduction requires that a phenomenologist examine the essence of a mental object, with the intention of drawing out the absolutely necessary and invariable components that make the mental object what it is. This is achieved by the method known as eidetic variation. It involves imagining an object of the kind under investigation and varying its features. The changed feature is inessential to this kind if the object can survive its change, otherwise it belongs to the kind's essence."} \autocite{enwiki:995672467}
  \item We wish to study the nature of \emph{being}. We define earlier on that the being of any existent is in its phenomenon. Hence, we wish to study the \emph{being-of-phenomenon.} However, \emph{being} itself is also a phenomenon -- that's how we can talk about and reason about it.
  \begin{enumerate}
    \item Hence, there are two concepts we need to understand clearly:
    \item \textbf{The phenomenon-of-being}: an appearance of being.
    \item \textbf{The being-of-phenomenon}: the being of appearances.
  \end{enumerate}
  \item "Is the phenomenon-of-being [that we can reason about] identical to the \emph{being-of-the phenomena}?" \autocite[6]{sartre}
  \begin{enumerate}
    \item After all, remember -- the purpose of our investigation is in the \emph{being-of-the-phenomena}.
    \item Which admittedly, we can only reach through the intermediary of the phenomena (the appearance) of being.
    \item These are somewhat tricky and nit-picky differences, but it is important to keep them in mind!
  \end{enumerate}
  \item The being-of-phenomena cannot be simply resolved into the phenomenon-of-being \autocite[7]{sartre}.
  \begin{enumerate}
    \item This is because phenomena itself can only exist on the foundation of being. Sartre calls being "the condition of all disclosure" for appearances.
  \end{enumerate}
  \item This leads to a certain important ontological insight, which is that \emph{"knowledge alone cannot account for being, i.e., that the being of the phenomenon cannot be reduced to the phenomenon of being"} \autocite[7]{sartre}.
  \item "The phenomenon of being requires the transphenomenality of being." \autocite[7]{sartre}
  \begin{enumerate}
    \item "Although the being of the phenomenon is co-extensive with the phenomenon, it must escape the phenomenal condition in which existence is possible [only as a condition] that it is revealed [i.e. a phenomenon]" \autocite[7]{sartre}.
  \end{enumerate}
  \item I'm not sure if I understand this completely, but essentially Sartre concludes that the being-of-phenomena has a transphenomenal nature, where it is grounded in something that is not strictly just phenomena.
\end{enumerate}

\subsection{The Prereflective Cogito and the Being of the Percipere}

\section{Part I: The Problem of Nothingness}

\subsection{Chapter 1: The Origin of Negation}

\subsection{Chapter 2: Bad Faith}

\section{Part II: Being-For-Itself}

\subsection{Chapter 1: The Immediate Structures of the For-Itself}

\subsection{Chapter 2: Temporality}

\subsection{Chapter 3: Transcendence}

\section{Part III: Being-For-The-Other}

\subsection{Chapter 1: The Other's Existence}

\subsection{Chapter 2: The Body}

\subsection{Chapter 3: Concrete Relations with the Other}

\section{Part IV: To Have, To Do, and To Be}

\subsection{Chapter 1: Being and Doing: Freedom}

\subsection{Chapter 2: To Do and To Have}

\section{Conclusion}
