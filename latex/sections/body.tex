\section{Introduction}

\subsection{I: The Idea of the Phenomenon}

\subsubsection*{Summary}
In this section, Sartre introduces us to the problem of modern philosophy, which is its usage of incompatible dualisms. He shows how Phenomenology seemingly resolves these dualisms, but then introduces a dualism of its own: that of the finite and infinite, insofar appearances are concerned. He defines phenomenology as a philosophy in which the \emph{being} of existents are not somehow `behind' the appearances of said existents (e.g. as Immanuel Kant's ontology posits), but rather \emph{the being of an existent is in its appearances}. Sartre then concludes that in order to properly know the nature of such a being, we must investigate the being of appearances in further detail.

\begin{enumerate}
  \item As a preface, Sartre states that one of the accomplishments of modern philosophy was to reduce `existents' to merely the "series of appearances that manifest it" \autocite[1]{sartre}.
  \item This is done in order to eliminate certain `troublesome dualisms.'
  \begin{enumerate}
    \item Such as the dualism of the internal, versus the external.
  \end{enumerate}
  \item "An appearance refers to the total series of appearances, not to some hidden reality that siphons off all the existent's \emph{being} for itself." \autocite[2]{sartre}
  \item \textbf{Noumenal world} versus \textbf{Phenomenal world}: by noumenal, we mean the material (and potentially inaccessible) world outside of us, versus the phenomenal world which consists of what appears to us. The phenomenal world can be understood as our mental world, in a sense.
  \item Sartre proceeds to give us \textbf{a working definition of Phenomenology:}
  \begin{enumerate}
    \item The being of an existent \emph{is in it's appearances}. "For the being of an existent is precisely the way which it \emph{appears}" \autocite[2]{sartre}
    \item The phenomenon is the \emph{absolute-relative}. It remains relative, because the phenomenon has to \emph{appear} to someone. However, it is absolute because the appearance is not merely our perception of a deeper, transcendent being.
    \item But rather, the \emph{appearance is being}.
  \end{enumerate}
  \item The benefit of phenomenology as opposed to other competing ontologies (e.g. Immanuel Kant's transcendental metaphysics) is that it resolves the `troublesome dualisms' of an internal versus external being. However, it brings out new problems:
  \begin{enumerate}
    \item The \textbf{dualism of the finite versus the infinite}: An object (i.e. existent) has an infinity of appearances, as there are an unlimited amount of phenomenal subjects to which the object can appear.
    \item This infinity is necessary for an \emph{objective phenomenology}. "The reality of this cup is that it is there, and that it is \emph{not} me." \autocite[4]{sartre}
    \begin{enumerate}
      \item "We can express this by saying that the series of its appearances is connected by a \emph{principle} that does not depend on my whim." \autocite[4]{sartre}
    \end{enumerate}
    \item However, keep in mind that as subjects, \emph{we only see a finite set of appearances for any given existent at any given time}.
    \begin{enumerate}
      \item And yet, the existent must have an infinite series of appearances.
      \item Hence, there is now \textbf{a dualism between finite and infinite appearances}.
      \item By resolving the dualism of the internal and the external, phenomenology has seemingly introduced a new dualism.
    \end{enumerate}
    \item "Thus, a \emph{finite} appearance indicates itself in its finitude, but at the same time in order to be grasped as an appearance-of-that-which-appears, it demands to be surpassed towards the infinite." \autocite[4]{sartre}.
    \item Hence, the nature of the phenomenon's being has certain \emph{transcendent} properties. What are these properties?
  \end{enumerate}
  \item Recall that the essence, or being of an existent is now completely in its appearance. Hence, in order to properly ground phenomenology, we must investigate the \emph{being} of the appearance itself.
\end{enumerate}

\subsection{II: The Phenomenon of Being and the Being of the Phenomenon}

\subsubsection*{Summary}
Sartre presents an important distinction between the \emph{phenomenon-of-being}, and the \emph{being-of-phenomena}. We wish to study the latter, not the former -- even though strictly speaking we only have access to the former (right now, at any rate). Sartre asks us whether or not the phenomenon-of-being is reducible to the being-of-phenomena, to which he concludes this reduction is not possible. This is because any phenomena (appearance) is founded on being: specifically, Sartre says that "being is the condition of a phenomena's disclosure" \autocite[7]{sartre}. Hence, the being-of-phenomena is not in the phenomena itself, but rather has a \emph{transphenomenal foundation}. That is the ultimate conclusion of this section.

\begin{enumerate}
  \item \textbf{Eidetic reduction}: \emph{"a technique in the study of essences in phenomenology whose goal is to identify the basic components of phenomena. Eidetic reduction requires that a phenomenologist examine the essence of a mental object, with the intention of drawing out the absolutely necessary and invariable components that make the mental object what it is. This is achieved by the method known as eidetic variation. It involves imagining an object of the kind under investigation and varying its features. The changed feature is inessential to this kind if the object can survive its change, otherwise it belongs to the kind's essence."} \autocite{enwiki:995672467}
  \item We wish to study the nature of \emph{being}. We define earlier on that the being of any existent is in its phenomenon. Hence, we wish to study the \emph{being-of-phenomenon.} However, \emph{being} itself is also a phenomenon -- that's how we can talk about and reason about it.
  \begin{enumerate}
    \item Hence, there are two concepts we need to understand clearly:
    \item \textbf{The phenomenon-of-being}: an appearance of being.
    \item \textbf{The being-of-phenomenon}: the being of appearances. (This is what we want to study!)
  \end{enumerate}
  \item "Is the phenomenon-of-being [that we can reason about] identical to the \emph{being-of-the phenomena}?" \autocite[6]{sartre}
  \begin{enumerate}
    \item After all, remember -- the purpose of our investigation is in the \emph{being-of-the-phenomena}.
    \item Which admittedly, we can only reach through the intermediary of the phenomena (the appearance) of being.
    \item These are somewhat tricky and nit-picky differences, but it is important to keep them in mind!
  \end{enumerate}
  \item The being-of-phenomena cannot be simply resolved into the phenomenon-of-being \autocite[7]{sartre}.
  \begin{enumerate}
    \item This is because phenomena itself can only exist on the foundation of being. Sartre calls being "the condition of all disclosure" for appearances.
  \end{enumerate}
  \item This leads to a certain important ontological insight, which is that \emph{"knowledge alone cannot account for being, i.e., that the being of the phenomenon cannot be reduced to the phenomenon of being"} \autocite[7]{sartre}.
  \item "The phenomenon of being requires the transphenomenality of being." \autocite[7]{sartre}
  \begin{enumerate}
    \item "Although the being of the phenomenon is co-extensive with the phenomenon, it must escape the phenomenal condition in which existence is possible [only as a condition] that it is revealed [i.e. a phenomenon]" \autocite[7]{sartre}.
  \end{enumerate}
  \item I'm not sure if I understand this completely, but essentially Sartre concludes that the being-of-phenomena has a transphenomenal nature, where it is grounded in something that is not strictly just phenomena.
\end{enumerate}

\subsection{The Prereflective Cogito and the Being of the Percipere}

\subsubsection*{Summary}
In this section, Sartre asks whether or not the being-of-phenomena can be found in \emph{knowledge}, which is the proportionality of an appearance's \emph{appearing} (e.g. we have more knowledge of a phenomenon, when it appears more strongly to us). He examines this knowledge-hypothesis of being (my term for it), before concluding that it is not an adequate answer, for to place the being of phenomena into knowledge invites an infinite regression.

He then (rather confusingly) goes on a tangent where he resolves this infinite regression, by showing that knowledge is related to consciousness -- for in order to know (percipere) something (a percipi), there has to be a knower (percipiens). He examines the nature of the percipiens (consciousness), and elaborates on it's characteristics. The most important is that consciousness is \emph{positional}, namely that it \emph{posits} towards something (i.e. an object of consciousness). But otherwise, consciousness is itself contentless and empty.

Finally, he explains how consciousness is self-conscious. He first posits that consciousness is by necessity self-conscious, and demonstrates the inadequacy of a simple consciousness-of-consciousness approach, because that too leads to an infinite regression. Instead, he shows that the self-consciousness of consciousness lies in a certain \emph{non-cognitive, non-positional consciousness}, which he calls the \emph{pre-reflective cogito}. It's called pre-reflective, because it is prior to the more ordinary positional (i.e. reflective) cogito.

In the most confusing part, he demonstrates that this pre-reflective cogito is an essential characteristic of consciousness itself, and that they are one and the same. Every act of consciousness is both a consciousness of an object (positional), and conscious of itself. This is justified using Husserl's factual necessity. With this, he finishes his digression on the nature of consciousness.

\begin{enumerate}
  \item Some introductory Latin definitions:
  \begin{enumerate}
    \item \textbf{Percipere}: Infinitive. To perceive, to learn, to secure.
    \item \textbf{Percipi}: Passive, the perceived.
    \item \textbf{Percipiens} The perceiver.
  \end{enumerate}

  \subsubsection{The Knowledge-Hypothesis of Being}
  \item Sartre addresses the hypothesis that the being of an appearance (i.e. being-of-phenomenon) is rooted in the way that it \emph{appears}.
  \begin{enumerate}
    \item "The being of an appearance is proportionate to its \emph{appearing}" \autocite[8]{sartre}.
    \item This hypothesis attempts to reduce the being-of-phenomenon to a simple matter of our knowledge (where knowledge is taken as the proportion of an existent's appearing to us).
  \end{enumerate}
  \item In Sartre's latin vocabulary, he christians this knowledge-hypothesis with the following proportionality: The phenomena-of-being appears to a perceiver, as a \emph{percipi} to an \emph{percipiens}. This proportionality is \emph{knowledge}.
  \begin{enumerate}
    \item Hence, under the knowledge-hypothesis, the being-of-phenomena lays within knowledge.
    \item However, this hypothesis is inadequate, since we essentially move the burden of "where is being" on towards knowledge:
  \end{enumerate}
  \item "If any metaphysics presupposes a theory of knowledge, it is equally true that any theory of knowledge presupposes a metaphysics" \autocite[8]{sartre}.
  \item A potential response to this inadequacy of the knowledge-hypothesis is to state that the being of the percipi is founded in the being percipiens, i.e. the perceiver.
  \begin{enumerate}
    \item "One might allow that the percipi refers to a being that escapes the laws of appearance while stil maintaining that this transphenomenal being is the \emph{subject}'s being." \autocite[9]{sartre}
    \item This points us towards the direction of consciousness. Could the being-of-phenomena lie within consciousness?
    \item In order to properly evaluate the knowledge-hypothesis of being, we must have a good understanding of consciousness. This is what Sartre does next.

    \subsubsection{Understanding Consciousness (As a Potential Resolution for the Being-of-Phenomena)}
    \item Sartre presents a very Husserl-style \textbf{phenomenological definition for consciousness}:
    \begin{enumerate}
      \item "As Husserl showed, \emph{all} consciousness is consciousness \emph{of} something. In other words, there is no [act of] consciousness that does not \emph{posit} a transcendent object" \autocite[9]{sartre}
      \item Consciousness is a \emph{posit}-ional consciousness of objects in the world, where consciousness has no content, but only aims to reach (or in other words, \emph{posits}) towards objects.
      \item "Every knowing consciousness can only be knowledge of it's [posited] object" \autocite[9]{sartre}.
      \item Essentially consciousness is an action towards things (it posits). It is not an object which has content.
    \end{enumerate}
    \item However, even though consciousness is a contentless action towards an object, \emph{consciousness must always be aware of its own consciousness towards said object}. In other words, the act of consciousness must be conscious of itself being said act.
    \begin{enumerate}
      \item This is absolutely necessary, otherwise if I was conscious of an object without being conscious that I am conscious of an object, I would become an \emph{unconscious conscious}, which is an absurdity.
      \item However, this strict requirement that consciousness must be aware of itself, also leads to it's downfall so far as a hypothesis for the being-of-phenomena as consciousness, which Sartre will present in the following.
    \end{enumerate}
  \end{enumerate}

  \subsubsection{The Inadequacy of Consciousness-of-Consciousness}
  \item "The reduction of consciousness to knowledge effectively imports the subject-object duality within consciousness" \autocite[10]{sartre}.
  \item Furthermore, recall how consciousness must be aware of itself? This leads to a consciousness-of-consciousness. But wouldn't that consciousness need another consciousness, e.g. a consciousness-of-consciousness-of-consciousness? This by necessity leads to an infinite recursion.
  \begin{enumerate}
    \item "Either we stop at some term within the series -- in which case the phenomenon in its totality collapses into the unknown (i.e. we always come up against a reflection that is not conscious of itself and is the final term) -- or we declare an infinite regress to be necessary, which is absurd" \autocite[11]{sartre}.
  \end{enumerate}

  \subsubsection{The (Reflective) Non-Positional Consciousness Towards Positional Consciousness Hypothesis}
  \item In order to prevent the absurdity of this infinite recursion, we cannot have an isomorphic relationship of consciousness-to-consciousness. Rather, we must have an "immediate and non-cognitive relationship of self to self" \autocite[11]{sartre}.
  \begin{enumerate}
    \item "Any positional consciousness of an object is at the same time a non-positional consciousness of itself"
  \end{enumerate}
  \item My understanding of this step is not very clear, but the essential meaning is as follows. Every act of consciousness must have a relationship to itself which is non-consciousness (or at least, non-positionally conscious). Sartre provides the following example, which I will reproduce:
  \begin{enumerate}
    \item Imagine yourself counting cigarettes (sartre's example). You count that there are twelve of them. That's an objective property which you are \emph{positionally} conscious of (i.e. your consciousness posits the twelve-ness of these cigarettes).
    \item However, this positional awareness (i.e. consciousness) of the number of cigarettes appears to you \emph{as a direct, immediate property of the world.} You don't have a positional consciousness of counting them.
    \begin{enumerate}
      \item "I do not "know myself as counting". Proof of this can be seen in the fact that children who are capable of spontaneous addition are unable to \emph{explain} afterwards how they did it."
    \end{enumerate}
    \item Hence, any positional act of consciousness contains a reflective act that is non-cognitive and non-positional.
  \end{enumerate}
  \item "Thus, reflection lacks any kind of primacy in relation to the reflected consciousness: it is not by the means of [reflection] that the [primary, positional] consciousness is revealed to itself" \autocite[12]{sartre}
  \begin{enumerate}
    \item "On the contrary, non-reflective consciousness is what makes reflection possible. There is a prereflective \emph{cogito}, which is the condition of the Cartesian [positional, primary] cogito" \autocite[12]{sartre}
  \end{enumerate}
  \item In summary, consciousness is contentless and positional. However, consciousness itself requires a reflection against a non-cognitive, non-positional act of consciousness, which Sartre calls a pre-reflective \emph{cogito}
  \item What is the nature of this pre-reflective \emph{cogito?} Sartre explores it next.

  \subsubsection{The Nature of the Pre-Reflective Cogito}
  \item \textbf{The pre-reflective cogito}, i.e. the non-positional consciousness is essential to the positional consciousness.
  \item We cannot talk of positional consciousness at all, without acknowledging that it is integrally tied with the consciousness of itself. Sartre puts this into more definite words:
  \begin{enumerate}
    \item "Any conscious existence exists as the consciousness of existing. The most basic consciousness of consciousness is not positional, because it and the consciousness of which it is conscious [i.e. the aforementioned positional consciousness] are one and the same."
    \item "In a single movement, consciousness determines itself as consciousness of perception, and as perception" \autocite[12]{sartre}
    \item "[This integral consciousness of self] is the only possible mode of existence for any consciousness of something" \autocite[13]{sartre}.
  \end{enumerate}
  \item Let me try to explain the above in my own words. The syllogism is thus:
  \begin{enumerate}
    \item Consciousness is positional and contentless.
    \item However, to be conscious of something, we must \emph{also} be conscious of our consciousness of something. We must be conscious of our positional consciousness.
    \item It is not possible to have this second consciousness to be the same as the first consciousness, for then we would yield an infinite regression, which is absurd.
    \item Hence, the second consciousness must be by necessity non-cognitive.
    \item We conclude that the consciousness of consciousness must be integral to the first consciousness.
  \end{enumerate}
  \item It's not the best explanation, but it captures the gist of Sartre's idea.

  \subsubsection{Sartre's Elaboration on the Nature and Qualities of Consciousness}
  \item Given the above understanding of consciousness, Sartre proceeds to explain and explicate certain characteristics. I'll give a rough sketch of the points he makes:
  \item Consciousness in its being implies the existence of its essence. There's no essence behind consciousness. This is justified by \textbf{Husserl's doctrine of factual necessity}:
  \begin{enumerate}
    \item Consciousness does not necessarily have to exist. But once it does exist, its non-existence is inconceivable.
  \end{enumerate}
  \item "Consciousness is prior to nothingness, and ``derives from being''" \autocite[15]{sartre}
  \item "This does not imply at all that consciousness is the foundation of its being."
  \item The sort of conclusion of this section is that consciousness has a certain bootstrapping nature -- it's derives from it's own being (?)
  \item "[Consciousness] is a pure ``appearance,'' where this means it exists only to the extent to which it appears. But it is precisely because consciousness is pure appearance, because it is a total void (since the entire world is outside it), because of this identity within it between its appearance and its existence, that it can be considered as the absolute" \autocite[16]{sartre}.
\end{enumerate}

\subsection{IV. The Being of the Percipi}
\subsubsection*{Summary}
This chapter of the introduction is ontologically sophisticated (as always, with Sartre) but thankfully simple in its general logical progression. After coming to a good understanding that consciousness has it's own being (which is a transphenomenal being), we ask ourselves the big question: "is the transphenomenal being-of-phenomena" located in the transphenomenal being-of-consciousness?

At first, this seems like a plausible argument. However, Sartre spends this chapter refuting it -- demonstrating that the being-of-phenomena absolutely cannot be derived or founded in the being-of-consciousness itself. He presents two arguments which propose the above, and then refutes them. The first argument is that phenomena are \emph{passive}, and hence it must derive it's being from the consciousness to whom it appears, which is active. Sartre refutes this argument by demonstrating that \emph{passivity is a relationship between being and (another) being}, and hence phenomena must have its own being, and not just be an emptiness to which the being-of-consciousness fills.

The second argument is that phenomena is relative to the consciousness to whom it appears, but Sartre likewise refutes this argument by showing that phenomena and consciousness do not have a two-way relationship. Phenomena is relative to consciousness only, there is no case that consciousness is relative to phenomena. Hence phenomena cannot derive its being from something that it has no access to.

Essentially, the goal of this chapter is to demonstrate that \textbf{phenomena must have its own being independent of the being-of-consciousness.} Thus, there is such a thing as a being-of-phenomena.

\begin{enumerate}
  \item After the lengthy elaboration in the preceding section, we come to the definite thesis that there is such a thing that is consciousness, which has its own being. Furthermore, consciousness is an essential condition for appearances.
  \item "We have escaped idealism, according to which being is measured by knowledge \ldots\ for idealism, all being is \emph{known}, including thought itself \ldots\ and the philosopher in search of thought is obliged to consult the constituted sciences in order to derive thought from them as their condition of possibility" \autocite[17]{sartre}
  \begin{enumerate}
    \item This is the poor neuroscientist cutting slices of the pre-frontal cortex trying to find the being of thought.
  \end{enumerate}
  \item Essentially, we now have a good working understanding of the \emph{being-of-percipiens} -- i.e. the being-of-consciousness. However, we must now ask ourselves: \emph{"Is the being-of-consciousness the foundation of the being-of-phenomena?"}
  \item At first, this seems plausible. After all, we demonstrate in the prior lengthy digression that consciousness is a necessary condition for phenomena. After all, you can't have anything appear if there is no one for it to appear to.
  \item "Are we satisfied? We have found a transphenomenal being, but is that really the being to which the phenomenon of being points? Is [the transphenomenal being of consciousness] really the being of the phenomenon?" \autocite[17]{Sartre}.
  \item Oh boy. It's not, and Sartre will demonstrate in this chapter which is called the \textbf{Being of the Percipi}.
  \item This is because the percipi (i.e. the perceived thing, the phenomena) has it's own being which is \emph{irreducible to the being of the percipiens} (i.e. the perceiver, consciousness)
  \begin{enumerate}
    \item "The known [thing] cannot be absorbed into our knowledge of it, we must recognise its \emph{being}. This \emph{being}, as we are told, is the \emph{percipi} \ldots\ The most we can say is that [the percipi] is relative to [the percipiens]" \autocite[17]{sartre}.
  \end{enumerate}

  \subsubsection{Refuting Two Attempts To Derive Being-of-Phenomena From Consciousness}
  \item Sartre goes on to answer two other attempts to found the being-of-phenomena into the being-of-consciousness.
  \begin{enumerate}
    \item The first objection is that the phenomena is \emph{passive}, and hence phenomena must derive it's being from the consciousness to whom it appears (the consciousness is active).
    \begin{enumerate}
      \item But Sartre goes on to demonstrate that this passivity is an active action between two beings:
      \item "Passivity does not involve the very being of the passive existent: it is a relation between one being and another being, and not being a being and a nothingness." \autocite[18]{sartre}
      \item Hence even though phenomena is passive, it still has a being that comes from elsewhere than consciousness.
    \end{enumerate}
    \item The second objection is that phenomena is \emph{relative}, to the consciousness to which it appears. "Is it conceivable that the being of the known [being-of-phenomena] should be relative to our knowledge of it?" \autocite[20]{sartre}
    \begin{enumerate}
      \item This also cannot be the case, being an existent (i.e. a phenomena) does not have a two-way relationship to the consciousness to whom it appears. An existent is relative to the consciousness, but a consciousness is \emph{not} relative to the existent. Hence, just because phenomena is relative, does not mean the being-of-phenomena stems from the being-of-consciousness.
      \item "The perceived being stands before a consciousness that it cannot penetrate and that it cannot make contact with, and as it is cut off from consciousness, it exists cut off from its own existence [which is absurd]." \autocite[20]{sartre}
    \end{enumerate}
  \end{enumerate}
  \item "Thus, there is no case in which either of the two determinations of \emph{relativity} or \emph{passivity} -- which concerns the ways of being -- is applicable to being itself. The esse [being] of the phenomena cannot be its percipi. The transphenomenal being-of-consciousness cannot provide the foundation for the phenomenon's transphenomenal being." \autocite[20]{sartre}
\end{enumerate}

\subsection{The Ontological Proof}
\subsubsection*{Summary}
Sartre presents the ontological proof, which demonstrates that there exists such a thing which is a transphenomenal (outside of the phenomena) being for phenomenon. This is the being-of-phenomena (\emph{not} the phenomena-of-being!) which we have been searching for all along. He proves the existence of the transphenomenal being-for-phenomenon by using the being-of-consciousness as an instrument. Specifically, he looks at the way in which consciousness is positional towards objects (of consciousness).

First, he shows that the objects of consciousness must be related to the transphenomenal being of consciousness itself. They can only be related in two ways -- either the transphenomenal being of consciousness is located in the object, or the transphenomenal being of consciousness is related to the transphenomenal being of phenomena (which we seek).

The first possibility is false, because a concrete object cannot contain a transphenomenal consciousness. This I understand reasonably well. Sartre goes on to show that the second possibility is equally false, for we have already demonstrated that the transphenomenality of the being-of-phenomena cannot be derived from the transphenomenality of the being-of-consciousness.

However, Sartre than proceeds to show a third possibility. Which is that the transphenomenal being of phenomena is found in consciousness not through certain forms of \emph{presence}, but rather from \emph{absence}. What does this mean? We are conscious of an object, because it appears to us as a phenomena. However, any object can have an arbitrarily infinite amount of appearances, but we only see one. Hence, the being of the object is defined to us not as the presence of an appearance, but rather as the \emph{absence of all but one appearances}.

From this, Sartre concludes that the being-of-phenomena is a non-being of the being-of-consciousness. Or in other words, the being-of-phenomena comes from the non-being which sets it apart as an object away from the being of our consciousness, hence making it \emph{objective}. This is my best understanding of this section so far.

\begin{enumerate}
  \item Sartre begins with the claim that "the transphenomenality of consciousness actually requires the phenomenon's being to be transphenomenal," to which he then goes to demonstrate using an "ontological proof" \autocite[20]{sartre}.
  \begin{enumerate}
    \item \textbf{transphenomenal}: the quality of being beyond the phenomena, i.e. transcendent.
  \end{enumerate}
  \item This ontological proof is very difficult to understand. It's logical structure seems to be of the following order:
  \begin{enumerate}
    \item We grant that consciousness must be consciousness \emph{of} something (such as an object). Now we flip the terms, and look at the above postulate from the perspective of the object. There are two possibilities:
    \begin{enumerate}
      \item Either consciousness is constitutive [i.e. the structure which describes] of it's object's being,
      \item Or consciousness is in its innermost nature related to a transcendent being.
    \end{enumerate}
    \item The first possibility is absurd, because if an object is structured by consciousness, it would be conscious itself.
    \item The second possibility is the one which Sartre follows.
  \end{enumerate}
  \subsubsection{The Object of Consciousness Derives Its Being Negatively}
  \item What does it mean for consciousness to be related to a transcendent being? It's not possible for consciousness to directly relate to a transcendent being, because remember -- consciousness is "real subjectivity," and to posit anything transcendent beyond the conscious subject is against the very definition of consciousness.
  \item Sartre resolves this issue by discovering that the being of the object of consciousness comes from a negative act:
  \begin{enumerate}
    \item "The fact that it is necessarily impossible for the infinite number of terms [appearances] in the series to stand before consciousness simultaneously, in conjunction with the fact that all but one of these terms is really absent, is the foundation of objectivity." \autocite[21]{sartre}
    \item "[For] if these impressions were present -- even if their number were infinite -- they would become merged into subjectivity; [hence it is the absence of the infinite term of impressions] which is what gives them [the object] objective being."\autocite[21]{sartre}.
  \end{enumerate}
  \item Hence, the being-of-phenomena (i.e. the being of the object) comes from consciousness as a a "pure non-being" \autocite[22]{sartre}.

  \subsubsection{Summary on the Being-of-Consciousness}
  \item Sartre wraps things up here by giving some concluding remarks on the self-bootstrapping nature of consciousness.
  \item "Consciousness is a being whose existence posits its essence and, inversely, it is conscious of a being whose essence implies its existence" \autocite[23]{sartre}
  \begin{enumerate}
    \item If consciousness exists, it posits that it exists.
    \item If there is a phenomena, the essence of the phenomena implies the existence of a consciousness to which it appears.
  \end{enumerate}
  \item "Consciousness is a being for whom in its being there is a question of its being, insofar as this being implies \textbf{a being other than itself}." \autocite[23]{sartre}
  \begin{enumerate}
    \item What is this \textbf{being other than itself [consciousness]}? Sartre claims that this is the transphenomenal being of phenomena, the thing which we are looking for all along.
  \end{enumerate}

\end{enumerate}
\subsection{Being in Itself}
\subsubsection*{Summary}

\begin{enumerate}
  \item "Consciousness is a revealed-revelation of existents, and these existents appear before consciousness on the foundation of their being" \autocite[24]{sartre}
  \begin{enumerate}
    \item In other words: Consciousness is an appearing appearance of existents.
    \item It is an appearing appearance because we are conscious of our consciousness.
    \item These existents appear since they are (they appear since they have being).
  \end{enumerate}
  \item Hence we have the fundamental law of existents (my words): For something to exist, it has to be.
  \item Now Sartre presents a new, and important step in his reasoning. \textbf{Consciousness is ontico-ontological}: "Consciousness [as an action can] always surpass an existent, not towards its being but towards this being's \emph{meaning}" \autocite[24]{sartre}.
  \begin{enumerate}
    \item What is the \emph{meaning} of a being? Sartre says that it is the phenomenon-of-being (?). "The meaning itself has a being," and because meaning has a being in the first place, it can manifest itself as a phenomenon.
  \end{enumerate}
  \item I'm not entirely clear, but it seems to me that the ultimate aim which we seek to understand is this meaning of being. It is the a phenomenon, which can appear to our consciousness. However, we do not need to go on to find the meaning of the being of meaning anymore, because they are the same. Hence, we no longer have a vicious cycle.
  \item Sartre notes that we must make the distinction between two types of being. Right now we are investigating the being-of-phenomena, which comes from meaning. However, this is different from the being of consciousness, which comes from being-for-itself.
  \item Sartre says there are two distinct regions of being:
  \begin{enumerate}
    \item The being of the \emph{prereflective cogito}
    \item the being of the phenomenon (which we are investigating now).
  \end{enumerate}
\end{enumerate}

\section{Part I: The Problem of Nothingness}
Where does nothignness come from? This is the thematic question of section one. The general progression of this section's content and argument is as follows:

\begin{enumerate}
  \item First, we inquire about negation, which appears as a simple and rather direct phenomenon of nothingness.
  \begin{enumerate}
    \item After all, it's easy to posit positive being by making definite statements, e.g. "there is an apple." However, when we make a negation of a definite statements e.g. "there is no apple," it is clear that we are making a statement that's rooted in a certain conception of non-being.
  \end{enumerate}
  \item We ultimately conclude that it is not possible to derive negation from being, but instead negation must be derived from a certain definite non-being, i.e. \emph{nothingness}
  \item We try to investigate where this nothingness comes from. Sartre ultimately concludes that this \emph{nothingness cannot come from regular being, but it has to come from a being through which can be its own nothingness}
  \begin{enumerate}
    \item Sartre concludes that this being is the human being (i.e. the Daesin).
  \end{enumerate}
  \item Hence, in order to investigate nothingness, we have to investigate the human being.
  \item How is it possible for the human consciousness to experience nothingness? The answer to this is \emph{anguish}.
\end{enumerate}

\subsection{Chapter 1: The Origin of Negation}

\subsubsection{Questioning}
In this section, Sartre presents the necessity for non-being in all forms of questioning. Specifically:

\begin{enumerate}
  \item In every question, we confront a being that we interrogate.
  \item The answer to the interrogation can be either `Yes' (affirmative) or `No' (negative). To allow for an affirmative answer by necessity presupposes the possibility of a negative answer.
  \item After all, the very being of an affirmative answer is defined by its shadow, which is the negative part. When we say that X is Z, we are also simultaneously saying X is not A, B, C, \ldots\ Y.
  \item Hence, there is such a thing as non-being
\end{enumerate}

\noindent
With this, Sartre introduces the concept of non-being as a component of reality, towards which we must investigate further.

\subsubsection{Negations}

\begin{enumerate}
  \item "It is not true that negation is merely a quality of judgement" \autocite[38]{sartre}
  \item In the process of questioning, we expect a being (which is the answer). But we can equally receive a non-being as a response. Sartre's analogy is that if a watch-maker questions a watch on why it's not working, it is perfectly plausible to receive a non-being as a response, e.g. the mainspring is missing.
  \item "A being is \emph{fragile} if it bears within its being a clear-cut possibility of non-being." \autocite[40]{sartre}
  \item It seems that nothingness is distinct from the process of thought which is negation. We need nothingness in order to separate beings from each other, but the thought process that is negation is more simple and less fundamental.
  \item Furthermore, Sartre claims that the thought process of negation must come from the being of nothingness.
  \item "If there is being everywhere, it is not only nothingness that becomes inconceivable: from being we can never derive negation" \autocite[44]{sartre}.
\end{enumerate}
\subsubsection{The Dialectical Conception of Nothingness}
This section seems to be a general review of a dialectical (i.e. Hegelian) conception of nothingness, but presents a few important conclusions. Sartre demonstrates that \emph{it is impossible to dismiss nothingness} as either a shadow of being, nor as the absence or something before or after being. But rather nothingness has a definite existence as a special sort of being, which has the characteristic of its own negation (which he will elaborate in a later section). This crux of this demonstration lies in the neccessity of defining positive being using negations (in his example with distance and lengths), as well as with the existence of \textbf{negatities}.

\noindent
In conclusion, nothingness cannot be outside of being.

\begin{enumerate}
  \item Sartre presents a few examples here on how it is absolutely impossible to abstract away the being of nothingness into a simple quality of regular (i.e. \emph{positive}) being:
  \begin{enumerate}
    \item "Take for example the notion of distance \ldots\ it is easy to see the [distance] contains a negative moment: two points are distant when a specific length \emph{separates} them."
    \item "How might wish toreduce distance to being \emph{no more than} the length of the segments of which the [two points are] the limits \ldots\ [but] in this case we have switched the direction of our attention \ldots\ negation, expelled from the segment and its length, will take refuge in the two \emph{limits}" \autocite[55]{sartre}.
  \end{enumerate}
  \item This leads to the introduction of the existence of \textbf{negatities}, i.e. \emph{negative entities}. These things cannot be accounted for as positive being, but as beings which "inhabited in their internal structure by negation as a necessary condition of their existence" \autocite[56]{sartre}
  \item "Nothingness can only nihilate itself on the ground of being: if nothingness can be given, it is neither before being nor after being; nor is it, in a general way, outside being; rather, it is right inside being, in its heart, like a worm." \autocite[57]{sartre}
\end{enumerate}
\subsubsection{The Phenomenological Conception of Nothingness}

\subsubsection{The Origin of Nothingness}

\subsection{Chapter 2: Bad Faith}

\section{Part II: Being-For-Itself}

\subsection{Chapter 1: The Immediate Structures of the For-Itself}

\subsection{Chapter 2: Temporality}

\subsection{Chapter 3: Transcendence}

\section{Part III: Being-For-The-Other}

\subsection{Chapter 1: The Other's Existence}

\subsection{Chapter 2: The Body}

\subsection{Chapter 3: Concrete Relations with the Other}

\section{Part IV: To Have, To Do, and To Be}

\subsection{Chapter 1: Being and Doing: Freedom}

\subsection{Chapter 2: To Do and To Have}

\section{Conclusion}
