\chapter{Part III: Being-For-The-Other}

\section{Chapter 1: The Other's Existence}

\subsection{The Problem}

This is the introductory section \autocite[307 -- 309]{sartre} which lays out Sartre's exposition for \enquote{the problem.} Sartre admits that in all of the preceding discussions of the being-for-itself (i.e. the conscious being) and the being-in-itself (i.e. objective beings), we have left out one important and concrete mode of ontology. This mode of ontology is of the being-for-the-other -- which from my understanding so far, is the mode of being that a being-for-itself takes, when it is contemplated by another being-for-itself. The example that Sartre uses to best highlight the distinct ontological nature of a \emph{being-for-the-other} is through the emotion of \emph{shame.} Sartre uses shame as the emotion to portray this distinction because shame is unique in how it contains a component of self-recognition. It also neccesiates an \emph{Other}, for it is impossible to feel self by oneself. \textcquote[308]{sartre}{shame in its primary structure is \emph{shame before somebody.}}

\begin{enumerate}
  \item \textcquote[308]{sartre}{Shame is \emph{recognition.} I recognise that \emph{I am} as the Other sees me.}
  \item \textcquote[308]{sartre}{Thus the Other has not only shown me what I was; he has constituted me in a new type of being, obliging me to support new qualifications.}
  \item \textcquote[309]{sartre}{But this new being that appears \emph{for} the Other does not reside \emph{in} the Other; I am responsible for it \ldots\ In this way shame is shame \emph{of myself before the Other}; these two structures are inseparable. But, by the same token, I need the Other in order to fully grasp all the structures of my being; my for-itself refers to my for-the-other.}
\end{enumerate}

\subsection{The Reef of Solipsism}

Sartre begins his exploration on the Other's existence \autocite[309 -- 322]{sartre} through an examination of the inadequacy of both \emph{realism} and \emph{idealism} to account for the existence of the Other. We are confronted with the question of \enquote{How can there be an Other?} Right now, in this pre-Other ontology of ours, there are only two kinds of beings -- beings-for-itself (the conscious being) which is the self -- and beings-in-itself, objects of our positional consciousness \autocite[309]{sartre}.

\subsubsection{Inadequacies of \emph{Realism} to justify the existence of the \emph{Other}}

We can attempt to furnish an explanation for the existence of the Other through \emph{Realism.} We observe that there is another body, that can be the subject of our intuition and consciousness. We observe the empirical phenomena associated with the other's body -- and that there are patterns that can be followed. We ultimately come to the conclusion that that the other body's behaviour can only be explained by the existence of another positional consciousness (a being-for-itself) other than our own. \textcquote[311]{sartre}{The hypothesis that best accounts for [the Other body's] ways of behaving is that of a consciousness that is analogous to mine, whose various emotions it reflects.}

However, this hypothesis -- while plausible, can never be ontologically certain -- the realist will always have other equally plausible explanations of the Other as only a body -- a mechanistic being, a perspective of pure behavioralism. \textcquote[311]{sartre}{if the Other is accessible to us only through the knowledge that we have of him, and if this knowledge is only conjectural, the Other's existence is only conjectural.}  In fact, Sartre goes on to critique the inadequacies of the realist explanation for the existence of the Other as downright a recourse to idealism. After all, if the existence of the other is conjecture and hypothesis -- it is no different from saying that it has no real foundation, but is merely ideal \autocite[312]{sartre}.

\subsubsection{Inadequacies of \emph{Idealism} to justify the existence of the \emph{Other}}

Moving on, Sartre then looks at explanations of the \emph{Other} from the perspective of \emph{Idealism} \autocite[312]{sartre}. He takes an early look at Immanuel Kant's perspective of the transcendental metaphysics -- which presents the other is an object of our experience. This makes the Other entirely a phenomenon. However, there still exists the difficulty of explaining the \emph{being} that's behind the Other. It's hard to not see the Other as \emph{noumenal} -- which would make it a thing that's entirely beyond our intuition altogether.

In this line of reasoning, Sartre proceeds to examine other \autocite[315]{sartre} potential explanations for the foundation of the Other's being, such as the Other as a \enquote{regulative concept} or \enquote{organising principle.} All are weak and inadequate:

\textcquote[315]{sartre}{Consequently, the Other is not, within my experience, a phenomenon that points toward my experience [i.e. any idealist perception] but rather one that refers in principle to phenomena that are situated outside any possible experience for me \ldots\ the concept of the Other is not purely instrumental [i.e. any organising principle]: rather than its existing \emph{for} use in the unification of phenomena, we ought on the contrary to say that various categories of phenomena seem only to exist \emph{for} it.}

\subsubsection{Where is the Other?}

It's that final line -- where \enquote{various categories of phenomena seem only to exist \emph{for} it} -- that hints towards the ultimate ontological foundation of the Other:

\begin{enumerate}
  \item \textcquote[316]{sartre}{That \emph{Other}, whose relation to myself cannot be apprehended, and who is never given, is gradually constituted by us as a concrete object:
  \marginnote{Do we once again see the familiar them of seperation, or negation, as the root behind a being's ontology?}
  he is not the instrument of which I make use to predict an event within my experience; rather, it is the events in my experience that used to constitute the Other as Other, i.e. as a system of representations that is beyond reach as concrete and knowable object.}
  \item From this fundamentally separate foundation of the Other's being, we are able to experience co-incident phenomenae which are valid and possible events in our experience, i.e. \textcquote[316]{sartre}{the Other's feelings, the Other's ideas, the other's acts of will, the Other's character.}
  \item \textcquote[316]{sartre}{That is because the Other is not, in point of fact, the only one whom I see \textbf{but the one \emph{who sees me.}} I aim at the Other insofar as he is a connected system of experiences out of reach, within which I figure as one object among others.}
  \begin{enumerate}
    \item \textcquote[316]{sartre}{The Other, on contrary, is presented as in some sense the \emph{radical negation of my experience,} \textbf{since he is the one for whom I am not a subject but an object.}}
  \end{enumerate}
  \item \textcquote[319]{sartre}{At the origin of the problem of the Other's existence there is a fundamental mental presuppposition: that the Other is indeed the \emph{other}, which is to say that [it] \emph{is not} me; here, therefore, we are apprehending \textbf{a negation as a constitutive structure of Other-being.}}
\end{enumerate}

\noindent
Sartre ends this section \autocite[316 -- 321]{sartre} with further looks at how both Kantian idealism and realism are insufficient to give account for the Other, but the specifics are not that important. He than proceeds to examine the perspective of the other from the perspective of Husserl, Hegel, and Heidegger in the following section.

\subsection{Husserl, Hegel, Heidegger}

In this section, Sartre continues with the exploration of the other from the perspective of Husserl, Hegel, and then Heidegger-- and discusses both their strengths, as well as their inadequacies.

\subsubsection*{Husserl's Conception of the Other}

Husserl's phenomenology makes the other a neccessary part of the world's existence, where \textcquote[322]{sartre}{The Other is present within it not only as this concrete and empirical appearance but as a permanent condition of its unity and its richness.} However, this account for the existence of the Other fails to explain how is it possible for the self to know the Other. \textcquote[323]{sartre}{It is evident too that the meaning of \enquote{the Other} cannot come from experience, or from an argument from analogy made on the occasion of experience: rather, and quite to the contrary, it is in light of the concept of the \emph{Other} that experience is interpreted.} If there is a realm of experiences that are rendered objective by the other, we may always have access to the experiences themselves. But how can we have access to the other behind these experiences. Here, Sartre critiques the Husserl explanation as sharing the same flaws of Kant's idealism-- namely that the Other it speaks about has a transcendental, unreachable side. \textcquote[325]{sartre}{\ldots\ Husserl was able to establish between my being and the Other's is that of \emph{knowledge}; [however] he cannot therefore, any ore than Kant could, escape Solipsism.}

\subsubsection*{Hegel's Conception of the Other}

The Hegelian conception of the Other \autocite[325 -- 328]{sartre} is a richer, more elaborate development, where the Other is seem as a self-recognition of consciousness. Hegel's Lord and Bondsman relationship is used to explain aspects of both the self-recognition, as well as of the difference between the self and other.

However, ultimately Hegel's conception also proved to be lacking, as Sartre explains starting from \autocite[329]{sartre}. I'm not sure if I follow his reasoning in the entirety, but the essense seems to be how the Lord-and-Bondsman relationship requires the consciousness to hold itself in objectivity. This is ontologically impossible, since the for-itself cannot conceive of itself as an in-itself \autocite[334]{sartre}.

\subsubsection*{Heidegger's Conception of the Other}

What is Heidegger's conception of the Other? It is perhaps the most complete one yet. I think the thing that is the most persuasive to me, is the idea that the relationship between the self and the other cannot be mediated in any case as relationships of knowledge and being, or being and knowledge -- but it is only possible on the condition of being and being.

\begin{enumerate}
  \item \textcquote[336]{sartre}{What have we gained from this lengthy critique? Simply this: if it must be possible to refute solipsism, my relation to the Other is, first and fundamentally, a relation of being to being, and not of knowledge to knowledge.}
  \begin{enumerate}
    \item \enquote{Indeed, we have seen the failure of Husserl, who, on this particular front, measures being by knowledge, and the failure of Hegel, who identifies knowledge and being.}
  \end{enumerate}
  \item \textcquote[336]{sartre}{It seems that Heidegger, in \emph{Being and Time}, has profited from the meditations of his precursors and has profoundly absorbed this twofold necessity:}
  \begin{enumerate}
    \item \enquote{The relation of \enquote{human-realities} must be a relation of being;}
    \item \enquote{this relation must make \enquote{human-realities} depend on each other in their essential being.}
  \end{enumerate}
  \item \textcquote[338]{sartre}{Our [the self and the Other] relation is not a \emph{frontal} opposition; rather, it is an interdependence \emph{alongside:} insofar as I make it the case that a world exists as a structure of equipment, of which I make use for the purpose of my human-reality, I come to be determined in my being by a being who makes it the case that the same world exists as a structure of equipment for the purpose of his reality.}
  \item \textcquote[338]{sartre}{What I make myself to be is, therefore, a mode of being. And this truth means that I am responsible for my being for the \emph{Other} insofar as I actualise it freely, in authenticity or inauthenticity.}
  \marginnote{This section of Sartre's discussion on Heidegger's conception of the Other touches upon the themes of authenticity, in-authenticity, individuality, and the pursuit of such. I find this section to be of immense personal interest to me, especially the idea of acting ever more authentically.}
  \item \textcquote[339]{sartre}{In particular, when I am in the mode of inauthenticity -- of the \enquote{they} -- the world sends back to me something like an impersonal reflection of my inauthentic possibilities in the guise of implements and structures of equipment that belongs to \enquote{everyone,} and which belongs to me insofar as I am \enquote{everyone:} ready-made clothes, public transport, parks, gardens, public spaces, shelters that are built so that \emph{anyone} can shelter there, etc. In this way I become acquainted with myself as \emph{anyone} by the referential structure of equipment, which refers to me as a \enquote{\emph{Worumwillen},}\footnote{Heidegger's translators render the phrase \emph{Worumwillen} as \enquote{for-the-sake-of-which.}}  but as a complete interchangeability in terms of the relation}
  \marginnote{What is the specific nature of this common inauthenticity called the \enquote{they}, and how do we manifest it in our everyday lives or as a member of an impersonal society?}
  \item \textcquote[339]{sartre}{\textbf{Authenticity and individuality have to be won:} I will be my own authenticity \emph{only if,} under the influence of the call of conscience (\emph{Ruf des Gewissens}) I throw myself toward death, as toward my ownmost possibility, with resolute-decision (\emph{Entschlossenheit}). At that moment I am disclosed to myself in authenticity, and I raise others, too, along with me, in the direction of authenticity.}
  \marginnote{I know that Sartre is quoting Heidegger here, but what does he mean by the call of conscience, and by the throwing towards death?}
\end{enumerate}

Sartre's presentation turns towards a critique starting at \autocite[340]{sartre}. For Heidegger to claim that the foundation of the Other's existence is in ourselves through the process of our own \emph{being-with} (e.g. the analogy of the team), it is still absolutely neccesary for us to explain the nature of the being-with, which he does not do satisfactorily.

\begin{enumerate}
  \item \textcquote[341]{sartre}{What would need to be shown, in fact, is that \enquote{being-with-Pierre} or \enquote{being-with-Anny} is a constitutive structure of my concrete-being. But that is impossible, from the standpoint that Heidegger has taken up.}
  \item \textcquote[342]{sartre}{My \emph{being-with}, apprehended on the basis of \enquote{my} being, can be regarded only as a pure demand, founded in my being, and which does not constitute the least proof of the Other's existence, the least bridge between me and the other.}
  \item \textcquote[344]{sartre}{The nature of the Other's existence is that of a contingent and a irreducible fact. We \emph{encounter} the other; we do not constitute him.}
\end{enumerate}

\subsection{The Look}

\noindent
After examining the prior three accounts for the Other, as well as sharing their merits as well as flaws, Sartre begins his exploration of the Other from his own account. Sartre titles this section \enquote{The Look,} \autocite[347]{sartre} and this is because understanding the look between the self and the Other, as well as the Other and the self, is important to understanding the ontology of the Other. From my understanding, the Other cannot be an \emph{object} of the being-for-itself, but in the image of the drain or whirlpool, it is a locus of being which drains the objectivity of the being-for-itself just like how a massive body warps space and time.

\noindent
In fact, the definitive nature of the Other's being seems to lie in the fact that it is capable of looking -- of both looking towards objects of it's own experience, but also of looking to the being-for-itself:

\begin{enumerate}
  \item \textcquote[352]{sartre}{What is the original presence of the Other to which it refers? \ldots\ If the object-Other is defined in connection with the world, as the object who sees what I see, it must be possible to sum up my fundamental connection with the subject-Other through the constant possibility of my \emph{being seen} by the Other}
  \item \textcquote[353]{sartre}{\enquote{Being-seen-by-the-Other} is the \emph{truth} of \enquote{seeing-the-Other.}}
  \item \textcquote[353]{sartre}{The description of this fundamental connection tht must provide the basis of any theory of the Other; if the Other is, as a matter of principle, \emph{the one who looks at me}, we ought to be able to explicate the meaning of the Other's look}
\end{enumerate}

\noindent
Starting at \autocite[354]{sartre} Sartre begins his exposition on the Look.

\begin{enumerate}
  \item \textcquote[355]{sartre}{\ldots\ Thus the look is in the first place an intermediary by which I am referred to myself. What is the nature of this intermediary? What is the meaning for me of being seen?}
  \item I am too tired to complete my notes for this section, but I highlighted some important quotations around \autocite[362 -- 366]{sartre}. I should make sure to revisit this section, if only because Sartre presents an interesting image using the watchman (guard) and the robber.
  \marginnote{I should make sure to revisit Sartre's story of the nightwatchman and the robber, since it seems to present an interesting duet of the Other and the one looked upon.}
\end{enumerate}

\subsubsection{On the Other's Look, Objectively}

\noindent
A new day's worth of notes: In the previous section, Sartre talks about the Other's look from the perspective of the \emph{cogito} of the being-for-itself (i.e. the person being looked at by the other). In this subsequent section starting from \autocite[368 -- 375]{sartre}, Sartre examines the Other's look from the perspective of \emph{objectivity}. As I am much better rested now, I have more quotes to present:

\begin{enumerate}
  \item \textcquote[368]{sartre}{In the first place, the \emph{Other's look}, as the necessary condition of my objectivity, is the destruction of all objectivity for me. The Other's look reaches me through the world and is not only a transformation of myself but a complete metamorphosis of the \emph{world}.}
  \item \textcquote[369]{sartre}{But, in addition, by freezing my possibilities, the Other reveals to me that it is impossible for me to be an object unless it is for another freedom. I cannot be an object for myself \ldots\ for how could I be an object, other than for a [looking] subject?}
  \item \textcquote[369]{sartre}{By the same token [as the Other's look freezes my freedom (of possibilies) into the objecthood of (probabilities)], I experience his infinite freedom. That is because it is for and through a [the Other's] freedom -- and only for and through it -- that my possibilities can be limited and frozen.}
  \item \textcquote[370]{sartre}{Thus, through his look, I experience the Other concretely as a free and conscious subject who makes it the case that there is a world by temporalising himself towards his own possibilities.}
  \item \textcquote[371]{sartre}{The presence to me of the Other-look is not, therefore, an item of knowledge; neither it is a projection of my being, or a form of unification or category. It \emph{is}, and I am unable to derive it from myself.}
\end{enumerate}

\noindent
Essentially, it sounds like just as my own look (as the being-for-itself) is one of freedom that objectifies the world, the Other's look is it's own freedom which both objectifies and limits myself. Starting from \autocite[372]{sartre}, Sartre begins to qualify the above observations with some caveats: \textquote{But here we must also note that: \ldots}

\begin{enumerate}
  \item \textcquote[372]{sartre}{My objecthood for me [stemming from being looked upon by the Other] is not in any way an explication of Hegel's \enquote{\emph{Ich bin Ich.}} There is no question here of a formal identity, and \emph{my object-being or being-for-the-Other is radically different from my being-for-myself.}}
  \begin{enumerate}
    \item This is an important caveat to keep in mind, for it would be inconsistent if our \emph{being-for-itself} is affected by the look of another \emph{being-for-itself}.
  \end{enumerate}
  \item \textcquote[374]{sartre}{Moreover, the Other does not constitute me as an object for myself but \emph{for him}. In other words, he is not used as a regulative or constitutive concept of instances of \emph{knowledge} that I can have of myself.}
  \item \textcquote[375]{sartre}{Thus my me-object [i.e. the object-ness of myself when I am Looked upon by the Other] is neither knowledge nor a unity of knowledge but unease, a lived separation from the for-itself's ecstatic unity, \ldots\ The fact of the Other is incontestable and reaches right to my heart. I actualise it through my \emph{unease}.}
\end{enumerate}

\subsubsection{On False Looks}

\noindent
Having completed the exploration of the effects of the look on our being-for-itself and it's relationship to objectivity, Sartre now begins a new section on a different line of thought \autocite[375 -- 376]{sartre}. This exploration is sort of addressing the question of false or mistaken looks, i.e. if I encounter a puppet or a painting that has eyes and appears to be another man at first, do I still go through all those complex ontological motions? Essentially, the question that is posed is: \emph{How does this account of the Other's look remain true and valid, during cases where the self believes there is an Other, but in fact there is none?}

\textcquote[376]{sartre}{I grasp myself a certain \enquote{being-looked-at} with its own structures [all of which we mentioned above], directing me to the Other's real existence. But it is possible that I am mistaken: \ldots\ perhaps these concrete objects were not \emph{really} manifesting a look. In that case, what becomes of my certainty that I am \emph{being looked at?} My shame was in effect \emph{shame before somebody}: but there is nobody there. Does it not thereby become \emph{shame before nobody} \ldots\ a \emph{false} shame?}

Sartre does not find this to be a difficult objection, and after reading with him, I agree. Essentially, the nature of the relationship between the self and the Other is an ontological relationship, on a level which is above, and transcendent of the factual level. The image that he presents is that the motion of \emph{shame}, where we believe we are seen -- is still as real and as visceral as when we think we are seen, even if we realise it is a false alarm. His quotes best explain his conception on the difference between the factual presence of a looking-Other, versus the ontological level of the Other which is present in our very \emph{human-reality}:

\begin{enumerate}
  \item \textcquote[376]{sartre}{Being-looked-at cannot therefore \emph{depend} on the object which manifests the look \ldots\ If therefore, \emph{being-looked-at} considered in its purest form, is not linked \emph{to the Other's body} any more than my consciousness of being a consciousness \ldots}
  \item \textcquote[377]{sartre}{In brief, what therefore was it that misleadingly appeared and was then destroyed during the false alert [of being looked at]? It is not the subject-Other, or his presence to me: it is the Other's \emph{facticity}, i.e. the contingent connection between the Other and an object-being within my world.}
  \marginnote{And surely, this connection between the Other and an object-being is \emph{truly} a contingent one. Just think about the technologies of tele-presence -- surveillance cameras, livestreams, et cetera.}
\end{enumerate}

\subsubsection{On Absence}

\noindent
Starting from here \autocite[378]{sartre}, Sartre addresses a slightly different topic -- the idea of \emph{absence} (opposite of \emph{presence}), which has been implicit in the whole getting-caught-by-the-Other discussion. After all, if we get started and look up -- believing that we were caught in the Look of the Other -- but only to see that there is nobody, a false alarm, it means that there is an \emph{absence} of the other's \emph{object-being}.

\begin{enumerate}
  \item \textcquote[376 -- 377]{sartre}{[Absence] is a connection of being between human-realities and not between human-reality and the world. It is through his relation to Thérèse that Pierre is absent \emph{from this location.}}
  \item \textcquote[380]{sartre}{Being-for-the-Other is a constant fact of my human reality and I grasp it, with its \emph{factual} necessity, in the slightest thought that I form about myself.}
  \item \textcquote[381]{sartre}{To move away, to come closer, or to discover this particular Other-object is only to enact empirical variations on the fundamental theme of my being-for-the-Other.}
  \item \textcquote[381]{sartre}{I may well believe that a man is watching me in the shadows, and discover that it is a tree trunk which I took to be a human being: my fundamental presence to all men, the presence to me of all men, is not thereby altered, because the appearing of some man as an object within the field of my experience is not what teaches me that there are men. \textbf{My certainty of the Other's existence is independent of these experiences, and it is the former, on the contrary, that makes them possible.}}
\end{enumerate}

\noindent
In summary, absence is a quality that is only possible between \textquote{two or more human-realities, which necessiate the fundamental presence of each of these realities for the others.} What follows now is a good and succinct summary of the Look and the Other, offered by Sartre in \autocite[382]{sartre}:

\textcquote[382]{sartre}{We are now able to grasp the nature of the look: in every look an object-Other appears within my perceptual field as a concrete and probable presence and, on the occasion of various attitudes of this Other, I determine myself to grasp -- through shame, anguish, etc. -- my \enquote{being-looked-at.} This \enquote{being-looked-at} is presented as the pure probability that I am presently the concrete \emph{this} -- a probability that can derive its meaning, and its very nature as probable, only from a fundamental certainty that the Other is always present to me insofar as I am always present \emph{for the Other}. The experience of my condition as a man as an object for \emph{all} other living men, thrown into the arena before millions of eyes -- and escaping millions of times from myself -- is concretely actualised by me on the occasion of an object's arising within \emph{my} universe, if this object indicates to me that I am probably an object now as a \emph{differentiated this} for some consciousness. That is the \emph{totality of the phenomenon that we are calling \enquote{the Look.}} Each look makes us concretely experience \ldots\ that we all exist for all living men.}

\noindent
Finally, in some concluding remarks, Sartre talks about shame, fear, and pride in pages \autocite[392 -- 395]{sartre}. He best sums it up by saying: \textcquote[395]{sartre}{Shame, fear, and pride are therefore my original reactions, they are merely the various ways in which I recognise the Other as a subject out of reach, and they include within themselves an understanding of my ipseity which can and does provide me with the motivation to constitute the Other into an object.}

\section{Chapter 2: The Body}
\marginnote{Starting from here, my notes are not as exhaustive. I should make sure to revisit these sections to make sure I understand things as completely as I can.}
From now on I will only try to take more high-level and over-arching notes, and refrain from providing such an exhaustive level of quotations. In this chapter, Sartre presents the three ontological dimensions of the body.

\subsection{The body as Being-For-Itself: Facticity}

This is the first ontological dimension of the body. I'm not sure if I understood this section \autocite[409 -- 453]{sartre} of Sartre's work as well as I could, but it seems like at least for the first part, Sartre is trying to refute the mechanistic and reductionist idea of the body as a being of pure physicality, or biology. My body cannot be reduced to experiences or sensations, but rather it is simply facticity. Now I admit I do not understand this process completely, but it seems persuasive.

\subsection{The Body-for-the-Other}

Section \autocite[453 -- 468]{sartre}. This is the second ontological dimension of the body, which is the body of the self for the Other.

\subsection{The Third Ontological Dimension of the Body}

Section \autocite[468 -- 478]{sartre}. This is the third ontological dimension: the body as an object for the Other, which I perceive. Or to slightly rephrase it-- the third ontological dimension is my perception of the being of my body, insofar as it is an object for the Other.

\section{Chapter 3: Concrete Relations with the Other}

We are given an introduction to the \emph{concrete} relationships between the for-itself towards the Other, in \autocite[479 -- 482]{sartre}. By concrete relationships, Sartre means in opposition to the previous chapter and it's three \emph{ontological} relationships. These concrete relationships stem from the fact that we percieve the ontological relationships of the for-itself through the body -- although Sartre takes care to caution us not to think of the Body as an instrumental or efficient means: \textcquote[479]{sartre}{It is not that the body is the instrument and cause of my relations with the Other. But it constitutes their meaning, and it marks out their meanings: it is as a body-in-situation that I apprehend the Other's transcended-transdencence and it is as a body-in-situation that I experience myself, as alienated for the Other's benefit.}

\begin{enumerate}
  \item \textcquote[481]{sartre}{These are the two basic attitudes that I take up in relation to the Other:}
  \begin{enumerate}
    \item \textquote{Transcending the Other's transcendence or, on the contrary,}
    \item \textquote{Swallowing up this [the Other's] transcendence within me without taking away its character of transcendence.}
  \end{enumerate}
\end{enumerate}

\noindent
These two different modalities of being are discussed in the subsequent parts of this chapter:

\subsection{Our First Attitude Toward the Other: Love, Language, Masochism}

The layout of this section is roughly as follows. First there's an introductory segment where we talk about the first attitude to the other on more abstract, ontological grounds \autocite[482 -- 485]{sartre}, which is then subsequently followed by it's four more concrete manifestations. The first one is on love \autocite[485 -- 493]{sartre}, and the second one is on seduction \autocite[492 -- 493]{sartre} (which is closely related to language), the third one is on language \autocite[493 -- 499]{sartre}, and finally the third is on masochism \autocite[499 -- 501]{sartre}.

\begin{enumerate}
  \item \textcquote[483]{sartre}{The relations we are concerned with here are not at all unilateral relations to an object-in-itself but reciprocal and shifting relationships. The following descriptions should therefore be envisaged within the perspective of \emph{conflict}. \emph{Conflict is the original meaning of being-for-the-Other}.}
  \begin{enumerate}
    \item \textcquote[483]{sartre}{If we set out from the Other's initial relevation as the \emph{Look}, we are obliged to acknowledge that we experience our elusive being-for-the-Other in the form of an [object's] \emph{possession}.}
    \item \textcquote[483]{sartre}{Thus I have some understanding of [my being-for-the-Other's] ontological structure; I am responsible for my being-for-the-Other but I am \emph{not} its foundation; it appears to me, therefore, in the form of a contingent datum for which I am nonetheless responsible, and the Other founds my being insofar as this being takes the form of the \enquote{there is.}}
    \item \textcquote[483]{sartre}{[Hence] my object-being [in the look of the Other] is \emph{intolerably} contingent and a pure \enquote{possession} of me by another.}
  \end{enumerate}
  \item This objecthood which our being experiences as we are regarded by the Other's look, in our condition of the being-for-the-Other is \enquote{intolerably contingent}, as per Sartre's own words. It is an endangerment of our being, where our freedom in the being-for-itself is reduced in the other's eyes as a simple, \emph{objective} being-in-itself. This ontological \emph{endangerment} is the foundation of Sartre's conflict, where there are only two possible modalities of resolution:
  \begin{enumerate}
    \item \textcquote[483]{sartre}{[this objecthood in the Other's look is what] I have to retrieve and to found, in order to be the foundation of myself. But this is conceivable \emph{only if I can assimilate the Other's freedom.} In this way, my project to reclaim myself is \textbf{fundamentally a project to reabsorb the Other.}}
  \end{enumerate}
  \item What does it mean for us to embark upon this project, or to \enquote{reabsorb the Other?} Sartre explains this under two qualifications:
  \begin{enumerate}
    \item \textcquote[483]{sartre}{In order to do this, I do not stop affirming the Other, i.e., denying, in relation to myself, that I am the other: as the foundation of my being, the other cannot become so diluted in me without my being-for-the-Other vanishing.}
    \item \textcquote[484]{sartre}{The other whom I want to assimilate is not at all the object-other \ldots\ My concern is not to erase my objectivity by objectifying the other -- which would amount to my \emph{delivering myself} from my being-for-the-Other -- but, quite to the contrary, \emph{it is as an other-who-looks} that I wish to assimilate the other.}
  \end{enumerate}
  \item In more practical terms, when we \enquote{assimilate the Other's freedom}, we are neither denying the existence of the Other, nor are we objectifying the Other. We need to preserve the other's ontological character as a \emph{looking-Other}, or in Sartre's words, a \enquote{other-who-looks.}
  \item However, this project to reabsorb the other is ultimately impossible. \textcquote[484]{sartre}{This ideal can be achieved only if I overcome the original contingency of my relations to the Other, i.e. the fact of there being no relationship of internal negativity between the negation through which the Other establishes himself as other from me, and the negation through which I establish myself as other from the other}.
  \begin{enumerate}
    \item Recall how in the previous sections on the ontology of the other, we realise there has to no relationship between the self and the Other, in order to preserve the Other's alterity and distinct being? This means that we cannot actually reabsorb the other, since there's no relation of \enquote{internal negativity,} as Sartre explained above.
    \item But just because something is impossible (and indeed, this impossibility is both \emph{de facto} and \emph{de jure}, as Sartre states in \autocite[485]{sartre}) -- doesn't mean we can't \emph{attempt} this project.
  \end{enumerate}
  \item How does the attempt of this unrealisable project look like in practice? In terms of \emph{concrete relations with the Other}? Well \ldots
  \item \textcquote[485]{sartre}{This unrealisable ideal, insofar as it haunts my project for myself in the presence of the Other, cannot be equated with love, insofar as love is an enterprise, i.e., an organic set of projects concerning my own possibilities. But \emph{it is} love's ideal, its motivation and its goal, the value that belongs to it.}
  \item \textcquote[485]{sartre}{\textbf{Love as a basic relation to the Other is a set of projects through which I aim to realise this value.}}
\end{enumerate}

\subsubsection{On Love}

Here \autocite[485 -- 493]{sartre}, Sartre examines love as a \enquote{basic relation to the Other.} So what is love, under Sartre's existentialist conception? This is a very rough summary, but my understanding is as follows:

\noindent
The primary relationship between the being-for-itself and the \emph{Other} (i.e. another being-for-itself that is not actually me) is a relationship of the \emph{Look}. Although Sartre doesn't quite use this particular phrase, I would say that this relationship is first and foremost, a relationship of \emph{fear}. That's because the Other is a pure negatity, someone from whom \enquote{meaning seems to drain away to.} Our selves experience an ontological shock, knowing that our being is not self-supporting -- but rather we are the \emph{object} of an Other from whom is fundamentally unknowable to us. 


\begin{enumerate}
  \item \textcquote[485]{sartre}{These projects [of love] place me in a direct relation with the Other's freedom. It is in this sense that \emph{love is conflict.}}
  \item \textcquote[485]{sartre}{We have noted that the Other's freedom is, in effect, the foundation of my being. But precisely because I exist [objectively] through the Other's freedom, I lack all security, and I am in danger in that freedom.}
  \item \textcquote[485]{sartre}{My project to reclaim my being can be fulfilledo nly if I take hold of that freedom and reduce it to being a freedom that submits to my freedom.}
\end{enumerate}

\noindent
So what does it mean for a lover to \enquote{take hold of [the beloved] freedom and reduce it t obeing a freedom that submits to my freedom?} Sartre proceeds to elaborate:

\begin{enumerate}
  \item \textcquote[486]{sartre}{Why does the lover want to be \emph{loved?} If in fact love were merely the desire for physical possession, it could, in many cases, be easily satisfied. [But this is not the case] \ldots\ It is certain therefore that love wants to captivate \enquote{consciousness.} But why does it want this? And how?}
  \item \textcquote[485]{sartre}{Why should I want to appropriate the Other, if not precisely because the Other makes me be?}
  \item \textcquote[485]{sartre}{But this implies, precisely, \emph{a specific mode of appropriation:} it is the other's freedom as such that we want to seize.}
\end{enumerate}

\noindent
There's no point in taking further notes from this section, because it's all good. It's all so expertly written. Instead, I recommend to just re-read \autocite[485 -- 493]{sartre} in it's entirety. Philosophy is beautiful!

\subsubsection{On Seduction}

Pages \autocite[492 -- 493]{sartre}.

\begin{enumerate}
  \item \textcquote[492]{sartre}{Through seduction I aim to constitute myself as a fullness of being and to make myself \emph{recognised} as such. To that end, I constitute myself as a signifying object. My actions must \emph{indicate} in two directions}
  \begin{enumerate}
    \item \textcquote[492]{sartre}{On one hand, they point toward a depth of objective, hidden being, which we refer to incorrectly as \enquote{subjectivity}; the action is not merely done for its own sake but indiates an infinite and undifferentiated series of other real and possible actions that I offer as constituting my objective and unperceived being. In this way I try to guide the transcendence that is transcending me, to refer it toward the infinity of my dead-possibilities, precsely in order to b e unsurpassable -- and to the extent to whih the only thing that cannot be surpassed is, precisely, the infinite.}
    \item \textcquote[492]{sartre}{On the other hand, each one of my actions attempt to indicate the world in the greatest possible breadth, and to present me as being linked to the world's vastest regions:}
    \begin{enumerate}
      \item \textquote{Whether I \emph{present} the world to my loved one and try to constitute myself as the necessary intermediary between her and the world,}
      \item \textquote{Or whether I simply demonstrate, through my ations, my endlessly varied powers over the world (money, power, connections, etc.)}
    \end{enumerate}
  \end{enumerate}
  \item \textcquote[493]{sartre}{Through these various methods I \emph{propose} myself as unsurpassable.}
\end{enumerate}

\subsubsection{On Language}

Pages \autocite[493 -- 499]{sartre}.

\begin{enumerate}
  \item To be completed later.
\end{enumerate}

\subsubsection{On Masochism}

Pages \autocite[499 -- 501]{sartre}.

\begin{enumerate}
  \item To be completed later.
\end{enumerate}

\subsection{Our Second Attitude Toward the Other: \\ Indifference, Desire, Hatred, and Sadism}

\begin{enumerate}
  \item To be completed later.
\end{enumerate}

\subsubsection{On Indifference}

Pages \autocite[503 -- 505]{sartre}.

\begin{enumerate}
  \item To be completed later.
\end{enumerate}

\subsubsection{On Sexuality}

Pages \autocite[505 -- 508]{sartre}.

\begin{enumerate}
  \item To be completed later.
\end{enumerate}

\subsubsection{On Desire}

Pages \autocite[508 -- 526]{sartre}.

\begin{enumerate}
  \item To be completed later.
\end{enumerate}

\subsubsection{On Sadism}

Pages \autocite[526 --]{sartre}.

\begin{enumerate}
  \item To be completed later.
\end{enumerate}

\subsection{\enquote{Being-With} (\emph{Mitsein}) and the \enquote{We}}

The section on Mitsein \autocite[543 -- 566]{sartre} forms the last section of the chapter on the \emph{Concrete Relations with the Other}. Sartre illustrates the unique ontological status of the \enquote{we-being} by a quick illustration: imagine you are in front of a cafe, and suddenly a minor traffic accident occurs on the street below. All of a sudden, you and your fellow cafe-goers are embroiled in an unexpected spectacle. Is there not an idea of \enquote{we} as the audience, and \enquote{them} as whomever we spectate? Sartre proceeds to unpack this \emph{mitsein} (lit. we-being), which was originally Heidegger's idea. He distinguishes two modalities of the \emph{mitsein}, one which he calls the \emph{we-object} \autocite[546 -- 557]{sartre}, and the other which he calls the \emph{we-subject} \autocite[557 -- 566]{sartre}.

\begin{enumerate}
  \item \textcquote[546]{sartre}{If the sentence \enquote{We are being looked at} is to refer to a real experience, it is necessary that in this experience I should feel myself committed with others in a community of transcended-transcendences of alienated \enquote{Mes.}}
\end{enumerate}

\subsubsection{The We-Object}
The We-Object is the form of the We when the for-itself is included within it. Essentially, that is when you and your compatriots are all captured in the objective glaze of an Other.

\subsubsection{The We-Subject}
In contrast, the we-subject is the for-itself identifying itself with it's compatriots. 

\noindent
In the end of this chapter, Sartre ultimately makes sure to caution us and to state that the fundamental modality of relations between consciousness is \emph{not} Mitsein, but ultimately one of conflict \autocite[564]{sartre}.
