\section{Part III: Being-For-The-Other}

\subsection{Chapter 1: The Other's Existence}

\subsubsection{The Problem}

This is the introductory section \autocite[307 -- 309]{sartre} which lays out Sartre's exposition for \enquote{the problem.} Sartre admits that in all of the preceding discussions of the being-for-itself (i.e. the conscious being) and the being-in-itself (i.e. objective beings), we have left out one important and concrete mode of ontology. This mode of ontology is of the being-for-the-other -- which from my understanding so far, is the mode of being that a being-for-itself takes, when it is contemplated by another being-for-itself. The example that Sartre uses to best highlight the distinct ontological nature of a \emph{being-for-the-other} is through the emotion of \emph{shame.} Sartre uses shame as the emotion to portray this distinction because shame is unique in how it contains a component of self-recognition. It also neccesiates an \emph{Other}, for it is impossible to feel self by oneself. \textcquote[308]{sartre}{shame in its primary structure is \emph{shame before somebody.}}

\begin{enumerate}
  \item \textcquote[308]{sartre}{Shame is \emph{recognition.} I recognise that \emph{I am} as the Other sees me.}
  \item \textcquote[308]{sartre}{Thus the Other has not only shown me what I was; he has constituted me in a new type of being, obliging me to support new qualifications.}
  \item \textcquote[309]{sartre}{But this new being that appears \emph{for} the Other does not reside \emph{in} the Other; I am responsible for it \ldots\ In this way shame is shame \emph{of myself before the Other}; these two structures are inseparable. But, by the same token, I need the Other in order to fully grasp all the structures of my being; my for-itself refers to my for-the-other.}
\end{enumerate}

\subsubsection{The Reef of Solipsism}

Sartre begins his exploration on the Other's existence \autocite[309 -- 322]{sartre} through an examination of the inadequacy of both \emph{realism} and \emph{idealism} to account for the existence of the Other. We are confronted with the question of \enquote{How can there be an Other?} Right now, in this pre-Other ontology of ours, there are only two kinds of beings -- beings-for-itself (the conscious being) which is the self -- and beings-in-itself, objects of our positional consciousness \autocite[309]{sartre}.

\paragraph{Inadequacies of \emph{Realism} to justify the existence of the \emph{Other}}

We can attempt to furnish an explanation for the existence of the Other through \emph{Realism.} We observe that there is another body, that can be the subject of our intuition and consciousness. We observe the empirical phenomena associated with the other's body -- and that there are patterns that can be followed. We ultimately come to the conclusion that that the other body's behaviour can only be explained by the existence of another positional consciousness (a being-for-itself) other than our own. \textcquote[311]{sartre}{The hypothesis that best accounts for [the Other body's] ways of behaving is that of a consciousness that is analogous to mine, whose various emotions it reflects.}

However, this hypothesis -- while plausible, can never be ontologically certain -- the realist will always have other equally plausible explanations of the Other as only a body -- a mechanistic being, a perspective of pure behavioralism. \textcquote[311]{sartre}{if the Other is accessible to us only through the knowledge that we have of him, and if this knowledge is only conjectural, the Other's existence is only conjectural.}  In fact, Sartre goes on to critique the inadequacies of the realist explanation for the existence of the Other as downright a recourse to idealism. After all, if the existence of the other is conjecture and hypothesis -- it is no different from saying that it has no real foundation, but is merely ideal \autocite[312]{sartre}.

\paragraph{Inadequacies of \emph{Idealism} to justify the existence of the \emph{Other}}

Moving on, Sartre then looks at explanations of the \emph{Other} from the perspective of \emph{Idealism} \autocite[312]{sartre}. He takes an early look at Immanuel Kant's perspective of the transcendental metaphysics -- which presents the other is an object of our experience. This makes the Other entirely a phenomenon. However, there still exists the difficulty of explaining the \emph{being} that's behind the Other. It's hard to not see the Other as \emph{noumenal} -- which would make it a thing that's entirely beyond our intuition altogether.

In this line of reasoning, Sartre proceeds to examine other \autocite[315]{sartre} potential explanations for the foundation of the Other's being, such as the Other as a \enquote{regulative concept} or \enquote{organising principle.} All are weak and inadequate:

\textcquote[315]{sartre}{Consequently, the Other is not, within my experience, a phenomenon that points toward my experience [i.e. any idealist perception] but rather one that refers in principle to phenomena that are situated outside any possible experience for me \ldots\ the concept of the Other is not purely instrumental [i.e. any organising principle]: rather than its existing \emph{for} use in the unification of phenomena, we ought on the contrary to say that various categories of phenomena seem only to exist \emph{for} it.}

\paragraph{Where is the Other?}

It's that final line -- where \enquote{various categories of phenomena seem only to exist \emph{for} it} -- that hints towards the ultimate ontological foundation of the Other:

\begin{enumerate}
  \item \textcquote[316]{sartre}{That \emph{Other}, whose relation to myself cannot be apprehended, and who is never given, is gradually constituted by us as a concrete object:
  \marginnote{Do we once again see the familiar them of seperation, or negation, as the root behind a being's ontology?}
  he is not the instrument of which I make use to predict an event within my experience; rather, it is the events in my experience that used to constitute the Other as Other, i.e. as a system of representations that is beyond reach as concrete and knowable object.}
  \item From this fundamentally separate foundation of the Other's being, we are able to experience co-incident phenomenae which are valid and possible events in our experience, i.e. \textcquote[316]{sartre}{the Other's feelings, the Other's ideas, the other's acts of will, the Other's character.}
  \item \textcquote[316]{sartre}{That is because the Other is not, in point of fact, the only one whom I see \textbf{but the one \emph{who sees me.}} I aim at the Other insofar as he is a connected system of experiences out of reach, within which I figure as one object among others.}
  \begin{enumerate}
    \item \textcquote[316]{sartre}{The Other, on contrary, is presented as in some sense the \emph{radical negation of my experience,} \textbf{since he is the one for whom I am not a subject but an object.}}
  \end{enumerate}
  \item \textcquote[319]{sartre}{At the origin of the problem of the Other's existence there is a fundamental mental presuppposition: that the Other is indeed the \emph{other}, which is to say that [it] \emph{is not} me; here, therefore, we are apprehending \textbf{a negation as a constitutive struce of Other-being.}}
\end{enumerate}

\noindent
Sartre ends this section \autocite[316 -- 321]{sartre} with further looks at how both Kantian idealism and realism are insufficient to give account for the Other, but the specifics are not that important. He than proceeds to examine the perspective of the other from the perspective of Husserl, Hegel, and Heidegger in the following section.

\subsubsection{Husserl Hegel, Heidegger}

\begin{enumerate}
  \item To be completed later.
\end{enumerate}

\subsubsection{The Look}

\begin{enumerate}
  \item To be completed later.
\end{enumerate}

\subsection{Chapter 2: The Body}

\subsubsection{The body as Being-For-Itself: Facticity}

\begin{enumerate}
  \item To be completed later.
\end{enumerate}

\subsubsection{The Body-for-the-Other}

\begin{enumerate}
  \item To be completed later.
\end{enumerate}

\subsubsection{The Third Ontological Dimension of the Body}

\begin{enumerate}
  \item To be completed later.
\end{enumerate}

\subsection{Chapter 3: Concrete Relations with the Other}

\subsubsection{Our First Attitude Toward the Other: Love, Language, Masochism}

\begin{enumerate}
  \item To be completed later.
\end{enumerate}

\subsubsection{Our Second Attitude Toward the Other: Indifference, Desire, Hatred, and Sadism}

\begin{enumerate}
  \item To be completed later.
\end{enumerate}

\subsubsection{\enquote{Being-With} (\emph{Mitsen}) and the \enquote{We}}

\begin{enumerate}
  \item To be completed later.
\end{enumerate}
