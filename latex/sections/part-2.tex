\section{Part II: Being-For-Itself}

\subsection{Chapter 1: The Immediate Structures of the For-Itself}

\subsubsection{Self-Presence}

\begin{enumerate}
  \item \textbf{The In-Itself:} Something that has an infinite density of being, a plentitude. I think when Sartre talks about the (Being-)In-Itself, he is talking directly about the being of the existent (perhaps analogous to the being-of-the-phenomena?).
  \item "Identity is the limiting concept of unification \ldots\ at its extreme limit, unity vanishes and passes over into identity." \autocite[123]{sartre}
  \item "Consciousness is characterised, on the contrary, by its \emph{decompression of being.} Indeed, it is impossible to define it as self-coincident." \autocite[123]{sartre}
  \begin{enumerate}
    \item What this means is that the being of consciousness is not a being that's in-itself. The example that Sartre gives is that when we talk about my belief, I cannot say that my consciousness \emph{is} by belief. But rather only that "my belief is a consciousness (of) my belief." \autocite[123]{sartre}
    \item See how in this base, the being-of-consciousness is not infinitely dense?
  \end{enumerate}
  \item \textbf{Self-Presence as the foundation for self-consciousness:} I'm not too certain in my understanding of this right now, but Sartre elaborates on this in \autocite[126]{sartre}. The gist of it seems to be:
  \begin{enumerate}
    \item The being-in-itself is the regular being of existents, i.e. the being-of-the-phenomena.
    \item However, what is the being-for-itself? The very use of the word \emph{for} implies a strong reflective action.
    \item We cannot qualify the being-\emph{for}-itself using any regular conception of the being-in-itself.
    \item The key difference in the being-for-itself is that there's a separation which makes the being not its own coincidence, but still requires its own unity (bottom of \autocite[126]{sartre})
    \item Hence this self-presence must have some sort of separation, which will be shown to be \emph{nothing}.
    \item "The law of being of the for-itself as the ontological foundation of consciousness is to be itself in the form of self-presence." \autocite[126]{sartre}
  \end{enumerate}
  \item \textbf{Self-Presence as an act of separation from the self:} Sartre elaborates this at \autocite[127]{sartre}. Self-Presence is taken as something different, or apart from identity -- which is the dense plentitude of being, as we have explained above. In fact:
  \begin{enumerate}
    \item "The principle of identity is the negation of any type of relation within the being-in-itself"
    \item "On the contrary, self-presence presupposes that an intangible fissure has slipped inside being. If it is present to itself, that is because it is not completely itself. Presence is an immediate degradation of coincidence, because it presupposes separation."
    \item "But if we ask now \emph{what} separates the subject fro himself, we are forced to admit that it is \emph{nothing.}" \autocite[127]{sartre}
  \end{enumerate}
  \item So once again, we are back at the discovery that nothing is essential for being -- in this case, nothing is essential for self-presence.
\end{enumerate}

\subsubsection{The For-Itself's Facticity}

\begin{enumerate}
  \item I really don't understand this section as well as I'd like. I will have to revisit it later.
\end{enumerate}

\subsubsection{The For-Itself and the Being of Value}

\begin{enumerate}
  \item
\end{enumerate}

\subsubsection{The For-Itself and the Being of Possibles}

\begin{enumerate}
  \item
\end{enumerate}

\subsubsection{My Self and the Circuit of Ipseity}

\begin{enumerate}
  \item
\end{enumerate}

\subsection{Chapter 2: Temporality}

\subsubsection{Phenomenology of the Three Temporal Dimensions}

\begin{enumerate}
  \item
\end{enumerate}

\subsubsection{The Ontology of Temporality}

\begin{enumerate}
  \item
\end{enumerate}

\subsubsection{Original Temporality and Psychological Temporality: Reflection}

\begin{enumerate}
  \item
\end{enumerate}

\subsection{Chapter 3: Transcendence}

\subsubsection{Knowledge as a Type of Relation Between the For-Itself and the In-Itself}

\begin{enumerate}
  \item
\end{enumerate}

\subsubsection{On Determination as Negation}

\begin{enumerate}
  \item
\end{enumerate}

\subsubsection{Quality and Quantity, Potentiality and Equipmentality}

\begin{enumerate}
  \item
\end{enumerate}

\subsubsection{World-Time}

\begin{enumerate}
  \item
\end{enumerate}

\subsubsection{Knowledge}

\begin{enumerate}
  \item
\end{enumerate}
