\section{Part II: Being-For-Itself}

\subsection{Chapter 1: The Immediate Structures of the For-Itself}

\subsubsection{Self-Presence}

\begin{enumerate}
  \item \textbf{The In-Itself:} Something that has an infinite density of being, a plentitude. I think when Sartre talks about the (Being-)In-Itself, he is talking directly about the being of the existent (perhaps analogous to the being-of-the-phenomena?).
  \item "Identity is the limiting concept of unification \ldots\ at its extreme limit, unity vanishes and passes over into identity." \autocite[123]{sartre}
  \item "Consciousness is characterised, on the contrary, by its \emph{decompression of being.} Indeed, it is impossible to define it as self-coincident." \autocite[123]{sartre}
  \begin{enumerate}
    \item What this means is that the being of consciousness is not a being that's in-itself. The example that Sartre gives is that when we talk about my belief, I cannot say that my consciousness \emph{is} by belief. But rather only that "my belief is a consciousness (of) my belief." \autocite[123]{sartre}
    \item See how in this base, the being-of-consciousness is not infinitely dense?
  \end{enumerate}
  \item \textbf{Self-Presence as the foundation for self-consciousness:} I'm not too certain in my understanding of this right now, but Sartre elaborates on this in \autocite[126]{sartre}. The gist of it seems to be:
  \begin{enumerate}
    \item The being-in-itself is the regular being of existents, i.e. the being-of-the-phenomena.
    \item However, what is the being-for-itself? The very use of the word \emph{for} implies a strong reflective action.
    \item We cannot qualify the being-\emph{for}-itself using any regular conception of the being-in-itself.
    \item The key difference in the being-for-itself is that there's a separation which makes the being not its own coincidence, but still requires its own unity (bottom of \autocite[126]{sartre})
    \item Hence this self-presence must have some sort of separation, which will be shown to be \emph{nothing}.
    \item "The law of being of the for-itself as the ontological foundation of consciousness is to be itself in the form of self-presence." \autocite[126]{sartre}
  \end{enumerate}
  \item \textbf{Self-Presence as an act of separation from the self:} Sartre elaborates this at \autocite[127]{sartre}. Self-Presence is taken as something different, or apart from identity -- which is the dense plentitude of being, as we have explained above. In fact:
  \begin{enumerate}
    \item "The principle of identity is the negation of any type of relation within the being-in-itself"
    \item "On the contrary, self-presence presupposes that an intangible fissure has slipped inside being. If it is present to itself, that is because it is not completely itself. Presence is an immediate degradation of coincidence, because it presupposes separation."
    \item "But if we ask now \emph{what} separates the subject fro himself, we are forced to admit that it is \emph{nothing.}" \autocite[127]{sartre}
  \end{enumerate}
  \item So once again, we are back at the discovery that nothing is essential for being -- in this case, nothing is essential for self-presence.
\end{enumerate}

\subsubsection{The For-Itself's Facticity}

\begin{enumerate}
  \item It seems like this section Sartre ties down the being-for-itself into the being-in-itself. Essentially, there has to be some sort of foundation for the for-itself.
  \item "Thus the for-itself is supported by a constant contingency that it takes up, and assimilates, without ever being able to get rid of it. We may call this constantly evanescent contingency of the in-itself -- which haunts the for-itself and ties it to being in-itself without ever allowing itself to b e grasped -- the for-itself's \emph{facticity.}" \autocite[133]{sartre}
  \item It seems to me that this distinction -- the idea that the being-for-itself is founded upon a factual circumstance -- is important to avoid the illusion and absurdity of solipsism.
  \begin{enumerate}
    \item "The for-itsef, even while it chooses the \emph{meaning} of its situation and constitutes itself in situation as its own foundation, does \emph{does not choose} its position." \autocite[134]{sartre}
  \end{enumerate}
\end{enumerate}

\subsubsection{The For-Itself and the Being of Value}

\begin{enumerate}
  \item \textbf{The Lack:} The lack is the form of negation which "most deeply establishes an internal relation between what we negate and what our negation applies to \ldots\ [the lack is the form of negation] which penetrates most deeply into being -- the one that constitutes \emph{in its being} the being to which its negation applies with the being that it negates." \autocite[137]{sartre}
  \begin{enumerate}
    \item "The lack does not belong to the nature of the in-itself, which is entirely positive. It appears within the world only when human-reality arises." \autocite[138]{sartre}
    \item For example, given an unfinished circle -- it is technically an open curve that is complete in its being as an open curve. It is only through the realm of human-reality, specifically human desire, to which we give to it the lack -- the lack in which it is not a circle. \autocite[139]{sartre}
  \end{enumerate}
  \item Human reality itself must be a lack, because only through a lack can we derive lacks. Sartre talks about this very definitively in \autocite[139]{sartre}:
  \begin{enumerate}
    \item "A psychological state whose existence had the sufficiency of that curve [i.e. the unfinished circle] could not in addition make the slightest `call' for anything else: it would be itself, without any relation to anything other than itself"
    \item "In order to constitute it as a hunger or thirst, an external transcendence would be required."
    \item "No recourse to psychophysiological parallelism [i.e. the doctrine that the psychological is a direct parallel to the physiological] can enable us to escape these difficulties:" \autocite[139]{sartre}
    \begin{enumerate}
      \item Any physiological signs of a lack of water in an organism only posits a positive being of the state of the organism, refering to itself. Sartre presents this in vivid detail in \autocite[139]{sartre}
      \item "[Any] exact correspondence between the mental and the physiological [requires] that correspondance [to be] established only on the basis of an ontological identity" \autocite[139]{sartre}
    \end{enumerate}
  \end{enumerate}
  \item \textbf{Desire is a \emph{lack} of being:} "and is haunted in its [desire's] innermost being by the being that it desires [i.e. lacks]." \autocite[140]{sartre}
  \item \textbf{Lack is a trinity:} When we lack something, there are three components to the act of lacking \autocite[138]{sartre}:
  \begin{enumerate}
    \item \textbf{The \emph{manqué} (i.e. the lack):} The item that is missing
    \item \textbf{The incomplete existent:} That from which [the item] is missing [i.e. the existent].
    \item \textbf{The hypothetical whole:} A totality that is broken apart by the lack, and could which be restored by the synthesis of the missing item with the existent.
  \end{enumerate}
  \item In \autocite[139]{sartre} talks about how \textbf{value} comes from this lack. I need to investigate this further.

  \subsubsection*{The Being to Which Consciousness Aims For}
  \item Sartre presents a rather tricky, but essential understanding on \textbf{the emergence of value from lack and desire}. It's essential that we understand what value is, and where does it come from. Right now, take value in this case to mean ethical/personal value, i.e. what is important to us, or what we aim for. Sartre's presentation goes as follows:
  \begin{enumerate}
    \item Recall that lack is a trinity.
    \item Further recall that \textbf{the human-reality is a lack} (since otherwise, the being of the human condition would be positive, and there would be no such thing as lacking).
    \item Hence, \emph{if the human-reality is a lack, what are the components of the lack's trinity?} Sartre answers this in \autocite[140]{sartre}, where he states:
    \begin{enumerate}
      \item \textbf{The \emph{manqué} (i.e. the lack):} "The \emph{itself-as-being-in-itself}" \autocite[141]{sartre}
      \item \textbf{The incomplete existent:} "The element that plays the role of the existent is given to the \emph{cogito} as the immediacy of the \emph{desire.}" \autocite[140]{sartre}
      \item \textbf{The hypothetical whole:} \ldots
    \end{enumerate}
    \item From the incomplete trinity above, Sartre asks: what is this hypothetical whole from which the lack of the human-reality presupposes?
    \item It seems to me that this hypothetical whole is a transcendence towards a better whole, a better version of the being [i.e. self] (?)
    \item "This constantly absent being which haunts the for-itself is itself -- but frozen in the in-itself [i.e. as an object]." \autocite[142]{sartre}
    \begin{enumerate}
      \item My intepretation of this sentence is essentially thus: Our human-reality is defined by a negative thing, a lacking. But a lacking must presuppose first a thing that is lacking (e.g. the missing puzzle piece), which Sartre cals the manqué -- as well as the incomplete existent (e.g. the puzzle-hole) and the hypothetical whole (e.g. the complete puzzle.)
      \item The incomplete existent is manifest as desire.
      \item But the thing which we are lacking in our human-reality is another state of human-reality or being, which is the object of our consciousness. Our consciousness wishes to be something else, to be another consciousness -- which it is not.
      \item Hence the ultimate, hypothetical, and unachievable synthesis of what we lack from the lacking is where \emph{value comes from}.
    \end{enumerate}
    \item \textbf{Value: a transcendent thing which our current being lacks, which eludes our being.} \autocite[146]{sartre}
    \item "Value arrives to the world through human-reality" \autocite[147]{sartre}.
    \item "Value haunts being insofar as it founds itself and not insofar as it is: it haunts \emph{freedom}. So value's relation to the for-itself is quite distinctive: it is the being that the for-itself has to be, insofar as it is the foundation of the nothingness of its being." \autocite[148]{sartre}.
  \end{enumerate}
\end{enumerate}

\noindent
As a sort of parting remark on this section, it seems that Sartre's ontology places an important role on the idea of a \emph{transcendence}, or a \emph{transcendent} thing. Whenever we are looking for something (i.e. some being, or quality of being) which does not exist in the thing (i.e. the being) itself, but comes from something which is beyond the given thing (i.e. being), we are looking for a transcendent thing. The transcendent thing is like a higher object to which a shadow is cast.

\subsubsection{The For-Itself and the Being of Possibles}

\begin{enumerate}
  \item In this section, Sartre takes the concept of \emph{lacking} and relates it to the concept or being of \emph{possibles.} He derives possibility from lacking through a similar transcendental meditation.
  \item The being of possibility is not in the being of any existents, but rather comes from the human-reality.
  \item However, possibility is also not subjective!
  \item Possibility is not within the being of the human-reality, but it is also transendent. It seems to be something outside of human reality. \autocite[158]{sartre}
  \item "Let us call the for-itself's relation to the possible that it is the `circuit of ipseity' -- and the totality of being, insofar as it is traversed by the \emph{circuit of ipseity}, the `world.'" \autocite[158]{sartre}
  \item I'm not entirely certain at this point, but it sounds like that the world is the totality of possibility (in the context of human-reality), while the self traverses a subset of that as the circuit of ipseity.
\end{enumerate}

\subsubsection{My Self and the Circuit of Ipseity}

This seems to be a summary of the above sections and the chapter in general. I should revisit it sometime, in particular \autocite[161]{sartre}.

\subsection{Chapter 2: Temporality}

\subsubsection{Phenomenology of the Three Temporal Dimensions}

\begin{enumerate}
  \item
\end{enumerate}

\subsubsection{The Ontology of Temporality}

\begin{enumerate}
  \item
\end{enumerate}

\subsubsection{Original Temporality and Psychological Temporality: Reflection}

\begin{enumerate}
  \item
\end{enumerate}

\subsection{Chapter 3: Transcendence}

\subsubsection{Knowledge as a Type of Relation Between the For-Itself and the In-Itself}

\begin{enumerate}
  \item
\end{enumerate}

\subsubsection{On Determination as Negation}

\begin{enumerate}
  \item
\end{enumerate}

\subsubsection{Quality and Quantity, Potentiality and Equipmentality}

\begin{enumerate}
  \item
\end{enumerate}

\subsubsection{World-Time}

\begin{enumerate}
  \item
\end{enumerate}

\subsubsection{Knowledge}

\begin{enumerate}
  \item
\end{enumerate}
