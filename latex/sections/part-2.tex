\section{Part II: Being-For-Itself}

A quick definitional reference:

\begin{enumerate}
  \item \textbf{Being-for-itself}: The being \emph{for-itself} is the kind of being of consciousness. To quote Oxford Dictionary, being for-itself \textcquote{oxford}{is the mode of existence of consciousness, consisting in its own activity and purposive nature.}
  \item \textbf{Being-in-itself:} is the existence of ordinary, non-conscious objects, like tables or chairs.
\end{enumerate}

\subsection{Chapter 1: The Immediate Structures of the For-Itself}

\subsubsection{Self-Presence}

\begin{enumerate}
  \item \textbf{The In-Itself:} Something that has an infinite density of being, a plentitude. I think when Sartre talks about the (Being-)In-Itself, he is talking directly about the being of the existent (perhaps analogous to the being-of-the-phenomena?).
  \item \textcquote[123]{sartre}{Identity is the limiting concept of unification \ldots\ at its extreme limit, unity vanishes and passes over into identity.}
  \item \textcquote[123]{sartre}{Consciousness is characterised, on the contrary, by its \emph{decompression of being.} Indeed, it is impossible to define it as self-coincident.}
  \begin{enumerate}
    \item What this means is that the being of consciousness is not a being that's in-itself. The example that Sartre gives is that when we talk about my belief, I cannot say that my consciousness \emph{is} by belief. But rather only that \textcquote[123]{sartre}{my belief is a consciousness (of) my belief.}
    \item See how in this base, the being-of-consciousness is not infinitely dense?
  \end{enumerate}
  \item \textbf{Self-Presence as the foundation for self-consciousness:} I'm not too certain in my understanding of this right now, but Sartre elaborates on this in \autocite[126]{sartre}. The gist of it seems to be:
  \begin{enumerate}
    \item The being-in-itself is the regular being of existents, i.e. the being-of-the-phenomena.
    \item However, what is the being-for-itself? The very use of the word \emph{for} implies a strong reflective action.
    \item We cannot qualify the being-\emph{for}-itself using any regular conception of the being-in-itself.
    \item The key difference in the being-for-itself is that there's a separation which makes the being not its own coincidence, but still requires its own unity (bottom of \autocite[126]{sartre})
    \item Hence this self-presence must have some sort of separation, which will be shown to be \emph{nothing}.
    \item \textcquote[126]{sartre}{The law of being of the for-itself as the ontological foundation of consciousness is to be itself in the form of self-presence.}
  \end{enumerate}
  \item \textbf{Self-Presence as an act of separation from the self:} Sartre elaborates this at \autocite[127]{sartre}. Self-Presence is taken as something different, or apart from identity -- which is the dense plentitude of being, as we have explained above. In fact:
  \begin{enumerate}
    \item \enquote{The principle of identity is the negation of any type of relation within the being-in-itself.}
    \item \enquote{On the contrary, self-presence presupposes that an intangible fissure has slipped inside being. If it is present to itself, that is because it is not completely itself. Presence is an immediate degradation of coincidence, because it presupposes separation.}
    \item \textcquote[127]{sartre}{But if we ask now \emph{what} separates the subject fro himself, we are forced to admit that it is \emph{nothing.}}
  \end{enumerate}
  \item So once again, we are back at the discovery that nothing is essential for being -- in this case, nothing is essential for self-presence.
\end{enumerate}

\subsubsection{The For-Itself's Facticity}

\begin{enumerate}
  \item It seems like this section Sartre ties down the being-for-itself into the being-in-itself. Essentially, there has to be some sort of foundation for the for-itself.
  \item \textcquote[133]{sartre}{Thus the for-itself is supported by a constant contingency that it takes up, and assimilates, without ever being able to get rid of it. We may call this constantly evanescent contingency of the in-itself -- which haunts the for-itself and ties it to being in-itself without ever allowing itself to b e grasped -- the for-itself's \emph{facticity.}}
  \item It seems to me that this distinction -- the idea that the being-for-itself is founded upon a factual circumstance -- is important to avoid the illusion and absurdity of solipsism.
  \begin{enumerate}
    \item \textcquote[134]{sartre}{The for-itsef, even while it chooses the \emph{meaning} of its situation and constitutes itself in situation as its own foundation, does \emph{does not choose} its position.}
  \end{enumerate}
\end{enumerate}

\subsubsection{The For-Itself and the Being of Value}

\begin{enumerate}
  \item \textbf{The Lack:} The lack is the form of negation which \textcquote[137]{sartre}{most deeply establishes an internal relation between what we negate and what our negation applies to \ldots\ [the lack is the form of negation] which penetrates most deeply into being -- the one that constitutes \emph{in its being} the being to which its negation applies with the being that it negates.}
  \begin{enumerate}
    \item \textcquote[138]{sartre}{The lack does not belong to the nature of the in-itself, which is entirely positive. It appears within the world only when human-reality arises.}
    \item For example, given an unfinished circle -- it is technically an open curve that is complete in its being as an open curve. It is only through the realm of human-reality, specifically human desire, to which we give to it the lack -- the lack in which it is not a circle. \autocite[139]{sartre}
  \end{enumerate}
  \item Human reality itself must be a lack, because only through a lack can we derive lacks. Sartre talks about this very definitively in \autocite[139]{sartre}:
  \begin{enumerate}
    \item \textcquote[139]{sartre}{A psychological state whose existence had the sufficiency of that curve [i.e. the unfinished circle] could not in addition make the slightest `call' for anything else: it would be itself, without any relation to anything other than itself}
    \item \textcquote[139]{sartre}{In order to constitute it as a hunger or thirst, an external transcendence would be required.}
    \item \textcquote[139]{sartre}{No recourse to psychophysiological parallelism [i.e. the doctrine that the psychological is a direct parallel to the physiological] can enable us to escape these difficulties:}
    \begin{enumerate}
      \item Any physiological signs of a lack of water in an organism only posits a positive being of the state of the organism, refering to itself. Sartre presents this in vivid detail in \autocite[139]{sartre}
      \item \textcquote[139]{sartre}{[Any] exact correspondence between the mental and the physiological [requires] that correspondance [to be] established only on the basis of an ontological identity.}
    \end{enumerate}
  \end{enumerate}
  \item \textbf{Desire is a \emph{lack} of being:} \textcquote[140]{sartre}{and is haunted in its [desire's] innermost being by the being that it desires [i.e. lacks].}
  \item \textbf{Lack is a trinity:} When we lack something, there are three components to the act of lacking \autocite[138]{sartre}:
  \begin{enumerate}
    \item \textbf{The \emph{manqué} (i.e. the lack):} The item that is missing
    \item \textbf{The incomplete existent:} That from which [the item] is missing [i.e. the existent].
    \item \textbf{The hypothetical whole:} A totality that is broken apart by the lack, and could which be restored by the synthesis of the missing item with the existent.
  \end{enumerate}
  \item In \autocite[139]{sartre} talks about how \textbf{value} comes from this lack. I need to investigate this further.

  \subsubsection*{The Being to Which Consciousness Aims For}
  \item Sartre presents a rather tricky, but essential understanding on \textbf{the emergence of value from lack and desire}. It's essential that we understand what value is, and where does it come from. Right now, take value in this case to mean ethical/personal value, i.e. what is important to us, or what we aim for. Sartre's presentation goes as follows:
  \begin{enumerate}
    \item Recall that lack is a trinity.
    \item Further recall that \textbf{the human-reality is a lack} (since otherwise, the being of the human condition would be positive, and there would be no such thing as lacking).
    \item Hence, \emph{if the human-reality is a lack, what are the components of the lack's trinity?} Sartre answers this in \autocite[140]{sartre}, where he states:
    \begin{enumerate}
      \item \textbf{The \emph{manqué} (i.e. the lack):} \textcquote[141]{sartre}{The \emph{itself-as-being-in-itself}.}
      \item \textbf{The incomplete existent:} \textcquote[139]{sartre}{The element that plays the role of the existent is given to the \emph{cogito} as the immediacy of the \emph{desire.}}
      \item \textbf{The hypothetical whole:} \ldots
    \end{enumerate}
    \item From the incomplete trinity above, Sartre asks: what is this hypothetical whole from which the lack of the human-reality presupposes?
    \item It seems to me that this hypothetical whole is a transcendence towards a better whole, a better version of the being [i.e. self] (?)
    \item \textcquote[142]{sartre}{This constantly absent being which haunts the for-itself is itself -- but frozen in the in-itself [i.e. as an object].}
    \begin{enumerate}
      \item My intepretation of this sentence is essentially thus: Our human-reality is defined by a negative thing, a lacking. But a lacking must presuppose first a thing that is lacking (e.g. the missing puzzle piece), which Sartre cals the manqué -- as well as the incomplete existent (e.g. the puzzle-hole) and the hypothetical whole (e.g. the complete puzzle.)
      \item The incomplete existent is manifest as desire.
      \item But the thing which we are lacking in our human-reality is another state of human-reality or being, which is the object of our consciousness. Our consciousness wishes to be something else, to be another consciousness -- which it is not.
      \item Hence the ultimate, hypothetical, and unachievable synthesis of what we lack from the lacking is where \emph{value comes from}.
    \end{enumerate}
    \item \textbf{Value: a transcendent thing which our current being lacks, which eludes our being.} \autocite[146]{sartre}
    \item \textcquote[147]{sartre}{Value arrives to the world through human-reality.}
    \item \textcquote[148]{sartre}{Value haunts being insofar as it founds itself and not insofar as it is: it haunts \emph{freedom}. So value's relation to the for-itself is quite distinctive: it is the being that the for-itself has to be, insofar as it is the foundation of the nothingness of its being.}
  \end{enumerate}
\end{enumerate}

\noindent
As a sort of parting remark on this section, it seems that Sartre's ontology places an important role on the idea of a \emph{transcendence}, or a \emph{transcendent} thing. Whenever we are looking for something (i.e. some being, or quality of being) which does not exist in the thing (i.e. the being) itself, but comes from something which is beyond the given thing (i.e. being), we are looking for a transcendent thing. The transcendent thing is like a higher object to which a shadow is cast.

\subsubsection{The For-Itself and the Being of Possibles}

\begin{enumerate}
  \item In this section, Sartre takes the concept of \emph{lacking} and relates it to the concept or being of \emph{possibles.} He derives possibility from lacking through a similar transcendental meditation.
  \item The being of possibility is not in the being of any existents, but rather comes from the human-reality.
  \item However, possibility is also not subjective!
  \item Possibility is not within the being of the human-reality, but it is also transendent. It seems to be something outside of human reality. \autocite[158]{sartre}
  \item \textcquote[158]{sartre}{Let us call the for-itself's relation to the possible that it is the \enquote{circuit of ipseity} -- and the totality of being, insofar as it is traversed by the \emph{circuit of ipseity}, the \enquote{world.}}
  \item I'm not entirely certain at this point, but it sounds like that the world is the totality of possibility (in the context of human-reality), while the self traverses a subset of that as the circuit of ipseity.
\end{enumerate}

\subsubsection{My Self and the Circuit of Ipseity}

This seems to be a summary of the above sections and the chapter in general. I should revisit it sometime, in particular \autocite[161]{sartre}.

\subsection{Chapter 2: Temporality}

\subsubsection{Phenomenology of the Three Temporal Dimensions}

In this part, Sartre wishes to examine the past, the present, and the future -- without the explanation of time being a simple series of `nows' or moments, since this naive approach yields Xeno's paradox. In this discussion, he presents what he later refers to as the \emph{three temporal ecstasies}, which are acts of unification. \marginnote{My understandings of Sartre's temporal ecstasies is not very clear at the moment. How can I deepen my understanding of their them?}

\begin{enumerate}
  \subsubsection*{The Past}
  \item Sartre rejects the naively materialist (or in his terms, the psychophysiological parallelism) of the theory of `memory traces,' where the past is seen as something that is departed, and hence every memory is merely a physical, present trace in the mind.
  \item In the next pages, he presents a few non-materialist approaches to understanding where the past derives its being, and goes on to reject all of them.
  \item His conclusion is that the past must derive its being from the person to whom the past is for. He elaborates most keenly on this conclusion at \autocite[169]{sartre}. He presents an example with Pierre:
  \begin{enumerate}
    \item \textcquote[170]{sartre}{Of \emph{whom} is this past-Pierre the past? It cannot be in relation with an universal Present which purely affirms being; it is therefore the past of \emph{my actuality.} And as a matter of fact Pierre has been for-me and I have been for-him.}
  \end{enumerate}
  \item \textcquote[170]{sartre}{There are therefore beings that \enquote{have} pasts.} However, this \emph{does not} mean all beings have pasts! Rather, \emph{only a specific type of being} has a past, which Sartre elaborates in \autocite[172]{sartre}:
  \begin{enumerate}
    \item \textcquote[172]{sartre}{There is a past only for a present that cannot exist without being its past `over there,' behind it. In other words, \emph{the only beings that have a past are those beings for whom there is a question, in their being, of their past being} -- beings that \emph{have} their past \emph{to be}.}
    \item In my own words, the only types of beings that have a past, are the sort of beings which contain a question of their own being.
    \item or in other words, the only beings that have a past, are the beings that are beings \emph{for-itself,} (i.e. the being-of-consciousness). Beings that are only \emph{in-itself} (i.e. the being-of-phenomena) do not have pasts!
  \end{enumerate}
  \item Sartre then proceeds to learnedly make the important nuanced qualification that this \textcquote[172]{sartre}{\emph{does not settle the question of the past of living things}.}
  \begin{enumerate}
    \item Remember how we defined that the only beings which have a past, are the beings which are for-itself? In a more vulgar manner of speaking, we're talking about beings that are conscious.
    \item There are of course, plenty of living things like moss or algae which obviously do not fulfill this definition.
  \end{enumerate}
  \item In \autocite[174]{sartre} Sartre talks about the relationship between the past and death. There are some particularly memorable quotes (you should revisit the cited page):
  \begin{enumerate}
    \item \textcquote[174]{sartre}{Ultimately, at the infinitisimal instant of my death, I will no longer be anything but my past. It alone will define me.}
    \item \textcquote[174]{sartre}{Through death, the for-itself [being of consciousness] changes for eternity into in-itself [being of phenomena], to the extent to which it has entirely slipped into the past. Thus the past is the \emph{ever-increasing totality of the in-itself that we are.}}
  \end{enumerate}

  \item \textcquote[176]{sartre}{To explain the world in terms of becoming, conceived as a synthesis of being and non-being, is easily done. But has anyone considered that no being that becomes could be such a synthesis unless it were, in relation to itself, \emph{an act that founded its own nothingness?}}
  \item \textcquote[180]{sartre}{To sum it up, [the past] is an inversion of value, the for-itself reclaimed by the in-itself, thickened by the in-itself to the point at which it can no longer exist as a reflection for the reflecting, or as a reflecting for the reflection, but merely as an in-itself sign of the reflecting-reflection pair.}
  \begin{enumerate}
    \item This is an important summary on Sartre's conclusion on the nature of the past. To put into more simple words, the past is the \emph{being-for-itself} (i.e. the being of consciousness) which has became the \emph{being-in-itself}, the mere being of the phenomenon.
  \end{enumerate}
  \subsubsection*{The Present}
  \item \textcquote[181]{sartre}{[Any] strict analysis that aimed to rid the present of everything it is not -- i.e. its immediate past and future -- would in fact find nothing more than an infinitesimal instant \ldots\ the ideal term of an infinitely pursued division: a nothingness.}
  \begin{enumerate}
    \item With this opening passage, Sartre presents the fundamental problem of \emph{the present}, and relates it thematically to the earlier conceptions of nothingness which we discovered in the past.
    \item The first realisation that Sartre presents, is that the idea of the present -- or formally speaking, the attribute of \emph{presence} -- is a quality that only exists between two beings.
    \begin{enumerate}
      \item \textbf{Presence:} the quality of an object being \emph{present}.
    \end{enumerate}
    \item \textcquote[181]{sartre}{The in-itself cannot be present, any more than it can be past; it \emph{is}, quite simply. There can be no question of any one in-itself existing in some kind of simultaneity alongside another in-itself -- other than from the point of view of a being who was co-present to the two in-itselfs, and who had its own capacity for presence.}
  \end{enumerate}
  \item \textcquote[181]{sartre}{Therefore \textbf{the present can only be the for-itself's presence to being-in-itself.}}.
  \begin{enumerate}
    \item If I am understanding this argument properly, essentially the present is a quality which is only shared by a being which has consciousness in the first place. Objects (beings-in-itself) are present to a being-for-itself. But in a world without beings-for-itself (conscious beings), there would be no such thing as a present, or objects present to it.
  \end{enumerate}
  \item Now Sartre segways to a new section, where we investigate \textcquote[182]{sartre}{to which being does the for-itself make itself a presence?}
  \item \textcquote[183]{sartre}{Our presence to any being implies that we are linked to that being by an internal-connection; otherwise no link between the present and being would be possible. But this internal connection is negative: it denies, with respect to a present b eing, that it is that being to which it is present. Otherwise the internal connection would isappear into a straightforward identification.}
  \item I'm not sure how to quite summmarise this section, but it seems to me the goal of this is for Sartre to present the neccesity of negation in all of it's forms within the being of consciousness (i.e. the being-for-itself). Negation is neccessary for the past, for the present, and as we shall soon see, for the future as well.

  \subsubsection*{The Future}
  As a quick summary, Sartre's conception of the future likewise derives its ontological foundation from the negative element present within the being-for-itself (i.e. consciousness). Where the future cannot and does not come from neither a simple material relation or quality, nor does it come from a simple quality of the being-for-itself. But rather, it is that negative aspect, an \emph{lack}. The best way I can understand this argument is that just as the being of the for-itself flees from the past (because the past is what it's not), the for-itself has to flee \emph{towrards} something -- and it would not be inaccurate to call that thing to which it flees toward the \emph{future}.

  \item \textcquote[184]{sartre}{Let us note first that the in-itself cannot be the future, and nor can it contain any part of the future. When I look at this crescent moon, the full moon is in the future only \enquote{within the world} that is disclosed to human reality: it is through human reality that the future arrives in the world. In itself, this quarter of the moon is what it is. Nothing in it as potentiality. It is in actuality.}
  \item \textcquote[185]{sartre}{Even were we to accept, as Laplace does, a complete determinism that would enable us \emph{to predict} a future state, this future circumstance would still need to be profiled against an antecedent disclosure of the future as such, a being-to-come of the world.}
  \item With the first two statements, it seems clear to me that the future has to have a distinct, unique ontological existance -- that cannot be dismissed away as a simple property of the material world. This understanding is developed more concretely in Sartre's subsequent sections.
  \item \textcquote[185]{sartre}{Only a being who has to be its being, rather than merely being it, can have a future}
  \begin{enumerate}
    \item By \emph{a being who has to be its being}, Sartre is talking about the specific, ontologically distinct kind of being whose own being \emph{possesses the question of its being}. In other words, this being is the \emph{Daesin}, or perhaps more generally, a conscious being (i.e. the being for-itself). This is in contrast to beings that are merely objects, i.e. beings of-itself.
  \end{enumerate}
  \item Sartre states in the following paragraphs that the future is not merely \enquote{representation}, nor is it mere \enquote{a futurising intention}. This excludes the more popular and common ontologies of future-ness.
  \begin{enumerate}
    \item In fact, Sartre is quite clear and unequivocal about how future-ness cannot be derived as a mere property of the material world, of mere beings-in-itself.
    \item Likewise, the future is not a simple property of just consciousness alone -- this is a more nuanced thesis:
    \item \textcquote[186]{sartre}{The for-itself can neither be \enquote{pregnant with the future}, nor an \enquote{awaiting of the future}, except against the ground of an original and prejudicative relation of the self to itself.}
  \end{enumerate}
  \item So what is the future, under Sartre's phenomenological conception of ontology? There's a specific argument that Sartre makes, where the future derives it's being from a certain negative presence of the being-for-itself. I'll try to present this argument to the best of my understanding:
  \marginnote{This inversion of causes present in his conception of the future is very interesting. In fact, Sartre calls it \enquote{casuality in reverse \ldots\ the efficient power of a future state.} Whaat does this mean for causes in general?}
  \begin{enumerate}
    \item \textcquote[186]{sartre}{Let us take a simple example: this position which I keenly take up on the [tennis] court has meaning only through the movement I will make next, with my racket, to send the ball back over the net. But I am obeying neither my \enquote{clear representation} of the future movement, nor my \enquote{firm resolution} to accomplish it \ldots\ it is my future movement which, without even being thematically presented, turns backward to the positions I adopt, in order to illuminate, to connect, and to modify them.}
    \item \textbf{\textcquote[186]{sartre}{There is not a moment of my consciousness that is not similarly defined by an internal relation to a future; whether I write, I smoke, I drink, or I rest, the meaning of my [acts of] consciousness is alwyas at a distance, over there, outside.}}
    \item \textcquote[187]{sartre}{The future is \emph{what I have to be} insofar as I cannot be it.}
  \end{enumerate}
  \item \textcquote[187]{sartre}{Recall that the for-itself, confronted with being, presentifies itself as not being that being, and as having been its past. This presence is flight, because, in fleeing from the being that it is not [i.e. the past], presence flees from the being that it was. \emph{What} does it flee toward? Let us not forget that the for-itself, insofar as it presentifies itself to being in order to flee from it, is a lack \ldots\ From this we can grasp the meaning of the flight involved in presence: it is a flight towards \emph{its being}.}
  \item \textcquote[191]{sartre}{[The for-itself reaches the future] in vain: the for-itself can only ever be its future problematically, because it is separated from it by the nothingness that it is. In brief, the for-itself is free, and its freedom sets its own limit to itself. To be free is to be condemned to be free. \emph{Thus the future, insofar as it is the future, has no being.} It is not \emph{in itself} and nor does it have the for-itself's mode of being either, since it is the for-itself's \emph{meaning}. The future is not; it \emph{possibilises} itself.}
\end{enumerate}

\subsubsection{The Ontology of Temporality}

After examining the tripartite division of \emph{the past}, \emph{the present}, and \emph{the future,} Sartre turns to examine the ontology of temporality itself. In this section, he begins with the following dichotomy of \emph{static temporality} and \emph{dynamic temporality}, where:

\begin{enumerate}
    \item \textbf{Static Temporality:} The elements of \emph{before} and \emph{after}.

    \textcquote[193]{sartre}{What Kant calls the \emph{order} of time.}

    \item \textbf{Dynamic Temporality:} The fact of succession, the motion of how every after becomes a before.

    \textcquote[193]{sartre}{What Kant calls the \emph{course} of time.}
\end{enumerate}

\noindent
Sartre separates the two and begins an examination of each individually. We will begin with static temporality.

\begin{enumerate}
  \subsubsection*{Static Temporality}
  \item \textcquote[193]{sartre}{The \enquote{before-after} order [of static temporality] is defined in the first place by its irreversibility. We call a series of \enquote{successive} if its terms can only be considered one by one, and in only one direction.}
  \item \textcquote[193]{sartre}{Without the succession of \enquote{others} I could be what I want to be straightaway, and there would no longer be any distance between me and myself, or any separation between an action and a dream.}
  \item It is this very atomic separate-ness of the temporality of \emph{instants} that yields the ontological problem of temporality. After all, by reducing every moment to an instant, the casual order between instants seem to disappear. Sartre summarises this problem by stating:
  \item \textcquote[194]{sartre}{Thus, when we consider in isolation temporality's power to dissolve, we are forced to admit that \emph{the fact of having existed at any given instant does not constitute the right to exist at the folowing instant}, nor even mortage or an option on the future.}
  \item At this point, Sartre exams three competing solutions to this problem of succession and order in atomic temporal ontology:
  \begin{enumerate}
    \item \textbf{Kant}: Tries to resolve this by making the witness of time (i.e. the being who experiences time) temporal, and by having time come from a transcendental relationship of the witness towards God.
    \item \textbf{Descartes}: Same as Kant, except the ultimate unifying act of the witness with the temporal object is the \emph{I think} of reason.
    \item \textbf{Leibniz}: Rejects Kant and Descarte, and attempts to view all time as \enquote{pure relation of immanence and cohesion}, where time is continuous and not atomic at all.
  \end{enumerate}
  \item Ultimately, he finds all three approaches to be lacking and/or inadequate in some way.
  \item \textcquote[198]{sartre}{How can a timeless being, having to unify timeless elements, conceive of the kind of unification that belongs to succession? And if -- as we would need to agree in that case -- the \emph{esse} of time is a \emph{percipi}, how will the \emph{percipitur} be constituted? \ldots\ Thus, insofar as [time] is at the same time a form of separation and a form of synthesis, temporality will not permit us either to derive it from something timeless or to impose it \emph{from outside} on timeless things.}
  \item In that vein of questioning, both Sartre and the reader asks: \enquote{Who \emph{draws} time?}
  \item \textcquote[200]{sartre}{What may we conclude, at the end of this discussion? In the first place, this: temporality is a force that dissolves, but it does so within an act of unification; it is not so much a real multiplicity [but] as a quasi-multiplicity, the first draft of a dissociation within unity}
  \marginnote{Once again, we see the theme of nihilation in this act of dissolution. How does this relate to the broader theme of being arising from nothingness?}
  \begin{enumerate}
    \item \textcquote[200]{sartre}{Time cannot be a real multiplicity for it could not subsequently receive any unity and could not, in consequence, even exist in the form of real multiplicity}
    \item \textcquote[200]{sartre}{If we start by positing temporal unity, we are at risk of no longer even being able to understand anything about the irreversible succession as the \emph{meaning} of this unity.}
    \item \textcquote[200]{sartre}{We must conceive [temporality] as a unity that multiplies \emph{itself}, which means temporality can only be a relation of being \emph{within the same being.}}
  \end{enumerate}
  \item \textcquote[200]{sartre}{\textbf{Temporality is \emph{not}.} Only a being with a specific structure of being can, in the unity of its being, be temporal. \enquote{Before} and \enquote{After} is intelligible only as being what is \emph{before} itself.}
  \begin{enumerate}
    \item \textcquote[201]{sartre}{Rather, the for-itself, in existing, temporalises itself.}
  \end{enumerate}
  \marginnote{What sort of being does Sartre refer to? Is this likewise the conscious being, the being-for-itself which contains negation \enquote{in its heart like a worm?} Likewise, how does this relate to, or differ from ipseity?}
  \item \textcquote[201]{sartre}{Temporality must have the structure of ipseity.}

  \subsubsection*{The Birth/Emergence of Consciousness in Temporality}

  Around pages \autocite[204]{sartre} of the preceding section, Sartre goes on a parallel, but \emph{deeply} fascinating tangent on the absolute \emph{neccessity} of temporality for the being-for-itself. It begins with the question of \enquote{How can temporal things have a definite \emph{beginning point}?}, otherwise called the \enquote*{Problem of Birth} and ends up as a deeper investigation on the ontology of emergence.
  \marginnote{What parallels can we draw from the emergence of consciousness in temporal ontology, to the artificial creation of conscious minds?}

  \item \textcquote[203]{sartre}{In effect, it strikes us as scandelous that consciousness should come at some moment \enquote{appear} and should come to \enquote{inhabit} the embryo, or in short, that there should be one moment in which the living thing, as it develops, lacks any consciousness and another moment in which a consciousness without any past becomes imprisoned within it.}
  \item In order to resolve this paradox, Sartre takes his previous statements about the nature of temporality in being, and posits that absolutely it is impossible for any conscious being (being-for-itself) to lack a past.
  \item \textcquote[204]{sartre}{The for-itself's being is originally constituted by this relation to a being that is \emph{not} consciousness, existing within the complete night of identity, that the for-itself is, however, outside itself, behind itself}
  \begin{enumerate}
    \item \textcquote[204]{sartre}{The in-itself is what the for-itself was \emph{before}. In consequence, it makes perfect sense that our past does not appear to us as if it were limited by a clean line, with no smudges}
  \end{enumerate}
  \item \textcquote[204]{sartre}{There is no ontological problem: we do not have to ask ourselves how a consciousness can be born, because consciousness can appear to itself only as the nihilation of in-itself, i.e. as \emph{having already been born.}}

  \subsubsection*{The Temporal Dynamic}
  The central theme of this section seems to be to answer the question: \enquote{where does the dynamicism of temporality as a continuous progression/succession come from?} I'm not sure if I understand the entirety of the answer, but Sartre does present an interesting approach to this question from the perspective of change.

  \item In the first segment of Sartre's rhetorical progression, he explores the common idea of temporal progression as \emph{change}, either as a symptom of change, or as something which emerges from change. Specifically, he responds to the idea that \textcquote[208]{sartre}{temporality is reduced to being no more than the measure and order of change. Without change there would be no temporality, since time would have no purchase on the permanent and identical}. Sartre \emph{rejects} this conception, calling it one \enquote{based on many mistakes}.
  \item \textcquote[209]{sartre}{In brief, the change's \emph{unity} with the permanent is necessary for the constitution of a change as such.}
  \item \textcquote[209]{sartre}{The appeal to permanence in order to found change is, moreover, utterly useless. The idea is to show that any absolute change is, strictly speaking, no longer a change, since \emph{nothing} remains that is changing -- or in relation to which there could be change.}
  \item \textcquote[209]{sartre}{But in addition, when we are dealing with human-reality, what is necessary is pure and absolute change, which moreover is perfectly able to be a change while \emph{nothing} changes -- and which is duration itself. Even if we allowed that a for-itself could be an absolutely empty presence to a permanent in-itself-- \emph{the very existence of that consciousness would imply temporality,} since it would have to be, without changing, what it is, in the form of \enquote{having been it.}}

  \noindent
  Now at the start of \autocite[211]{sartre} Sartre begins his presentation on the main body of his argument, which is that dynamic temporality derives its motive force from the process of the being-for-itself fleeing from a in-itself past towards the future. He uses the image of \enquote{a hole constantly being filled}.
  \item \textcquote[210]{sartre}{The present cannot \emph{pass} [into the past] except by becoming the \enquote{before} of a for-itself that thereby constitutes itself as \enquote{after.} There is therefore just one phenomenon: the arising of a new present that \enquote{pastifies} the present that it \emph{was} and in the wake of the past-ification of a present, the appearing of a for-itself for whom that present will become the past.}
  \item We must take especial care and attention towards the generation of a \enquote{for-itself for whom that present will become the past.} It seems intimately related to the specific ontological being of past-pasts, and past-futures -- which Sartre presents shortly following:
  \begin{enumerate}
    \item \textcquote[211]{sartre}{The past of the present that has undergone in its past-ification becomes the past of a past -- or the \emph{pluperfect.} In relation to it, the present's heterogeneity with the past is immediately eliminated.}
    \item \textcquote[211]{sartre}{On the other hand, although the future is equally affected by the metamorphosis it does not cease to be the future -- which means it remains outside the for-itself, in front of it, beyond being -- but it becomes the future of a past, or the \emph{future-perfect}.}
  \end{enumerate}
  \marginnote{What is the exact relationship or ontological difference between the \enquote*{past of a past} (i.e. the pluperfect), and the \enquote*{future of a past} (i.e. the future-perfect)?}
  \item It's clear that the ontology of a past-of-a-past is essentially different from the ontology of a future-of-a-past, where the future-of-a-past still possesses some transcendental nature.
  \item  \textcquote[211]{sartre}{The connection between the past and the pluperfect is a connection in the mode of the in-itself and it appears upon the foundation of the present for-itself}
  \item At this point I am at able to understand the following thematic idea: As the future moves in this dynamic motion into the past, the future loses the transcendental property (which I admittedly still cannot properly define) but becomes a matter of the \emph{in-itself}. The past is absolutely fixed, and \emph{objective} -- it takes no part in the being of the living \emph{being-for-itself}, but is merely the mundane \emph{being-in-itself}.
  \item  \textcquote[213]{sartre}{The past is a backward fatality: the for-itself makes itself what it wants, but [the for-itself] cannot escape the necessity that a new for-itself will be, irremediably, what it wanted to be.}
  \item  \textcquote[211]{sartre}{The past, therefore, is a for-itself \emph{that has ceased to be a transcending presence to the in-itself. As itself \emph{in itself}, it has fallen \emph{into the midst of the world.}}}
  \item \textcquote[213]{sartre}{In the past the world hems me in and I become lost within a universal determinism, but I radically transcend my past toward my future, to just the extent to which I \enquote{was} that past.}
  \item \textcquote[213]{sartre}{What is the meaning of this arising of the for-itself? We must be careful not to regard it as the appearance of a new being. It is as if the present were a constant hole in being which, the moment it is filled in, constantly reappears: as if the present were in constant flight from the threat of becoming bogged down in \enquote{in itself,} a threat that continues until the in-itself's final victory, which drags it into a past that is no longer any for-itself's past. This victory is death, because death puts a radical stop to temporality, by past-ifying the entire system, or alternatively, by the in-itself's seizing back of the human totality.}
  \item \textcquote[216]{sartre}{The time of consciousness, therefore, is human-reality temporalising itself as a totality that is its own unfinished task; it is nothingness, sliding into a totality like a detotalising enzyme. This totality is simultaneously chasing after, and rejecting itself; it is unable to find any final term within itself for its surpassing, because it is its own surpassing, and surpasses itself towards itself; such a totality cannot, in any case, exist within the limits of an instant.}
  \end{enumerate}

\subsubsection{Original Temporality and Psychological Temporality: Reflection}

This is a great section, which serves both as a synthesis of the previous tri-partite division of temporality, as well as a means of deriving psychological reality from temporality. In summary, the following process seems to occur. First, Sartre asks the question of \enquote{How does the being-for-itself actually perceive the passage of time?,} and in order to answer this question he makes the distinction between \emph{original temporality} and \emph{psychological temporality}.

Original temporality is the abstract, ontological synthesis of the past, present, and the future.
\marginnote{Once again, I would like to have a better understanding on the process of the ecstatic synthesis of the three temporal ecstasies.}
It is the thing that we talk about when we are operating purely in the level of meaning and being, insofar the nature of the being-for-itself is concerned. Original temporality is concerned purely with the non-thetic nature of the being-for-itself (i.e. the un-conscious conscious). He talks about this division starting at \autocite[217]{sartre}

However, there is an impure derivative of original temporality, which comes as soon as the relation between the being-for-itself and temporality becomes \emph{thetic}. When the positional consciousness of the mind \emph{posits} about time, and the passage of time, all of a sudden the process becomes different. This is akin to moving one step down on the ladder of metaphysics. In order to answer what is psychological temporality, and indeed even distinguish the difference between the two -- Sartre has to go on a slight detour to talk about the nature of \emph{reflection} itself. Because remember -- if original temporality is purely a component of the being-for-itself, then in order for there to be any positional (i.e. psychological) awareness of temporality, the positional consciousness has to have \emph{itself} (i.e. the consciousness) as its object of reflection. Hence we must first understand reflection, in order to properly talk about the two.

\begin{enumerate}
  \item \textcquote[217]{sartre}{The for-itself endures in the form of a non-thetic consciousness (of) enduring. But I am a ble to \enquote{feel time passing} and to apprehend myself as a unity of scucession. In this case I am conscious \emph{of} enduring. This consciousness is thetic and closely resembles knowledge, just as duration being temporalised before my eyes comes close to being an object of consciousness. \emph{What kind of relation exists between original temporality and this psychological temporality,} that I encounter as soon as I apprehend myself \enquote{in the process of enduring?}}
  \item \textcquote[218]{sartre}{Reflection is the for-itself as conscious \emph{of} itself.}
  \item \textcquote[220]{sartre}{For a consciousness, to become reflected is to undergo a deep modification in its being and precisely to lose the \emph{selbstständigkeit} it possessed.}
  \item \textcquote[217]{sartre}{The person who reflects on me is not some kind of  pure, timeless gaze: it is myself, an enduring me, who is commited within the circuit of my ipseity, in danger within the world, and with my historicity.}
  \item \textcquote[224]{sartre}{Reflection is knowledge; that is beyond doubt; it possesses a positional character, and it affirms the consciousness it reflects on.}
  \item \textcquote[224]{sartre}{[Reflection's] knowledge is totalising: it is a lightning intuition without any contrasts, or any point of departure, or any point of arrival. Everything is given at once in a kind of absolute proximity.}
  \item \textcquote[224]{sartre}{But if reflective consciousness \emph{is} what it reflects on -- if this unity of being founds and limits the authority of reflection -- we ought to add that what we reflect on \emph{is} itself its past and its future.}
  \begin{enumerate}
    \item \textcquote[224]{sartre}{We can reach this conclusion, moreover, from the fact that \emph{thinking} is an act that commits the past, and is sketched out in advance by the future. \enquote{I doubt, therefore I am,} said Descartes.}
    \marginnote{The object of reflection is fundamentally temporal in nature, since the result of it's discernment requires the separation of the past and the future.}
    \item \textcquote[225]{sartre}{For there to be doubt, it is necessary for this suspension to be motivated by an insufficiency of reasons either to assert or to deny -- which refers to the past -- and that should be deliberately maintained until new elements intervene -- which is already a project of the future.}
  \end{enumerate}
  \item \textcquote[225]{sartre}{Now, if our findings are correct, reflection is a for-itself seeking to reclaim itself as a totality that is constantly in a state of incompleteness.}
  \begin{enumerate}
    \item \textcquote[225]{sartre}{Reflection, as a mode of being of the for-itself, must be in the form of temporalisation and that it is, itself, its past and its future.}
    \item \textcquote[255]{sartre}{That, by virtue of its nature, its authority and certainty extend as far as the possibilities that \emph{I am} and to the past that \emph{I was}.}
  \end{enumerate}
  \item \textcquote[226]{sartre}{Thus reflection is a consciousness \emph{of the three} ecstatic dimensions. It is a \emph{non-thetic consciousness} of flowing, and a \emph{thetic consciousness} of duration. For it, the past and the present of what it reflects on begin to exist as \emph{quasi-outsides}, in the sense that they are not only held within the unity of a for-itself that exhausts their being by having it to be, but also \emph{for} a for-itself that is separated from them by a nothingness.}
  \item \textcquote[227]{sartre}{Reflection, therefore, grasps temporality insofar as it is disclosed as the unique and incomparable mode of being of an ipseity, i.e., as historicity.}
\end{enumerate}

It is after having a good understanding of reflection do we come to understand original temporality, and then proceed to talk about pyschological temporality. I am very excited about Sartre's presentation of psychological temporality, because it is metaphysically interesting. Psychological temporality is based upon the original temporality that is a part of the being of the for-itself -- however, it is by neccessity derivative. I'm not sure if I understand the complete difference right now, but it seems like a pretty important concept behind psychological temporality is that our reflection on temporal states is in-complete -- there's some sort of \emph{nothingness} which seperates our being from the reflected being. In some sense, the objects of psycholgoical temporality are strictly beings \emph{in-itself}. Which does make sense, because whenever we conduct an act of reflection, we are naturally not creating a whole new consciousness from the mere act of reflection.

\begin{enumerate}
  \item \textcquote[228]{sartre}{This psychological duration constituted by the concrete flow of autonomous structures -- or, in other words, by the succession of psychological \emph{facts}, of \emph{facts} of consciousness -- cannot be called an illusion: \textbf{indeed their reality provides the object of psychology}; in practical terms, it is at the level of psychological fact that concrete relations between men \ldots\ are established.}
  \item \textcquote[229]{sartre}{We find ourselves therefore in the presence of two temporalities: original temporality, of which we \emph{are} the temporalisation; and psychological temporality, which exists at the same time both as something incompatible with our being's mode of being, and as an \emph{intersubjective reality}, an object of science, a goal of human actions.}
  \item \textcquote[229]{sartre}{At this point we need to distinguish pure reflection from impure or constituting reflection, \emph{because it is impure reflection that constitutes the succession of psychological facts, or psychè.} And in daily life, impure or constituting reflection is given first, even though it incorporates pure reflection within it as its original structure.}
  \item \textcquote[232]{sartre}{Thus, \textbf{reflection is impure when it presents itself as an \enquote{intuition of the for-itself in the in-itself;}} what is disclosed to it is not the temporal and insubstantial historicity of what it reflects on but -- beyond what it reflects on -- the actual substantiality of organised forms within the flow. The unity of these virtual beings is called \emph{a psychological life,} or the \emph{psychè,} a virtual and transcendent in-itself that subtends the for-itself]s temporalisation. Pure reflection is only ever a quasi-knowledge, but the only thing of which reflective knowledge is possible is the \emph{Psychè}.}
\end{enumerate}

\noindent
It is this degree of \emph{objectivity} which creates a whole new \emph{psychological world}, which contains its own facts and happenings.

\begin{enumerate}
  \item \textcquote[233]{sartre}{We should understand \emph{acts} as all of a person's synthetic activity, i.e., every ordering of means in view of ends -- not insofar as the for-itself is its own possibilities but insofar as the act represents a transcendent psychological synthesis that the for-itself is obliged to live.}
  \begin{enumerate}
    \item \textcquote[235]{sartre}{In short, the only way of presentifying these qualities, states, or acts is by apprehending them through a reflected consciousness, whose shadow they cast into the in-itself, in which they are objectified.}
    \item \textcquote[236]{sartre}{The psychological object, as the shadow cast by the reflected for-itself, possesses the characteristics of consciousness in degraded form.}
  \end{enumerate}
  \item \textcquote[233]{sartre}{The term \enquote{psychological} applies exclusively to a special category of  cognitive acts: the acts of the reflective for-itself.}
  \item \textcquote[242]{sartre}{In this way, reflective consciousness is constituted as a consciousness \emph{of} duration, and in consequence, psychological duration appears to consciousness. This psychological temporality, as the projection of original temporality into the in-itself, is a virtual being whose phantom flowing endlessly accompanies the for-itself's ecstatic temporalisation, insofar as reflection grasps this latter.}
  \item \textcquote[243]{sartre}{As soon as one takes up the standpoint of impure reflection -- the kind of reflection that seeks to determine the being that I am -- an entire world appears, to populate this temporality. This world -- a virtual presence, and the probable object of my reflective intention -- is the psychological world, or \emph{psychè}.}
  \marginnote{What is the relationship and neccessity of there being an \enquote{Other,} which allows the actualisation of an \enquote{Outside?}}
  \item \textcquote[243]{sartre}{And with this transcendent world, which takes up residence in the infinite becoming of antihistorical indifference, the temporality that we refer to as \enquote{internal} or \enquote{qualitative} -- which is an objectification of original temporality into in-itself -- is constituted, precisely, as a virtual unity of being. Here we find the first draft of an \enquote{outside} on itself, in its own eyes: but this \enquote{outside} is purely virtual. Later on we will see how being-for-the-Other \emph{actualises} the first draft of this \enquote{outside.}}
\end{enumerate}

\subsection{Chapter 3: Transcendence}

\subsubsection{Knowledge as a Type of Relation Between the For-Itself and the In-Itself}

I'm afraid my understanding of this section is poor, but at least rhetorically it seems like Sartre's argument consists of two sections. He first begins with a brief discussion on the question of \emph{knowledge}, asking what is the ontological foundation behind the being of knowledge. He spends the first section defining what knowledge is, and concludes that knowledge is strictly intuition:

\begin{enumerate}
  \item \textcquote[246]{sartre}{The only kind of knowledge is intuitive. Deduction and discourse, which are incorrectly labeled as \enquote{knowledge,} are only instruments leading to intuition.}
  \marginnote{What a delightful epistemology! I really should explore the idea of knowledge as intuition more -- in particular, Sartre's description of the ways we reach, or don't reach intuition are very interesting.}
  \begin{enumerate}
    \item \textquote{When we reach the intuition, the means that were used to reach it are set aside;}
    \item \textquote{In cases where we cannot reach it, reasoning and discourse are left in the position of signposts that point toward an intuition that is out of reach.}
    \item \textcquote[246]{sartre}{Finally, if an intuition has been reached but is not a present mode of my consciousness, the maxims that I employ persist, as the results of operations that I carried out earlier.}
  \end{enumerate}
  \item \textcquote[246]{sartre}{Intuition is the presence of consciousness to the thing [object of intuition].}
\end{enumerate}

It seems like knowledge has to be a relationship of some kind between the \emph{being in-itself} and the \emph{being for-itself.} He elaborates that this relationship is one of \emph{presence}, but beings that are simply in-itself (i.e. inanimate objects) cannot be present to anything. Hence, knowledge is the presence of the \emph{being-for-itself} to the \emph{being in itself}. We are present to the object of our knowledge. With this understanding of knowledge established, we begin asking the ontological questions between knowledge and being:

\begin{enumerate}
  \item \textcquote[247]{sartre}{What might this neccessity -- that consciousness should be conscious \emph{of} something -- mean, if we consider it at the level of ontology, i.e., within the perspective of being-for-itself?}
  \marginnote{Sartre proceeds to talk about negation in specific detail.}
  \item \textcquote[247]{sartre}{\emph{Not-being} is an essential structure of presence. Presence includes a radical negation, as presence to something that we are not. What is present to me is not me.}
  \item \textcquote[249]{sartre}{We need, in fact, to distinguish between two types of negation: external \emph{negation} and \emph{internal negation.}}
  \begin{enumerate}
    \item \textquote{[External negation] appears as a purely external connection between two beings, established by a witness. For example, when I say \enquote{the cup is not the inkwell,}  it is quite clear that the foundation of this negation lies neither in the cup nor in the inkwell.}
    \item \textquote{[Internal negation] is a particular negative property of my being, which characterises me from inside, and also that -- \emph{qua negativity} -- is a real quality of myself \ldots\ By \enquote{internal negation} we mean a relation between two beings, where the one that we deny in relation to the other qualifies the other, at the heart of its essence, precisely by its absence.}
  \end{enumerate}
  \item \textcquote[251]{sartre}{In this sense internal negation is a concrete ontological connection \ldots\ in internal negation, the for-itself is crushed against what it negates.}
  \item \textcquote[253]{sartre}{The for-itself's presence to the in-itself, which we cannot describe either in terms of continuity or in terms of discontinuity, is purely a \emph{negated identity.}}
\end{enumerate}

\noindent
So somehow, knowledge is a process of negation, specifically of internal negation. This seems to be understandable as far as how judgement (in the Kantian sense) is an act of discernment, but I'm still not very sure on my understanding of it right now.

\begin{enumerate}
  \item \textcquote[255]{sartre}{That is why the best term to use, to signify this internal relation of knowing and being, is the verb we were using just now: \enquote{\emph{to realize,}} with its double meaning, ontological and gnostic.}
  \item \textcquote[255]{sartre}{We can give the name \enquote{transcendence} to this internal and realising negation which discloses the in-itself by determining the for-itself in its being.}
\end{enumerate}

\subsubsection{On Determination as Negation}

\begin{enumerate}
  \item To be completed later.
\end{enumerate}

\subsubsection{Quality and Quantity, Potentiality and Equipmentality}

\begin{enumerate}
  \item To be completed later.
\end{enumerate}

\subsubsection{World-Time}

\begin{enumerate}
  \item To be completed later.
\end{enumerate}

\subsubsection{Knowledge}

\begin{enumerate}
  \item To be completed later.
\end{enumerate}
